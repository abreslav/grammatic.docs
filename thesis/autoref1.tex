\afsubsection{В первой главе} сообщаются предварительные сведения об инженерии языков и приводится обзор литературы. В частности, описываются основные понятия концепции \term{мета-моделирования}; изложение базируется на мета-мета-модели \tool{Ecore}, входящей в библиотеку \tool{EMF}. Также показывается, что для большинства языков можно выделить \term{целевые мета-модели}, описывающие их абстрактный синтаксис\footnote{В этом случае абстрактный синтаксис представляется в виде графов, и традиционные абстрактные синтаксические деревья являются частным случаем этого подхода.}. Для удобства дальнейшего использования вводится текстовая нотация для моделей и приводятся правила систем типов, отражающих требования мета-моделей. 

Далее приводится обзор разновидностей конкретного синтаксиса предметно-ориентированных языков. Рассматривается текстовый синтаксис на основе контекстно-свободных грамматик и на основе XML, а также основные виды графического синтаксиса: диаграммы, древовидное представление моделей и псевдо-текстовый синтаксис. Выделяются сходства и различия данных видов синтаксиса; в последующих частях работы рассматривается в основном текстовый синтаксис, важность которого обусловлена универсальностью и сравнительно низкими требованиями к инструментальной поддержке.

Приводится обзор средств разработки текстового синтаксиса, причем основное внимание уделяется механизмам композиции, использованным в соответствующих инструментах: модулям, параметризованным конструкциям (шаблонам и макроопределениям) и аспектам. Проводится сравнительный анализ реализаций этих механизмов, позволяющий сделать вывод о том, что поддержка более сложных из них в большинстве инструментов ограничена и нуждается в расширении.

Также приводится обзор средств автоматизации разработки самих механизмов композиции. Их сравнительный анализ показывает, что подходы, позволяющие автоматически расширять текстовые языки поддержкой механизмов композиции, нуждаются в дальнейшем улучшении.