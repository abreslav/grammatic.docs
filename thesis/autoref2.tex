\afsubsection{Во второй главе} описывается предметно-ориентированный язык \GRM{}, предназначенный для описания текстового синтаксиса и поддерживающий композицию с помощью модулей, шаблонов и аспектов. 
Этот язык является обобщенной
расширяемой нотацией для контекстно-свободных грамматик, которую можно применять при разработке различных инструментов. Реализация данного языка представляет собой библиотеку, позволяющую транслировать текстовые описания грамматик во внутреннее объектно-ориентированное представление на основе библиотеки \tool{EMF}, к которому имеется открытый программный интерфейс. 
Различные инструменты, такие как генераторы трансляторов и других программных средств, анализаторы грамматик и т.д., могут использовать данный язык как единый формат для ввода данных, что позволит разработчикам сосредоточиться на реализации функциональности, специфичной для их задачи. 

Язык \GRM{} использует расширенный набор операций для описания контекстно-свободных правил, соответствующий стандарту EBNF, что позволяет в той же нотации описывать и лексические анализаторы. Поскольку многим инструментам, кроме правил грамматики, требуется дополнительная информация, такая как код семантических действий или приоритеты бинарных операций, \GRM{} позволяет оснащать любые элементы грамматики \term{метаданными} в виде аннотаций, которые не интерпретируются самой библиотекой и просто преобразуются во внутреннее представление для дальнейшего использования. Аннотации состоят из пар ``имя-значение'', где значения могут иметь различные типы, в том числе и определяемые пользователем. Механизм метаданных позволяет использовать предложенный язык для решения широкого круга задач.

Основным достоинством \GRM{} является наличие поддержки шаблонов и аспектов.
\term{Шаблоны} (типизированные макроопределения) представляют собой фрагменты грамматик, содержащие формальные параметры. Результатом применения шаблона является подстановка конкретных элементов грамматики вместо формальных параметров. Такой механизм позволяет единообразно описывать конструкции, часто используемые в грамматиках разных языков. Например, списки с разделителями описываются следующим шаблоном:
\begin{lstlisting}
template List<element : Expression, sep : Expression> : Expression {
	<?element> (<?sep> <?element>)*
}
\end{lstlisting}
В угловых скобках после имени \code{List} указаны формальные параметры с типами (типы обычно можно не указывать, здесь они приведены для наглядности), в фигурных скобках записано тело шаблона. 
Шаблон можно использовать, указав в угловых скобках его имя и аргументы, занимающие места формальных параметров, например, выражение \code{<List ID, ','>} будет развернуто в \code{ID (',' ID)*}. Результатом применения шаблона может быть не только грамматическое выражение, но и любая другая конструкция языка \GRM{}, например, аннотация или одно или несколько правил грамматики. Это позволяет реализовать с помощью шаблонов концепцию \term{параметрических модулей}. 

\term{Аспекты} в языке \GRM{} реализуются следующим образом:
\begin{itemize}
\item \term{Точками встраивания} (join points) являются все элементы языка \GRM{}, такие как символы, различные выражения, продукции и аннотации.
\item \term{Срезы} (point-cuts), обеспечивающие квантификацию (quantification), описываются в виде образцов (patterns), использующих переменные, с которыми связываются конкретные фрагменты сопоставляемых выражений. Также используются \term{подстановочные знаки}, соответствующие произвольным конструкциям одного определенного типа (например, произвольным выражениям или произвольным символам).
\item \term{Советы} (advice), описывающие изменения, привносимые в точках встраивания, позволяют встраивать результаты разворачивания данных шаблонных выражений до, после или вместо фрагментов точек встраивания, а также добавлять аннотации к элементам грамматики.
\end{itemize}
Приведем пример \term{аспектного правила}, использующего данные понятия:
\begin{lstlisting}
example : .. e=example  .. ;    // Срез
	instead ?e : '(' <?e> ')' ; // Совет
\end{lstlisting}
Данный пример заменяет рекурсивное вхождение символа \code{example} в правой части на тот же символ, заключенный в скобки.

С помощью аспектов достигается повторное использование  и расширение спецификаций, в которых не предусмотрены такие возможности (например, эти спецификации не являются шаблонными). Кроме того, этот механизм позволяет разделять спецификацию на фрагменты с различным назначением, например, отделять метаданные для различных генераторов (в частности, семантические действия) от грамматических правил, значительно улучшая читаемость и модульность спецификаций.

В \tabref{GrmTable} приводятся результаты сравнения механизмов композиции \GRM{} с соответствующими механизмами в других инструментах.
\begin{table}[htb]
	\centering
\newcommand{\dissonly}[1]{}
\newcommand{\hd}[1]{{\begin{sideways}\parbox{25mm}{\tool{#1}}\end{sideways}}}
%{\begin{sideways}\parbox{15mm}{\bf text}\end{sideways}}
{\size{}\noindent
\begin{tabular}{|ll|c|c|c|c|c|c|c|c|c|c|c|c|}
\hline 
&
	&\hd{Yacc/Bison}
	&\hd{ANTLR}
	&\hd{Menhir}
	&\hd{xText}
	&\hd{ASF+SDF}
	&\hd{JastAdd}
	&\hd{AspectG}
	&\hd{Silver}
	&\hd{LISA}
	&\hd{Rats!}
	&\hd{\textbf{Grammatic}}\\
\hline
\dissonly{
\multicolumn{2}{|r|}{\it Ссылка}
	&\cite{BisonBook}
	&\cite{ANTLR}
	&\cite{Menhir}
	&\cite{xText}
	&\cite{ASF+SDF}
	&\cite{JastAdd}
	&\cite{AspectG}
	&\cite{Silver}
	&\cite{LISA}
	&\cite{Rats!}
	&\\
\hline 
}
\dissonly{\multicolumn{13}{|l|}{Модули}\\%&&&&&&&&&&&&&\\
&Разделение на файлы&-&+&+&+&+&+&+&+&+&+&+\\
&Атрибуты доступа&-&-&+&-&+&+&-&+&+&+&+\\}
%&&&&&&&&&&&&&\\
\multicolumn{13}{|l|}{Шаблоны грамматик. Параметры:}\\%&&&&&&&&&&&&&\\
&символы&-&-&-&-&+&-&-&-&-&-&+\\
&модули&-&-&-&-&-&-&-&-&-&+&-\\
&произвольные выражения&-&-&-&-&-&-&-&-&-&-&+\\
%&&&&&&&&&&&&&\\
\multicolumn{13}{|l|}{Шаблоны выражений. Параметры:}\\%&&&&&&&&&&&&&\\
&символы&-&-&+&-&-&-&-&-&-&-&+\\
&произвольные выражения&-&-&-&-&-&-&-&-&-&-&+\\
%&&&&&&&&&&&&&\\
\multicolumn{2}{|l|}{Шаблоны сем. действий}&-&-&-&-&-&-&-&-&+&-&+\\
%&&&&&&&&&&&&&\\
\multicolumn{13}{|l|}{Аспекты}\\%&&&&&&&&&&&&&\\
&Незнание&-&-&-&-&-&+&+&+&+&-&+\\
&Квантификация&-&-&-&-&-&-&+&+&+&-&+\\
%&&&&&&&&&&&&&\\
\multicolumn{13}{|l|}{Гибкость при повторном использовании}\\%&&&&&&&&&&&&&\\
&Добавление продукций&-&+&+&+&+&+&+&+&+&+&+\\
&Замена элементов&-&+&-&+&-&-&-&-&-&+&+\\
&Удаление элементов&-&-&-&-&-&-&-&-&-&+&+\\
&Изменение метаданных&-&-&-&+&+&+&+&+&+&+&+\\
\dissonly{
\hline
\multicolumn{2}{|r|}{Cумма}&0&3&4&4&5&5&5&6&7&7&13\\}
\hline
\end{tabular}
}
	\caption{Сравнение \GRM{} с существующими инструментами}\label{GrmTable}
\end{table}
Из таблицы видно, что \GRM{} поддерживает шаблоны и аспекты значительно лучше других инструментов. Это делает целесообразным использование данного языка для описания грамматик. Существующие проекты, в свою очередь, реализуют сложную функциональность, связанную с автоматизацией построения трансляторов и других инструментов. \GRM{} достаточно легко интегрируется с подобными инструментами, что позволяет пользоваться как преимуществами шаблонов и аспектов, так и уже наработанными достижениями в других областях. 

Предложенный язык был использован для декларативного описания компонент интегрированных сред разработки (подсветки синтаксиса и автоматического форматирования кода), для описания синтаксиса самого \GRM{}, а также семейства диалектов языка \tool{SQL92}. В последнем случае механизм аспектов позволил сделать описание очень кратким, поскольку для описания диалектов достаточно обозначить его отличия от стандартного варианта языка.