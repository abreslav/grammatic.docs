\afsubsection{Актуальность темы.}
Развитие технологии программирования исторически идет по пути повышения уровня абстракции поддерживаемого инструментами, используемыми при разработке программного обеспечения~(ПО). Решающим шагом на этом пути явилось создание языков программирования высокого уровня, позволяющих лишь в небольшой степени заботиться об особенностях конкретной аппаратной архитектуры. Повышение уровня абстракции --- один из ключевых факторов, определяющих сокращение сроков создания программных средств, поскольку высокоуровневые инструменты позволяют избегать определенных типов ошибок и повторно использовать разработанные решения, а также облегчают командную разработку.

С развитием языков высокого уровня неразрывно связан процесс развития автоматизированных инструментов разработки трансляторов, базирующийся на достижениях теории формальных языков и грамматик, в частности, алгоритмах преобразования контекстно-свободных грамматик в магазинные автоматы и формализация семантики с помощью атрибутных грамматик.

Дальнейшее повышение уровня абстракции привело к возникновению идеи \term{предметно-ориентированных языков} (ПОЯ), предназначенных для решения задач в относительно узкой предметной области и нередко непригодных за ее пределами. ПОЯ противопоставляются языкам общего назначения, являющимся вычислительно универсальными и позволяющим решать любые задачи. Основной мотивацией к разработке и использованию ПОЯ является тот факт, что моделирование предметной области в языках общего назначения часто бывает недостаточно явным, что приводит к большим объемам кода и затруднениям при чтении программ. При использовании ПОЯ эта проблема снимается, поскольку такие языки оперируют непосредственно понятиями предметной области и могут даже позволить специалистам в этой области, не имеющим квалификации разработчиков ПО, принимать участие в написании программ. В настоящее время ПОЯ применяются во множестве областей, начиная с систем управления базами данных и заканчивая системами моделирования бизнес-процессов.

Использование ПОЯ снижает затраты на разработку ПО в данной предметной области, но разработка самих ПОЯ также требует затрат. Если эти затраты высоки, то использование ПОЯ может быть нецелесообразным, поэтому возникает задача автоматизации разработки таких языков с целью минимизировать затраты на их создание и поддержку. Традиционные средства разработки трансляторов не обеспечивают необходимый уровень автоматизации, поэтому разрабатываются новые подходы и инструменты, позволяющие быстро разрабатывать небольшие языки с поддержкой все более сложных механизмов. В частности, существенный интерес представляют механизмы композиции, обеспечивающие повторное использование кода, написанного на ПОЯ. %Эти механизмы достаточно сложны и реализация их вручную требует существенных затрат, поэтому возникает необходимость в автоматизации решения этой задачи.
%Причем важно, чтобы получающиеся языки были удобны в использовании, иначе выгода от повышения уровня абстракции может быть сведена на нет неудобством языка. Общая цель использования ПОЯ --- повысить качество ПО, поэтому спецификации, написанные с использованием ПОЯ должны сами отвечать таким критериям качества как модульность, повторное использование, которые обеспечиваются механизмами (де-)композиции ПО, такими как модули, полиморфизм и аспекты. Для того, чтобы ПОЯ это могли, нужно, чтобы автоматизированные средства разработки позволяли поддерживать эту функциональность без существенных затрат времени.

\afsubsection{Предметом исследования} являются механизмы композиции, пригодные для использования в предметно-ориентированных языках.

\afsubsection{Целью работы} является исследование и обоснование подходов и методов, позволяющих автоматически расширять предметно-ориентированные языки механизмами композиции, поддерживающими повторное использование.

\afsubsection{Задачи исследования.} Достижение поставленной цели подразумевает решение следующих задач:
\begin{itemize}
%\item Сравнительный анализ механизмов композиции, используемых в современных предметно-ориентированных языках, с целью обоснования требований к средствам автоматизации.
\item Проектирование и реализация предметно-ориентированного языка для хорошо изученной области --- описания текстового синтаксиса искусственных языков --- поддерживающего все основные механизмы композиции в полном объеме, с целью выявления связей между этими механизмами и их характерных особенностей, влияющих на подходы к автоматизации.
\item Проверка адекватности разработанного языка нуждам конечных пользователей на примере описания синтаксиса сложных языков.
\item Обобщение рассмотренных механизмов композиции в виде формализованных языковых конструкций. Описание их семантики и соответствующих систем типов.
\item Разработка алгоритмов автоматического расширения языка механизмами композиции, обоснование их корректности.
\item Применение предложенного подхода к существующему предметно-ориентированному языку.
\end{itemize}

\afsubsection{Методы исследования} включают методы инженерии программного обеспечения, анализа алгоритмов и программ, аппарат теории типов, теории графов и теории формальных грамматик.

\afsubsection{Научная новизна} результатов работы состоит в том, что:
\begin{itemize}
\item Спроектирован и реализован предметно-ориентированный язык для описания текстового синтаксиса, поддерживающий композицию спецификаций с помощью модулей, шаблонов (типизированных макроопредений) и аспектов.
\item На основе указанного языка разработан генератор трансляторов, поддерживающий проверку типов в семантических действиях и гарантирующий отсутствие ошибок типизации в сгенерированном коде для многих языков реализации.
\item Предложена формализация механизмов композиции на основе шаблонов и аспектов, и доказаны свойства данной формализации, гарантирующие раннее обнаружение ошибок программиста при использовании этих механизмов.
\item Разработаны и апробированы алгоритмы, позволяющие автоматизировать расширение предметно-ориентированных языков механизмами композиции, основанными на шаблонах и аспектах.
\end{itemize}

\afsubsection{Практическую ценность} работы составляют:
\begin{itemize}
\item Разработанная библиотека, обеспечивающая трансляцию предложенного языка описания текстового синтаксиса.
\item Программные генераторы, использующие данную библиотеку, позволяющие автоматически получать трансляторы и компоненты интегрированной среды разработки.
\item Методика и алгоритмы расширения предметно-ориентированных языков механизмами композиции.
\end{itemize}

\afsubsection{На защиту выносятся следующие положения:} 
\begin{itemize}
\item Предметно-ориентированный язык для описания текстового синтаксиса, поддерживающий композицию спецификаций с помощью модулей, шаблонов и аспектов.
\item Подход к описанию и генерации трансляторов, позволяющий порождать код на нескольких языках программирования по одной спецификации, и гарантирующий отсутствие ошибок типизации в сгенерированном коде.
\item Метод автоматического расширения имеющихся описаний синтаксиса и семантики предметно-ориентированного языка таким образом, что результирующий язык поддерживает композицию с помощью шаблонов и аспектов.
\end{itemize}

\afsubsection{Достоверность научных результатов и выводов} обеспечивается формальной строгостью описания процесса композиции языков, обоснованностью применения математического аппарата, результатами тестирования алгоритмов и программного обеспечения.

\afsubsection{Внедрение результатов работы.} Результаты, полученные в ходе диссертационной работы, были использованы в компании OpenWay (Санкт-Петербург) при разработке предметно-ориентированного языка для написания отчетов, а также при выполнении НИОКР ``Технология разработки предметно-ориентированных языков'' по программе ``У.М.Н.И.К.'' Фонда содействия развитию малых форм предприятий в научно-технической сфере.

\afsubsection{Апробация работы.} Изложенные в диссертации результаты обсуждались на 11 российских и международных научных конференциях, семинарах и школах, включая V, VI и VII всероссийские межвузовские научные конференции молодых ученых (2008, 2009 и 2010~гг., Санкт-Петербург), международную научную конференцию ``Компьютерные науки и информационный технологии'' (2009 г., Саратов), международные научные школы ``Generative and Transformational Techniques in Software Engineering'' (2009~г., Брага, Португалия), ``Aspect-Oriented Software Development'' (2009~г., Нант, Франция) и ``15$^{th}$ Estonian Winter School in Computer Science'' (2010~г., Палмсе, Эстония), а также международные семинары ``Teooriapäevad'' (2009 и 2010~гг., Эстония), семинар Лаборатории математической логики и семантики языков программирования Научно-исследовательского института кибернетики Эстонской Академии наук (2009~г., Таллинн, Эстония) и научном семинаре ``Computer Science Клуба'' при ПОМИ РАН (2009~г., Санкт-Петербург).

\afsubsection{Публикации.} По теме диссертации опубликовано пять печатных работ (из них две статьи --- в изданиях, соответствующих требованиям ВАК РФ к кандидатским диссертациям по данной специальности).

\afsubsection{Структура и объем работы.} Диссертация состоит из введения, четырех глав, списка литературы (? наименований) и 3 приложений. Содержит ? с. текста (из них ? основного текста и ? --- приложений), включая ? рис. и табл.

