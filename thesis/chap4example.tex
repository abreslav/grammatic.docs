\chapter{Пример: расширение простого языка}

В качестве примера применения описанного выше метода в данном разделе описано расширение простого предметно-ориентированного языка \tool{Toy}. Данный язык позволяет описывать интерфейсы классов, например:
\begin{lstlisting}
class Iterator {
	fun hasNext() : Boolean;
	fun next() : Object;
}
class Collection {
	fun add(obj : Object) : Boolean;
	fun iterator() : Iterator;
}
\end{lstlisting}
Как видно из примера, классы имеют имена, а в теле содержат объявления функций. Грамматика данного языка, приведенная в \lstref{ToyGrammar} позволяет также указывать суперкласс после ключевого слова \code{extends}.

\begin{lstlisting}[float=htbp,label=ToyGrammar,caption=Грамматика языка \tool{Toy}]
specification : class*;
class
	: 'class' name ('extends' name)? '{'
		function*
	  '}'
	;
function
	: 'fun' name '(' <List parameter ','> ')' ':' type;
parameter 
	: name ':' type;
type 
	: name ('*' | '+' | '?')?;
name : NAME;
\end{lstlisting}

Целевая мета-модель данного языка приведена в \figref{ToyMM} (слева) в текстовой нотации. 
\begin{figure}[htbp]
	\centering
\begin{tabular}{b{.45\textwidth}b{.45\textwidth}}
%[float=htbp,label=ToyMM,caption=Целевая мета-модель языка \tool{Toy}]
\begin{lstlisting}[xleftmargin=0cm,framesep=0pt]
class Specification { 
	val classes : Class*; 
}
class Class {
	val name : Name;
	ref superclass : Class?;
	val functions : Function*;
}
class Function {
	val name : Name;
	val params : Parameter*;
	ref returnType : Class;
}
class Parameter {
	val name : Name;
	ref type : Class;
}
class Name { 
	attr name : String; 
}\end{lstlisting}%
&
\begin{lstlisting}[xleftmargin=0cm,framesep=0pt]
class SpecificationT extends Term { 
	val classes : Term*; 
}
class ClassT extends Term {
	val name : Term;
	ref superclass : Term?;
	val functions : Term*;
}
class FunctionT extends Term {
	val name : Term;
	val params : Term*;
	ref returnType : Term;
}
class ParameterT extends Term {
	val name : Term;
	ref type : Term;
}
class NameT extends Term { 
	attr name : Term; 
}\end{lstlisting}%
%\\
%\hline
%\begin{center}(а)\end{center}&
%\begin{center}(б)\end{center}\\
\end{tabular}
	\caption{Классы целевой мета-модели исходного языка и языка шаблонов }\label{ToyMM}
\end{figure}
Из названий классов видно, как они сопоставлены символам грамматики. На том же рисунке справа показан результат применения преобразования $\TC{\bullet}$ к классам целевой мета-модели. Согласно описанной выше процедуре, грамматика расширена, и к каждому правилу добавлена продукция ``\code{: term}''.
Соответствующее применение шаблона \code{Templates} (\lstref{TempG}) выглядит следующим образом:
\begin{lstlisting}
<Templates 
	('Specification' | 'Class' | 'Function' | 'Parameter' | 'Name'),
	(specification, class, function, parameter, name)>
\end{lstlisting}

Полученный язык позволяет записывать следующие шаблонные выражения, достаточно близкие к шаблонным классам \tool{C++}:
\begin{lstlisting}
template Iterator<N : String, T : Class> : Class {
	class <?N> {
		fun hasNext() : Boolean
		fun next() : <?T>
	}
}
template Collection<N : String, IN : String, T : Class> : Class {
	class <?N> {
		fun contains(object : <?T>) : Boolean
		fun add(object : <?T>) : Boolean
		fun iterator() : <Iterator <?IN> <?T>>
	}
}
\end{lstlisting}
Важным отличием от шаблонов \tool{C++} является необходимость указывать имена как параметры шаблонов. Эту проблему можно устранить, если добавить к процедуре разворачивания шаблонов стадию переименования, запрограммированную вручную. Добавление подобных функций до и после семантических операций, описанных выше, не представляет затруднений и гораздо проще, чем реализация всего механизма шаблонов полностью вручную.

Полученный при расширении языка \tool{Toy} механизм аспектов позволяет реализовать внешние декларации функций (аналогично Inter-Type Declarations в \tool{AspectJ}) следующим образом:
\begin{lstlisting}
on class <? : String> { ?funs=<? : Function*> } perform
	instead funs : <?funs>; fun toString() : String;
\end{lstlisting}
Данное аспектное правило добавляет в тело каждого класса функцию \code{toString()}, что демонстрирует возможности квантификации (все классы обрабатываются одним правилом) и незнания (классы не готовятся для расширения специальным образом и имеют смысле даже если аспект не применялся).