\section{Формализация языка шаблонов}

Для того, чтобы описать язык шаблонов более точно, мы приводим в этом разделе формальное описание его контекстно-свободного синтаксиса, системы типов и семантики. Чтобы не загромождать текст разбором однотипных случаев, мы упрощаем формализуемый язык, рассматривая только выражения (опуская определения символов и продукции), а среди них --- только операцию конкатенации и строковые литералы. Более полное описание вынесено в Приложение \bad{???}.

Контекстно-свободная грамматика упрощенного языка шаблонов приведена в \lstref{TempG}. В этой грамматике упоминаются только типы, соответствующие рассматриваемым нами конструкциям языка (\texttt{Sequence} для конкатенации и \texttt{Literal} для строковых литералов). Аппарат шаблонов описан в полном объеме: объявления и применения шаблонов и использование параметров.

\begin{lstlisting}[float=htbp,label=TempG,caption=Грамматика упрощенного языка шаблонов]
templateDef
	: 'template' NAME '<' 
			List<NAME type?, ','> 
	  	'>' type? '{' expression '}'
	;
type : ':' ('Expression' | 'Literal' | 'Sequence') ;
expression
	: expression expression
	: STRING
	: templateParameterRef
	: templateApplication
	;
templateParameterRef : '<' NAME '>'	;
templateApplication	: '<' NAME List<expression, ','> '>' ;
\end{lstlisting}

Для того, чтобы гарантировать, что результат применения шаблона будет корректным с точки зрения целевой мета-модели, мы определяем систему типов для языка шаблонов. Правила этой системы типов приведены на \figref{TempTypes}. Требования целевой мета-модели моделируются с помощью отношения ``подтип'', обозначаемого следующим образом:
$$
	\mbox{подтип} \preceq \mbox{супертип}.
$$
В остальном система типов выражена традиционным способом (см. \cite{???}). Слева от символа выводимости $\vdash$ пишется \term{контекст} (чаще всего обозначаемый $\Gamma$. В нашем случае контекст представляет собой множество утверждений вида $e : \tau$, где $e$ --- шаблонное выражение, а $\tau$ --- тип; такие утверждения читаются как ``\term{$e$ имеет тип $\tau$}''. 

\begin{figure}[htbp]
$$
\begin{array}{ccc}
\trule{}{\mathtt{Literal} \preceq \mathtt{Expression}}{lit-sub}
&&
\trule{}{\mathtt{Sequence} \preceq \mathtt{Expression}}{seq-sub}
\\
&&\,\\
\trule{}{STRING : \mathtt{Literal}}{lit-t}
&\quad&
\trule{}{\Gamma \cup \{p : \tau\} \vdash <p> : \tau}{par-t}
\\
&&\,\\
%
\trule{
	\Gamma \vdash e_1 : \mathtt{Expression} 
	\quad 
	\Gamma \vdash e_2 : \mathtt{Expression}
}{
	\Gamma \vdash e_1 \, e_2 : \mathtt{Sequence}
}{seq-t}
&&
\trule{
	\Gamma \vdash e : \tau
	\quad 
	\tau \preceq \sigma 
}{
	\Gamma \vdash e : \sigma
}{subtype}
\\
&&\,\\
\multicolumn{3}{c}{
\trule{
	\Gamma \cup \bigcup\limits_{i=1}^{n} \{ p_i : \tau_i \} \vdash b : \sigma
}{
	\Gamma \vdash \mathbf{template}\left(
		T <p_1 : \tau_1, \ldots, p_n : \tau_n> \,: \sigma \, \{ b \}
	\right)
}{temp-def}
}
\\
&&\,\\
\multicolumn{3}{c}{
\trule{
	\Gamma \vdash \mathbf{template}\left(
		T <p_1 : \tau_1, \ldots, p_n : \tau_n> \,: \sigma \, \{ b \}
	\right)
	\quad
	\Gamma \vdash a_1 : \tau_1 \, \cdots \, \Gamma \vdash a_n : \tau_n
}{
	\Gamma \vdash <T \, a_1, \ldots, a_n> : \sigma
}{app-t}
}
\end{array}
$$
	\caption{Система типов упрощенного языка шаблонов}\label{TempTypes}
\end{figure}

\newcommand{\Inst}[2]{\mathcal{I}_{#1} \left[ #2 \right]}%
Семантика языка шаблонов задается операцией \term{разворачивания}, описанной в композиционном стиле правилами на \figref{TempSem}. Мы обозначаем результат разворачивания шаблонов в выражении $e$ как $\Inst{\gamma}{e}$, где $\gamma$ (``\term{среда}'') является множеством значений параметров шаблонов вида $p = e$, где $p$ --- параметр, а $e$ --- шаблонное выражение. Как видно из правила \textsc{APP}, когда разворачивается применение шаблона, среда пополняется информацией о текущих значениях параметров, при этом само значение не разворачивается до тех пор, пока оно не будет использовано. Разворачивание значений использованных параметров выполняется в правиле \textsc{PAR}.

\begin{figure}[htbp]
$$
\begin{array}{ccc}
\trule{}{\Inst{\gamma}{STRING} = STRING }{str}
&\quad&
\trule{}{
	\Inst{\gamma}{e_1 \, e_2} = 
		\Inst{\gamma}{e_1} \, \Inst{\gamma}{e_2} 
}{seq}
\\
&&\,\\
\multicolumn{3}{c}{
\trule{
	\{p = e\} \subseteq \gamma
}{
	\Inst{\gamma}{<p>} = \Inst{\gamma}{e}
}{par}
}
\\
&&\,\\
\multicolumn{3}{c}{
\trule{
	\mathbf{template}\left(
		T \, <p_1, \ldots, p_n> \, \{ b \}
	\right)
}{
	\gamma' = \bigcup\limits_{i=1}^{n} \{ p_i = a_i \}
	\quad	
	\Inst{\gamma}{<T \, a_1, \ldots, a_n>} = \Inst{\gamma \cup \gamma'}{b}
}{app}
}
\end{array}
$$
	\caption{Семантика упрощенного языка шаблонов}\label{TempSem}
\end{figure}

Для того, чтобы убедиться, что данная система типов согласуется с приведенной семантикой, нужно показать, что выполняются свойства \term{сохранения типов} и \term{нормализации} \cite{???}. Оба эти свойства выполняются, если функция разворачивания получает на вход выражение, в котором соблюдаются правила системы типов. Это условие формализовано в следующем определении.

\begin{Def}\label{agree}
Среда $\gamma$ называется \term{согласованной с контекстом} $\Gamma$, если все ее элементы имеют допустимые типы:
$$
	\forall p \, : \, 
		\{p = e\} \subseteq \gamma 
			\Rightarrow 
		\left\{\begin{array}{l}		
		\{p : \tau\} \subseteq \Gamma \\
		\Gamma \vdash e : \tau
		\end{array}\right.
$$
\end{Def}

Теперь докажем первое из упомянутых выше свойств.

\begin{Lemm}
Если среда $\gamma$ согласована с контекстом $\Gamma$ и $\Gamma \vdash e : \tau$, то $\Gamma \vdash \Inst{\gamma}{e} : \tau$. Другими словами, преобразование $\Inst{}{\bullet}$ сохраняет типы.
\end{Lemm}
\begin{proof}
Будем вести индукцию по структуре выражения $e$, пользуясь правилами на \figref{TempSem}.\\
\textbf{База}. Правила \textsc{STR} и \textsc{SEQ} сохраняют типы.\\
\textbf{Предположение}. Пусть для выражений $e, b, a_1, \ldots, a_n$ данная лемма выполняется.\\
\textbf{Переход}. Рассмотрим правило \textsc{PAR}. Поскольку $\{p = e\} \subseteq \gamma$, по условию леммы выполняется
$$
\left\{\begin{array}{l}		
		\{p : \tau\} \subseteq \Gamma \\
		\Gamma \vdash e : \tau
		\end{array}\right.
$$
(см. Определение \ref{agree}). Следовательно $\Gamma \vdash (\Inst{\gamma}{<p>} = \Inst{\gamma}{e}) : \tau$.\\
Теперь рассмотрим правило \textsc{APP}. Пусть $$\Gamma \vdash <T a_1, \ldots, a_n> : \sigma,$$ тогда по условию леммы $\Gamma \vdash b : \sigma$, то есть $$\Gamma \vdash (\Inst{\gamma}{<T a_1, \ldots, a_n>} = \Inst{\gamma}{b}) : \sigma.$$
\end{proof}

Второе свойство --- нормализация --- формулируется следующим образом.

\begin{Lemm}
Если среда $\gamma$ согласована с контекстом $\Gamma$ и $\Gamma \vdash e : \tau$, то вычисление $\Inst{\gamma}{e}$ по правилам \figref{TempSem} заканчивается за конечное количество шагов, и результат не содержит применений шаблонов и шаблонных параметров.
\end{Lemm}
\begin{proof}
Снова будем вести индукцию по структуре выражения, пользуясь правилами на \figref{TempSem}.\\
\textbf{База}. Вычисление по правилу \textsc{STR} заканчивается за один шаг и результат не содержит применений шаблонов и шаблонных параметров.\\
\textbf{Предположение}. Пусть данная лемма выполняется для выражений $e, b, e_1, e_2, a_1, \ldots, a_n$.\\
\textbf{Переход}. Рассмотрим правило \textsc{PAR}. По предположению вычисление заканчивается за конечное количество шагов (время для $e$ плюс один шаг). Поскольку результат рекурсивного вызова не содержит применений шаблонов и шаблонных параметров, а правило их не добавляет (наоборот, удаляет параметр), то и результат их не содержит.\\
Для правил \textsc{SEQ} и \textsc{APP} рассуждения аналогичны с той лишь разницей, что в случае \textsc{APP} нужно учитывать, что рекурсия в шаблонах не допускается.
\end{proof}

Мы показали, что функция $\Inst{}{\bullet}$ корректно разворачивает все шаблонные параметры и применения шаблонов, а также что она не нарушает структурной корректности с точки зрения мета-модели. Результатом применения этой функции (в пустой среде) всегда является константное шаблонное выражение, которое, фактически, является предложением в нотации, не имеющей шаблонов. Таким образом, описанный здесь язык шаблонов работает корректно.

%%%%%%%%%%%%%%%%%%%%%%%%%%%%%%%%%%%
