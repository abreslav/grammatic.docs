\chapter{Автоматическое построение языков, поддерживающих аспекты}

общая мета-модель аспектов

\begin{lstlisting}
class Aspect {
	rules : AspectRule*;
}
class AspectRule {
	pattern : Abstraction;
	variablePatterns : Term*;
	substitutions : SubstitutionRule*;
}
class SubstitutionRule {
	variable : Variable;
	putInstead : Term;
}
\end{lstlisting}

общий синтаксис
 -- добавить wildcards:
\begin{lstlisting}
term : '<' '?' n=NAME type? '>'
	instead n : n?;
\end{lstlisting}

 -- добавить bindings !!! 
\begin{lstlisting}
aspect : aspectRule*;
aspectRule : 'on' aspectTerm 'perform' substRule* ;
aspectTerm : 
substRule : 'instead' NAME ':' term ;
\end{lstlisting}

специализированная мета-модель образцов (получение из шаблонов)

   Ничего не надо

алгоритм сопоставления с образцом

%\newcommand{\P}[1]{\mathcal{P}\left(#1\right)}
-- функция возвращает ассоциированный с переменной терм
$$
\begin{array}{c}
\infer[wcard]{
	(\match{x}{\wcard{?var}{\tau}}) = \meitem{var}{x}
}{RTT(x) = \tau}
\\
\infer[var]{
	(\match{x}{?var = P_x}) = \ME_x \mejoin \meitem{var}{x}
}{
\match{x}{P_x} = \ME_x
}
\\
\infer[mvar]{
	(\match{x}{?var}) = \ME \mejoin \meitem{var}{x}
}{
\meflatten{\ME}(?var) \cong x
}
\end{array}
$$

типы для аспектов
	

доказательство структурной корректности

\chapter{Пример: Автоматическое расширение языка Emfatic}

шаблоны

аспекты

\chapter{Сравнение предложенного подхода с существующими}

незнание в аспектах

локальность в шаблонах
