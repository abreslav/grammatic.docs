\documentclass{beamer}
\usepackage[utf8x]{inputenc}
\usepackage{ucs}
\usepackage[T2A]{fontenc}
\usepackage{cmap} 
\usepackage{cyrtimes} 
\usepackage{multirow}
\usepackage{rotating}
\usepackage{cmap}
\usepackage{url}
\usepackage{listings}
\usepackage{color}
\usepackage{proof}
\usepackage{amsthm}
\usepackage{amsmath}
\usepackage{amssymb}
\usepackage{array}
\usepackage{underscore}

\newtheorem{Th}{Теорема~}


\definecolor{Brown}{cmyk}{0,0.81,1,0.60}
\definecolor{OliveGreen}{cmyk}{0.64,0,0.95,0.40}
\definecolor{CadetBlue}{cmyk}{0.62,0.57,0.23,0}
\definecolor{MyDarkBlue}{rgb}{0,0.08,0.45} 


\newcommand{\tool}[1]{\textsc{#1}}
\newcommand{\GRM}{\textsc{Grammatic}}
\newcommand{\ATF}{\tool{Grammatic$^{SDT}$}}
\newcommand{\term}[1]{\emph{#1}}
\newcommand{\dissonly}[1]{}
\newcommand{\size}{\scriptsize}

\newcommand{\MM}[1]{\mathcal{#1}}%
\newcommand{\Inst}[2]{\mathcal{I}_{#1} \left[ #2 \right]}%
\newcommand{\match}[3][]{#2 \; \mathbf{match}_{#1} \; #3}
\newcommand{\matchNV}[3][]{#2 \; \overline{\mathbf{match}}_{#1} \; #3}
\newcommand{\ang}[1]{\mathsf{<}#1\mathsf{>}}
\newcommand{\wcard}[2]{\mathsf{<} #1 : #2 \mathsf{>} }
\newcommand{\ME}{\Upsilon}
\newcommand{\meitem}[2]{\left\{ #1 \mapsto #2 \right\}}
\newcommand{\meempty}{\left\{\right\}}
\DeclareMathOperator{\MEjoin}{\mbox{\Large$\uplus$\hspace{-2pt}}}
\DeclareMathOperator{\mejoin}{\uplus}
%\newcommand{\mejoin}{\uplus}
\newcommand{\mereplace}{\pitchfork}
\newcommand{\meflatten}[1]{\overline{#1}}
\newcommand{\subtype}{\preceq}
\newcommand{\suptype}{\succeq}
\newcommand{\myinfer}[3][]{\infer[\mbox{#1}]{#2}{#3}}
\newcommand{\matchList}[3]{\mathit{matchList}_{#1}(#2,\, #3)}
\newcommand{\matchSet}[3]{\mathit{matchSet}_{#1}(#2,\, #3)}
\newcommand{\trule}[3]{%
\myinfer[#3]{#2}{#1}%(\mbox{\textsc{#3}})%
}%
\newcommand{\rref}[1]{\mbox{\textit{#1}}}
\newcommand{\class}[4]{\mathbf{class}\; #1 \langle #2, #3, #4\rangle}%
\newcommand{\classf}[3]{\mathbf{class}\; #1 \langle #2, #3\rangle}%
\newcommand{\reference}[3]{\mathbf{#1}\langle #2 : #3\rangle}%
\newcommand{\attribute}[2]{\mathbf{attr}\langle #1 : #2\rangle}%
\newcommand{\type}[2]{#1\left(#2\right)}%
\newcommand{\valts}{\mathbf{val}}%
\newcommand{\valt}[1]{\type{\valts}{#1}}%
\newcommand{\refts}{\mathbf{ref}}%
\newcommand{\reft}[1]{\type{\refts}{#1}}%
\newcommand{\obj}[3]{#1@#2\left\{#3\right\}}%
\newcommand{\refv}[2]{\mathbf{ref}\; #1@#2}%
\newcommand{\subst}[2]{ #1 \mapsto #2 }
\newcommand{\apply}[2]{\left( #1 \right) \triangleright #2}


\lstdefinelanguage{Grammatic}
	{
		morestring=[b]',
		morekeywords={lex,empty,before,after,at,template},
		morecomment=[l]{//},
	}
\lstset{
	inputencoding=utf8x, 
	extendedchars=\true, 
	basicstyle=\ttfamily\scriptsize,
	keywordstyle=\sffamily\bfseries,
	stringstyle=\sffamily,
	commentstyle=\color{OliveGreen}\ttfamily,
	tabsize=4,
	captionpos=b,
	showstringspaces=false,
	xleftmargin=1cm,
	texcl,
	basewidth=.7em,
	language=Grammatic
}


\title{Механизмы композиции в предметно-ориентированных языках}
\author{Андрей Бреслав}
\date{\today}

\begin{document}
\newcommand{\ignore}[1]{}
\frame {\titlepage}


\frame{
\frametitle{Содержание работы}
\small
\textbf{Цель работы:} расширение предметно-ориентированных языков механизмами композиции, поддерживающими повторное использование.
\begin{itemize}
\item Язык \GRM{} 
	\begin{itemize}
		\item описание текстового синтаксиса;
		\item поддержка шаблонов и аспектов.
	\end{itemize}
\item Генератор трансляторов \ATF{}
	\begin{itemize}
		\item проверка типов в семантических действиях;
		\item генерация кода на многих языках реализации.
	\end{itemize}
\item Метод расширения языков механизмами композиции
	\begin{itemize}
		\item добавление шаблонов и аспектов;
		\item формальное описание семантики механизмов композиции;
		\item гарантии корректности с помощью системы типов.
	\end{itemize}
\end{itemize}
}

\frame{
	\frametitle{Язык \GRM{}}
	
\begin{itemize}
\item Операции EBNF
	\begin{itemize}
		\item описание лексической и синтаксической структуры
		\item строковые литералы в качестве нетерминалов
	\end{itemize}
\item Метаданные --- обобщенный механизм для представления аннотаций
	\begin{itemize}
		\item семантические действия
		\item автоматическое форматирование
		\item ...
	\end{itemize}
\item Конвертеры для интеграции с существующими инструментами
	\begin{itemize}
		\item пример: преобразование \GRM{} $\rightarrow$ ANTLR
		\item добавление к инструментам механизмов композиции
		\item метаданные описывают специфичную информацию
	\end{itemize}
\end{itemize}	
}

\begin{frame}[fragile]
	\frametitle{Шаблоны грамматических выражений}
Объявление шаблона:
\begin{lstlisting}
template List<
    item : Expression, 
    sep : Expression
  > : Expression {
  
    <?item> (<?sep> <?item>)* <?sep>?
    
}
\end{lstlisting}
Применение шаблона:
\begin{lstlisting}
<List INT, ','>
\end{lstlisting}
Результат применения:
\begin{lstlisting}
INT (',' INT)* ','?
\end{lstlisting}
\end{frame}

\begin{frame}[fragile]
	\frametitle{Параметризованные модули}
Шаблон грамматики:
\begin{lstlisting}[xleftmargin=0.3cm]
template SMMain<event : Expression, eventRef : Expression, 
        commandRef : Expression> : Symbol+ 
{
    system : <?event>* stateMachine;
    stateMachine : 'statemachine' NAME '{' state* '}';
    state : 'state' NAME '{' do? (transition ';')* '}';
    do : 'do' block;
    transition : 'on' <?eventRef> 'goto' stateRef;
    stateRef : NAME;
    block : '{' (<?commandRef> ';')* '}';
}
\end{lstlisting}
Применение шаблона:
\begin{lstlisting}[xleftmargin=0.3cm]
<SMMain event, eventRef, commandRef>
\end{lstlisting}
\end{frame}

\begin{frame}[fragile]
	\frametitle{Аспекты}
\begin{itemize}
\item Точки присоединения (join points) --- любые элементы грамматики
	\begin{itemize}
		\item символы, выражения, продукции...
	\end{itemize}
\item Срезы (point cuts) --- образцы, сопоставляемые с объектами в грамматике
	\begin{itemize}
		\item цитирование
		\item подстановочные знаки
		\item переменные
	\end{itemize}
\item Советы (advice)
	\begin{itemize}
		\item замена значения переменной (instead)
		\item добавление перед (before) или после (after) переменной
		\item удаление вхождений переменной (remove)
		\item присоединение метаданных (@)
	\end{itemize}
\end{itemize}	
\end{frame}

\begin{frame}[fragile]
	\frametitle{Примеры использования аспектов}
Замена:
\begin{lstlisting}[xleftmargin=0.3cm]
?i=.. (?sep=#lex ?i);
    instead sep : <?sep>+;
\end{lstlisting}
Вставка перед вхождением переменной:
\begin{lstlisting}[xleftmargin=0.3cm]
.. ?n=NUM ..;
    before n : '-'?;
\end{lstlisting}
Присоединение метаданных:
\begin{lstlisting}[xleftmargin=0.3cm]
example : .. ?e=('(' .. ')') .. ;
    @e : {a = 10} ;
\end{lstlisting}
\end{frame}

\begin{frame}[fragile]
	\frametitle{Сравнение с существующими подходами}
\newcommand{\hd}[1]{{\begin{sideways}\parbox{25mm}{\tool{#1}}\end{sideways}}}
%{\begin{sideways}\parbox{15mm}{\bf text}\end{sideways}}
{\small\noindent
\begin{tabular}{|ll|c|c|c|c|c|c|c|c|c|c|c|c|}
\hline 
&
	&\hd{Yacc/Bison}
	&\hd{ANTLR}
	&\hd{Menhir}
	&\hd{xText}
	&\hd{ASF+SDF}
	&\hd{JastAdd}
	&\hd{AspectG}
	&\hd{Silver}
	&\hd{LISA}
	&\hd{Rats!}
	&\hd{\textbf{Grammatic}}\\
\hline
\dissonly{
\multicolumn{2}{|r|}{\it Ссылка}
	&\cite{BisonBook}
	&\cite{ANTLR}
	&\cite{Menhir}
	&\cite{xText}
	&\cite{ASF+SDF}
	&\cite{JastAdd}
	&\cite{AspectG}
	&\cite{Silver}
	&\cite{LISA}
	&\cite{Rats!}
	&\\
\hline 
}
\dissonly{\multicolumn{13}{|l|}{Модули}\\%&&&&&&&&&&&&&\\
&Разделение на файлы&-&+&+&+&+&+&+&+&+&+&+\\
&Атрибуты доступа&-&-&+&-&+&+&-&+&+&+&+\\}
%&&&&&&&&&&&&&\\
\multicolumn{13}{|l|}{Шаблоны грамматик. Параметры:}\\%&&&&&&&&&&&&&\\
&символы&-&-&-&-&+&-&-&-&-&-&+\\
&модули&-&-&-&-&-&-&-&-&-&+&-\\
&произвольные выражения&-&-&-&-&-&-&-&-&-&-&+\\
%&&&&&&&&&&&&&\\
\multicolumn{13}{|l|}{Шаблоны выражений. Параметры:}\\%&&&&&&&&&&&&&\\
&символы&-&-&+&-&-&-&-&-&-&-&+\\
&произвольные выражения&-&-&-&-&-&-&-&-&-&-&+\\
%&&&&&&&&&&&&&\\
\multicolumn{2}{|l|}{Шаблоны сем. действий}&-&-&-&-&-&-&-&-&+&-&+\\
%&&&&&&&&&&&&&\\
\multicolumn{13}{|l|}{Аспекты}\\%&&&&&&&&&&&&&\\
&Незнание&-&-&-&-&-&+&+&+&+&-&+\\
&Квантификация&-&-&-&-&-&-&+&+&+&-&+\\
%&&&&&&&&&&&&&\\
\multicolumn{13}{|l|}{Гибкость при повторном использовании}\\%&&&&&&&&&&&&&\\
&Добавление продукций&-&+&+&+&+&+&+&+&+&+&+\\
&Замена элементов&-&+&-&+&-&-&-&-&-&+&+\\
&Удаление элементов&-&-&-&-&-&-&-&-&-&+&+\\
&Изменение метаданных&-&-&-&+&+&+&+&+&+&+&+\\
\dissonly{
\hline
\multicolumn{2}{|r|}{Cумма}&0&3&4&4&5&5&5&6&7&7&13\\}
\hline
\end{tabular}
}
\end{frame}

\begin{frame}[fragile]
	\frametitle{Применение}
\begin{itemize}
\item Диалекты языка SQL
	\begin{itemize}
		\item SQL92, PostgreSQL, Derby
		\item шаблоны для повторяющихся конструкций
		\item аспекты для отличий от стандарта
	\end{itemize}
\item Описание компонент IDE: подсветка синтаксиса, автоформатирование
	\begin{itemize}
		\item описание с помощью метаданных
		\item присоединение с помощью аспектов к одной грамматике
		\item сокращение дублирования
		\item контроль синхронности изменений
	\end{itemize}
\end{itemize}	
\end{frame}

\begin{frame}[fragile]
	\frametitle{Генератор трансляторов \ATF{}}
	
\begin{itemize}
\item Отсутствие ошибок в сгенерированном коде
	\begin{itemize}
		\item раннее обнаружение ошибок
		\item семантические действия на абстрактном языке
		\item проверка типов в семантических действиях
	\end{itemize}
\item Генерация кода на нескольких языках реализации
	\begin{itemize}
		\item интеграция с существующими генераторами
		\item декларативное описание системы типов
		\item единообразный механизм проверки и вывода типов
		\item генерация нескольких реализаций по одной спецификации
	\end{itemize}
\item Разработан на основе \GRM{}
	\begin{itemize}
		\item несколько спецификаций для одной грамматики
		\item механизмы композиции
	\end{itemize}
\end{itemize}	
\end{frame}

\renewcommand{\L}{{}}
\newcommand{\G}{ {\mathcal{G}_\L} }
\newcommand{\eql}{\cong_\L}
\newcommand{\lel}{\subtype_\L}
\newcommand{\gel}{\suptype_\L}
\newcommand{\TUP}{TupleType}
\newcommand{\Str}{String_\L}

\begin{frame}
	\frametitle{Параметризованная система типов}

\noindent
$\begin{array}{lll}

	\begin{array}{l}
		\trule{}{\Gamma \vdash \mathtt{NAME\#} : \Str}{token}\\
		\\
		\trule{}{\Gamma, \alpha : \tau \vdash \alpha : \tau }{attribute}\\
	\end{array}
&&
\trule{
\begin{array}{l}
	\tau_1 \ldots\tau_n \in \G\\
	\Gamma \vdash x_1 : \tau_1 \, \ldots \, \Gamma \vdash x_n : \tau_n
\end{array}
}{
	\Gamma \vdash (x_1, \ldots, x_n) : (\tau_1, \ldots, \tau_n)
}{tuple}

\\
&&\hspace{10pt}\\
\multicolumn{3}{c}{
\trule{	
%\begin{array}{lll}
		\tau_1 \ldots\tau_n,\, \sigma_1 \ldots \sigma_n,\, \rho_1 \ldots \rho_m \in \G
		&\,&
		\sigma_1 \gel \rho_1\,\ldots\, \sigma_m \gel \rho_m\\
		\Gamma \vdash f : (\sigma_1,\,\ldots,\,\sigma_m) \rightarrow (\tau_1,\,\ldots,\,\tau_n)
		&\,&
		\Gamma \vdash x_1 : \rho_1 \, \ldots \, \Gamma \vdash x_n : \rho_n\\
%\end{array}
}{\Gamma \vdash f(x_1,\,\ldots,\,x_m) : (\tau_1,\,\ldots,\,\tau_n)}{app}
}
\end{array}$

Параметры:
\begin{itemize}
\item множество \term{базовых типов} $\G$ (от \emph{ground});
\item выделенный \term{строковый тип} $\Str \in \G$;
\item отношение ``подтип-спертип'' $\lel \;\subseteq \G \times \G$.
\end{itemize}

\end{frame}


\begin{frame}[fragile]
	\frametitle{Сравнение с существующими подходами}
	
{
\small
\newcommand{\hd}[1]{{\begin{sideways}{\tool{#1}}\end{sideways}}}
\begin{tabular}{|ll|c|c|c|c|c|c|c|}
\hline
&&\hd{Coco/R}&\hd{Yacc/Bison}&\hd{Eli}&\hd{ANTLR}&\hd{GOLD}&\hd{SableCC}&\hd{\ATF{}}\\
\hline
\multicolumn{2}{|l|}{Генерация кода на разных языках}&+&+&-&+&+&+&+\\
\hline
\qquad\quad&по одной спецификации&-&-&-&-&+&+&+\\
\hline
\multicolumn{2}{|l|}{Поддержка семантических действий}&+&+&+&+&-&-&+\\
\hline
\multicolumn{2}{|l|}{Проверка типов в спецификациях}&-&-&-&-&n/a&n/a&+\\
\hline
\multicolumn{2}{|l|}{Ошибки в терминах спецификации}&-&-&+&-&n/a&n/a&+\\
\hline
\dissonly{\multicolumn{2}{|r|}{Сумма}&2&2&2&2&2&2&5\\
\hline}
\end{tabular}
}


Пример:
\begin{lstlisting}[language=Grammatic]
factor : VAR | INT | '(' expr ')' ;           
  factor(env : Environment) -> (result : Int) { 
    after VAR : result = value(env, VAR#);
    after INT : result = strToInt(INT#);
    at expr   : result = expr(env);     
  }
\end{lstlisting}

\end{frame}

\begin{frame}[fragile]
	\frametitle{Расширение языков поддержкой шаблонов}
\begin{block}{Механическая процедура}
\begin{itemize}
\item Расширение целевой мета-модели
	\begin{itemize}
		\item добавление понятий ``шаблон'', ``применение'', ``переменная''
		\item добавление специфичных шаблонных выражений
	\end{itemize}
\item Расширение конкретного синтаксиса
	\begin{itemize}
		\item возможны неоднозначности
		\item требуется коррекция вручную
	\end{itemize}
\item Расширение семантики и системы типов
	\begin{itemize}
		\item шаблонные правила для специфичных конструкций
	\end{itemize}
\end{itemize}	
\end{block}
\end{frame}

\begin{frame}[fragile]
	\frametitle{Базовые понятия в мета-модели}
{
\begin{tabular}{p{.45\textwidth}p{.45\textwidth}}
\begin{lstlisting}[xleftmargin=-.8cm]
class Type {
  attr typeName : String;
  attr multiplicity 
  			: Multiplicity;
}

enum Multiplicity {
  VAL, MANY_VAL, REF
}

class Abstraction {
  attr name : String;
  val parameters : Variable*;
  val body : Term;
  val type : Type?;
}

\end{lstlisting}
&
\begin{lstlisting}[xleftmargin=-.1cm]
abstract class Term {}

class VariableRef extends Term {
  ref variable : Variable;
}

class Application extends Term {
  ref abstraction : Abstraction;
  val arguments : Term*;
}

class Variable {
  attr name : String;
  val type : Type?;
}
\end{lstlisting}
\end{tabular}
}
\end{frame}


\newcommand{\TM}{\mathcal{TM}}
\newcommand{\TC}[1]{\mathcal{TC}\left(#1\right)}
\newcommand{\TR}[1]{\mathcal{TR}\left(#1\right)}
\newcommand{\TA}[1]{\mathcal{TA}\left(#1\right)}

\begin{frame}[fragile]
	\frametitle{Семантика разворачивания}
{\footnotesize
$$
\trule{
	\{p = e\} \subseteq \gamma
}{
	\Inst{\gamma}{\ang{?p}} = e
}{var-inst}
$$ 
$$
\trule{
	\mathbf{template}\left(
		T \, \ang{p_1, \ldots, p_n} \, \{ b \}
	\right)
}{
	\gamma' = \bigcup\limits_{i=1}^{n} \{ p_i = \Inst{\gamma}{a_i} \}
	\qquad
	\Inst{\gamma}{\ang{T \, a_1, \ldots, a_n}} = \Inst{\gamma \cup \gamma'}{b}
}{app-inst}
$$
$$
\trule{
%	C = \class{}{\_}{R=\{r_i\}}{A=\{a_i\}} &
	t = \obj{\TC{C}}{id}{r_i = rv_i,\, a_j = av_j}
}{
	\Inst{\gamma}{t} = \obj{\TC{C}}{id'}{r_i = \Inst{\gamma}{rv_i}, \,a_j = av_j }
}{ds-inst(С)}
$$}
\begin{Th}[О сохранении типов]\label{ThTP}
Если среда $\gamma$ согласована с контекстом $\Gamma$ и \mbox{$\Gamma \vdash e : \tau$}, то \mbox{$\Gamma \vdash \Inst{\gamma}{e} : \tau$}. Другими словами, преобразование $\Inst{}{\bullet}$ сохраняет типы.
\end{Th}
\begin{Th}[О нормализации]\label{ThNorm}
Если среда $\gamma = \cup \{p_i = e_i\}$ согласована с контекстом $\Gamma$, все $e_i$ имеют нормальную форму и $\Gamma \vdash e : \tau$, то результат вычисления $\Inst{\gamma}{e}$ имеет нормальную форму.
\end{Th}

\end{frame}

\begin{frame}[fragile]
	\frametitle{Система типов}
\scriptsize
$$
\trule{}{\Gamma \cup \{v : \tau\} \vdash \ang{?v} : \tau}{var}
$$ 
$$
\trule{
	\Gamma \cup \bigcup\limits_{i=1}^{n} \{ p_i : \tau_i \} \vdash b : \sigma
}{
	\Gamma \vdash \mathbf{template}\left(
		T \ang{p_1 : \tau_1, \ldots, p_n : \tau_n} \,: \sigma \, \{ b \}
	\right)
}{abstr}
$$
$$
\trule{
	\Gamma \vdash \mathbf{template}\left(
		T \ang{p_1 : \tau_1, \ldots, p_n : \tau_n} \,: \sigma \, \{ b \}
	\right)
	&
	\forall i : [1:n].\; \Gamma \vdash a_i : \tau_i \
}{
	\Gamma \vdash \ang{T \, a_1, \ldots, a_n} : \sigma
}{appl}
$$
$$
\myinfer[null]{
	\Gamma \vdash NULL : \tau^?
}{}
\quad
\myinfer[relax]{
	\Gamma \vdash x : \tau^?
}{
	\Gamma \vdash x : \tau
}
\quad
\myinfer[relax$^+$]{
	\Gamma \vdash x : \tau^*
}{
	\Gamma \vdash x : \tau^+
}
$$
$$
\myinfer[subtype]{
	\Gamma \vdash x : \sigma
}{
	\Gamma \vdash x : \tau &
	\tau \subtype \sigma
}
$$
$$
\myinfer[elist]{\Gamma \vdash [] : \tau*}{}
\quad
\myinfer[list]{
	\Gamma \vdash [t_1,\ldots,t_n] : \tau^+
}{
	\forall i \in [1:n].\; \Gamma \vdash Item(t_i, \tau)
}
$$
$$
\myinfer[eset]{\Gamma \vdash \{\} : \tau*}{}
\quad
\myinfer[set]{
	\Gamma \vdash \{t_1,\ldots,t_n\} : \tau^+
}{
	\forall i \in [1:n].\; \Gamma \vdash Item(t_i, \tau)
}
$$
$$
\myinfer[item]{
	\Gamma \vdash Item(t, \tau)
}{
	\Gamma \vdash t : \tau
}
\quad
\myinfer[item*]{
	\Gamma \vdash Item(\ang{?v}, \tau)
}{
	\Gamma \vdash v : \tau^*
}
\quad
\myinfer[item$^+$]{
	\Gamma \vdash Item(\ang{?v}, \tau)
}{
	\Gamma \vdash v : \tau^+
}
$$
$$
\myinfer[\mbox{ds-type(C)}]{
	\Gamma \vdash x : C
}{
	\begin{array}{l}
	x = \obj{\TC{C}}{id}{r_i = v_i; a_j = {v'}_j}\\
	\TM \Vdash x : \TC{C} 
	\end{array}	
	&
	\begin{array}{l}
	r_i = \TR{\rho_i : \tau_i}\\
	\Gamma \vdash v_i : \tau_i\\
	\end{array}	
}
$$
\end{frame}

\begin{frame}[fragile]
	\frametitle{Сравнение с существующими подходами: шаблоны}
	
\newcommand{\hd}[1]{{\begin{sideways}\parbox{30mm}{\tool{#1}}\end{sideways}}}
%{\begin{sideways}\parbox{15mm}{\bf text}\end{sideways}}
{\small\noindent
\begin{tabular}{|l|c|c|c|c|c|c|c|c|}
\hline 
	&\hd{Lisp}
	&\hd{MacroML}
	&\hd{Haskel TMP}
	&\hd{cpp}
	&\hd{Velocity и др.}
	&\hd{MPS}
	&\hd{Reuseware}
	&\hd{\textsl{Данная работа}}\\
\hline 
Поддержка многих языков&-&-&-&+&+&+&+&+\\
Применимо для текстовых языков&n/a&n/a&n/a&+&+&-&+&+\\
Замкнутость&+&+&+&+&+&n/a&-&+\\
Контроль корректности&n/a&+&+&-&-&+&+&+\\
Логика в теле шаблона&+&+&+&+/-&+&+&-&-\\
Используемый термин&M&M&T&M&T&T&CF&T\\
\dissonly{\hline
\multicolumn{1}{|r|}{Сумма}&3&3.5&3.5&3.5&4&3.5&3&4\\}
\hline
\end{tabular}
}

Пример:
\begin{lstlisting}
template Iterator<N : String, T : Class> : Class {
    class <?N> {
        fun hasNext() : Boolean
        fun next() : <?T>
    }
}
\end{lstlisting}
\end{frame}

\begin{frame}[fragile]
	\frametitle{Добавление аспектов}
\begin{itemize}
\item Расширение целевой мета-модели
\item Расширение конкретного синтаксиса
\item Расширение семантики и системы типов
	\begin{itemize}
		\item сопоставление с образцом
		\item применение аспекта
	\end{itemize}
\end{itemize}	

Мета-модель:
\begin{lstlisting}
class Aspect {
    rules : AspectRule*;
}
class AspectRule {
    pointcut : Term;
    advice : Map<Variable, Term>;
}
class Wildcard extends Term {
    type : Type;
}
\end{lstlisting}
\end{frame}

\begin{frame}[fragile]
	\frametitle{Семантика и типы}
Семантика применения:	\\
Пусть $\match{P}{e} = \ME \neq \bot$, \\$V = \left\{\langle v_i, t_i \rangle \,|\, i = 1..m \right\}$,  $\ME(v_i) = [e^i_1, \ldots, e^i_{n_i}]$, тогда
	$$\mathcal{R}@e
		= \apply{\bigsqcup\limits_{i=1}^{m} \bigsqcup\limits_{j=1}^{n_i}
			\subst{e^i_j}{\Inst{\meflatten{\ME}}{t_i}}}{e}$$
Типы:
$$
	\myinfer[wcard]{ \vdash \wcard{?}{\tau} : \tau}{}
$$
$$
	\myinfer[aspect]{
		(\mathbf{instead} \; \ang{?v} \, : \, t) \in Allowed(\mathcal{R})
	}{
		\mathcal{R} = \langle p, T, V \rangle &
		\Gamma(p) \vdash v : \tau &
		\Gamma(p) \vdash t : \sigma &
		\sigma \preceq \tau
	}
$$
\begin{Th}
Если $\mathcal{R}$ таково, что $V \subseteq Allowed(\mathcal{R})$, то в результате применения $\mathcal{R}@e$ не нарушаются структурные ограничения, накладываемые мета-моделью.
\end{Th}

\end{frame}

\begin{frame}[fragile]
	\frametitle{Сравнение с существующими подходами: аспекты}
	
\newcommand{\hd}[1]{{\begin{sideways}\parbox{30mm}{\tool{#1}}\end{sideways}}}
%{\begin{sideways}\parbox{15mm}{\bf text}\end{sideways}}
{\small\noindent
\begin{tabular}{|l|c|c|c|c|c|c|c|c|c|c|c|}
\hline 
	&\hd{JAMI}
	&\hd{AspectJ}
	&\hd{XCPL}
	&\hd{Reuseware}
	&\hd{Aspect.NET}
	&\hd{POPART}
	&\hd{XAspects}
	&\hd{Van Wyk'03}
	&\hd{Stratego/XT}
	&\hd{\textsl{Данная работа}}\\
\hline
\dissonly{
\multicolumn{1}{|r|}{\it Ссылка}
	&\cite{JAMI}
	&\cite{AspectJ}
	&\cite{XCPL}
	&\cite{Reuseware}
	&\cite{Aspect.NET}
	&\cite{POPART}
	&\cite{XAspects}
	&\cite{VanWyk03}
	&\cite{Bagge06}
	&\\
\hline 
}
Поддержка многих языков&+&-&+&+&+&+&+&+&+&+\\
\quad Независимость от платф. &-&-&+&+&-&+&-&+&+&+\\
\quad Неисполняемые языки&-&-&+&+&-&-&-&+&+&+\\
\quad Автоматизация&n/a&-&-&+&+&-&+&+&-&+\\
Незнание&+&+&+&-&+&+&+&+&+&+\\
Спец. язык срезов&-&+&+&-&+&+&+&-&+&+\\
Семантика встраивания&+&+&-&+&+&+&+&+&+&+\\
%Статическая семантика&-&+&-&+&+&+&+&+&+&+\\
\dissonly{
\hline
%\multicolumn{1}{|r|}{Сумма}&3&4&5&6&6&6&6&7&7&8\\}
\multicolumn{1}{|r|}{Сумма}&2&3&4&5&5&5&5&6&6&7\\}
\hline
\end{tabular}
}

Пример:
\begin{lstlisting}[xleftmargin=0.3cm]
on class <? : String> { ?funs=<? : Function*> } perform
    instead funs : <?funs>; fun toString() : String;
\end{lstlisting}

\end{frame}

\begin{frame}[fragile]
	\frametitle{Основные результаты}
	
\begin{itemize}
\item Разработан предметно-ориентированный язык описания текстового синтаксиса, реализующий механизмы композиции, основанный на использовании модулей, шаблонов и аспектов. Продемонстрирована эффективность механизмов композиции, использованных в данном языке.
\item Разработан генератор трансляторов, поддерживающий проверку типов в семантических действиях и гарантирующий отсутствие ошибок типизации в сгенерированном коде сразу для многих языков реализации.
\item Предложен метод автоматизации разработки предметно-ориентированных языков с поддержкой механизмов композиции, основанных на шаблонах и аспектов.
\item С использованием предложенного подхода созданы расширения языка \tool{Emfatic}.
\end{itemize}

\end{frame}

\end{document}