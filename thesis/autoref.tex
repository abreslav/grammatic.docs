\documentclass[12pt,a4paper]{article}

\usepackage{afterpage,fullpage,amsfonts,amsthm,amssymb,amsmath,indentfirst}

\usepackage{ucs}
\usepackage[utf8]{inputenc}
\usepackage[T2A]{fontenc}

\usepackage[english,russian]{babel}

\voffset=-5mm
\rightmargin=0pt

%\input ../sty/aut14.cli
\input{aut14.cli}
\makeatletter
\def\paragraph{\@startsection{paragraph}{4}{\z@}%
{2ex plus 0.2ex minus.1ex}{-1em}{\reset@font\bf}}
 \makeatother
\date{}
\catcode`@ = 11 \catcode`@ = 12
\def\hiphantom#1{\underline{\phantom{#1}}}

\sloppy

\selectlanguage{russian} \makeatletter
\renewcommand{\@oddhead}{\hfill{\large -\thepage-}\hfill}
\renewcommand{\@oddfoot}{}
\renewcommand{\@evenhead}{\hfill{\large -\thepage-}\hfill}
\renewcommand{\@evenfoot}{}
%\renewcommand{\@makefntext}[1]{\parindent=1em\noindent
% \hbox to 1.8 em {\hss\Large$^{\@thefnmark}$}#1}
\makeatother \large
%title

\newcommand{\afsection}[1]{\par \begin{center}\textbf{\MakeUppercase{#1}}\end{center}}
\newcommand{\afsubsection}[1]{\par \textbf{\underline{#1}}}

\begin{document}

Механизмы композиции в предметно-ориентированных языках

и т.д.

\thispagestyle{empty}
\newpage


\setlength{\topmargin}{-5mm} \makeatletter
\renewcommand{\@oddhead}{\hfill{\large -\thepage-}\hfill}
\renewcommand{\@oddfoot}{}
\renewcommand{\@evenhead}{\hfill{\large -\thepage-}\hfill}
\renewcommand{\@evenfoot}{}
\makeatother

\afsection{Общая характеристика работы}

\afsubsection{Актуальность темы.}
широкая тема, имена
компьютерные языки высокого уровня, разработка языков программирования ...
синтаксис
Хомский, Наур (Бэкус)
семантика Кнут
автоматизация разработки трансляторов
Ахо (Yacc), Турчин, Гуревич, Мартыненко

узкая тема
предметно-ориентированные языки -- еще большее повышение уровня абстракции
непосредственно оперируют понятиями предметной области
примеры: SQL, make, graphviz, \TeX.
порождающее программирование: краткость, исключение многих ошибок
вручную создаются языки в таких областях как описание трансляторов, бизнес-процессов, компонентного и сервис-ориентированного ПО, безопасности, веб-приложений и т.д.

мотивация специфичной задачи
чтобы использовать ПОЯ было выгодно, необходимо, чтобы их разработка требовала малых затрат времени и денег. Это мотивирует автоматизацию разработки. Причем важно, чтобы получающиеся языки были удобны в использовании, иначе выгода от повышения уровня абстракции может быть сведена на нет неудобством языка. Общая цель использования ПОЯ --- повысить качество ПО, поэтому спецификации, написанные с использованием ПОЯ должны сами отвечать таким критериям качества как модульность, повторное использование, которые обеспечиваются механизмами (де-)композиции ПО, такими как модули, полиморфизм и аспекты. Для того, чтобы ПОЯ это могли, нужно, чтобы автоматизированные средства разработки позволяли поддерживать эту функциональность без существенных затрат времени.

\afsubsection{Предметом исследования} являются механизмы композиции, пригодные для использования в предметно-ориентированных языках.

\afsubsection{Целью работы} является решение задачи, имеющей существенное значение в области автоматизации разработки предметно-ориентированных языков, а именно --- исследование и обоснование подходов и методов, позволяющих автоматически расширять предметно-ориентированные языки механизмами композиции, поддерживающими повторное использование.

\afsubsection{Задачи исследования.} Достижение поставленной цели подразумевает решение следующих задач:
\begin{itemize}
\item Сравнительный анализ механизмов композиции, используемых в современных предметно-ориентированных языках, с целью обоснования требований к средствам автоматизации.
\item Проектирование и реализация предметно-ориентированного языка для хорошо изученной области (описание текстового синтаксиса искусственных языков), поддерживающего все основные механизмы композиции в полном объеме, с целью выявления связей между этими механизмами и их характерных особенностей, влияющих на подходы к автоматизации.
\item Демонстрация адекватности разработанного языка нуждам конечных пользователей на примере описания синтаксиса языка структурных запросов к реляционным базам данных SQL и его диалектов.
\item Обобщение рассмотренных механизмов композиции в виде формализованных языковых конструкций.
\item Разработка алгоритмов автоматического расширения языка механизмами композиции и обоснование их корректности.
\item Демонстрация работоспособности предложенного подхода на примере языка описания объектно-ориентированных мета-моделей Emfatic.
\end{itemize}

\afsubsection{Методы исследования} включают

\afsubsection{Научная новизна} результатов работы состоит в том, что:

\afsubsection{Практическую ценность} работы составляют:

\afsubsection{На защиту выносятся следующие положения:} 

\afsubsection{Достоверность научных результатов и выводов}

\afsubsection{Внедрение результатов работы.} 

\afsubsection{Апробация работы.}

\afsubsection{Публикации.}

\afsubsection{Личный вклад автора.}

\afsection{Содержание работы}

\afsubsection{Во введении}

\afsubsection{Первая глава}

\afsubsection{Вторя глава}

\afsubsection{Третья глава}

\afsubsection{Четвертая глава}

\afsubsection{Основные результаты диссертационной работы}

\par
\begin{center}
\afsubsection{Публикации по теме диссертационной работы}
\end{center}


\end{document}