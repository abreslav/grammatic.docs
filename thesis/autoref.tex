\documentclass[12pt,a4paper]{article}

\usepackage{afterpage,fullpage,amsfonts,amsthm,amssymb,amsmath,indentfirst}

\usepackage{ucs}
\usepackage[utf8]{inputenc}
\usepackage[T2A]{fontenc}

\usepackage[english,russian]{babel}

\voffset=-5mm
\rightmargin=0pt

%\input ../sty/aut14.cli
\input{aut14.cli}
\makeatletter
\def\paragraph{\@startsection{paragraph}{4}{\z@}%
{2ex plus 0.2ex minus.1ex}{-1em}{\reset@font\bf}}
 \makeatother
\date{}
\catcode`@ = 11 \catcode`@ = 12
\def\hiphantom#1{\underline{\phantom{#1}}}

\sloppy

\selectlanguage{russian} \makeatletter
\renewcommand{\@oddhead}{\hfill{\large -\thepage-}\hfill}
\renewcommand{\@oddfoot}{}
\renewcommand{\@evenhead}{\hfill{\large -\thepage-}\hfill}
\renewcommand{\@evenfoot}{}
%\renewcommand{\@makefntext}[1]{\parindent=1em\noindent
% \hbox to 1.8 em {\hss\Large$^{\@thefnmark}$}#1}
\makeatother \large
%title

\newcommand{\tool}[1]{\textsc{#1}}

\newcommand{\afsection}[1]{\par \begin{center}\textbf{\MakeUppercase{#1}}\end{center}}
\newcommand{\afsubsection}[1]{\par \textbf{\underline{#1}}}

\begin{document}

Механизмы композиции в предметно-ориентированных языках

и т.д.

\thispagestyle{empty}
\newpage


\setlength{\topmargin}{-5mm} \makeatletter
\renewcommand{\@oddhead}{\hfill{\large -\thepage-}\hfill}
\renewcommand{\@oddfoot}{}
\renewcommand{\@evenhead}{\hfill{\large -\thepage-}\hfill}
\renewcommand{\@evenfoot}{}
\makeatother

\afsection{Общая характеристика работы}

\afsubsection{Актуальность темы.}
широкая тема, имена
компьютерные языки высокого уровня, разработка языков программирования ...
синтаксис
Хомский, Наур (Бэкус)
семантика Кнут
автоматизация разработки трансляторов
Ахо (Yacc), Турчин, Гуревич, Мартыненко

узкая тема
предметно-ориентированные языки -- еще большее повышение уровня абстракции
непосредственно оперируют понятиями предметной области
примеры: SQL, make, graphviz, \TeX.
порождающее программирование: краткость, исключение многих ошибок
вручную создаются языки в таких областях как описание трансляторов, бизнес-процессов, компонентного и сервис-ориентированного ПО, безопасности, веб-приложений и т.д.

мотивация специфичной задачи
чтобы использовать ПОЯ было выгодно, необходимо, чтобы их разработка требовала малых затрат времени и денег. Это мотивирует автоматизацию разработки. Причем важно, чтобы получающиеся языки были удобны в использовании, иначе выгода от повышения уровня абстракции может быть сведена на нет неудобством языка. Общая цель использования ПОЯ --- повысить качество ПО, поэтому спецификации, написанные с использованием ПОЯ должны сами отвечать таким критериям качества как модульность, повторное использование, которые обеспечиваются механизмами (де-)композиции ПО, такими как модули, полиморфизм и аспекты. Для того, чтобы ПОЯ это могли, нужно, чтобы автоматизированные средства разработки позволяли поддерживать эту функциональность без существенных затрат времени.

\afsubsection{Предметом исследования} являются механизмы композиции, пригодные для использования в предметно-ориентированных языках.

\afsubsection{Целью работы} является решение задачи, имеющей существенное значение в области автоматизации разработки предметно-ориентированных языков, а именно --- исследование и обоснование подходов и методов, позволяющих автоматически расширять предметно-ориентированные языки механизмами композиции, поддерживающими повторное использование.

\afsubsection{Задачи исследования.} Достижение поставленной цели подразумевает решение следующих задач:
\begin{itemize}
\item Сравнительный анализ механизмов композиции, используемых в современных предметно-ориентированных языках, с целью обоснования требований к средствам автоматизации.
\item Проектирование и реализация предметно-ориентированного языка для хорошо изученной области (описание текстового синтаксиса искусственных языков), поддерживающего все основные механизмы композиции в полном объеме, с целью выявления связей между этими механизмами и их характерных особенностей, влияющих на подходы к автоматизации.
\item Демонстрация адекватности разработанного языка нуждам конечных пользователей на примере описания синтаксиса языка структурных запросов к реляционным базам данных SQL и его диалектов.
\item Обобщение рассмотренных механизмов композиции в виде формализованных языковых конструкций. Описание их семантики и соответствующих систем типов.
\item Разработка алгоритмов автоматического расширения языка механизмами композиции, обоснование их корректности.
\item Демонстрация работоспособности предложенного подхода на примере языка описания объектно-ориентированных мета-моделей Emfatic.
\end{itemize}

\afsubsection{Методы исследования} включают методы инженерии программного обеспечения, анализа алгоритмов и программ, аппарат теории типов и методы количественной оценки характеристик программного обеспечения.

\afsubsection{Научная новизна} результатов работы состоит в том, что:
\begin{itemize}
\item Спроектирован и реализован предметно-ориентированный язык для описания текстового синтаксиса, поддерживающий композицию спецификаций с помощью модулей, шаблонов (типизированных макроопредений) и аспектов.
\item Предложена формализация механизмов композиции на основе шаблонов и аспектов, и доказаны свойства данной формализации, гарантирующие раннее обнаружение ошибок программиста при использовании этих механизмов.
\item Разработаны и апробированы алгоритмы, позволяющие автоматизировать расширение предметно-ориентированных языков механизмами композиции, основанными на шаблонах и аспектах.
\end{itemize}

\afsubsection{Практическую ценность} работы составляют:
\begin{itemize}
\item Разработанная библиотека, обеспечивающая трансляцию предложенного языка описания текстового синтаксиса.
\item Программные генераторы, использующие данную библиотеку, позволяющие автоматически получать трансляторы и компоненты интегрированной среды разработки.
\item Методика и алгоритмов расширения предметно-ориентированных языков механизмами композиции.
\end{itemize}

\afsubsection{На защиту выносятся следующие положения:} 
\begin{itemize}
\item Предметно-ориентированный язык для описания текстового синтаксиса, поддерживающий композицию спецификаций с помощью модулей, шаблонов и аспектов.
\item Метод, позволяющий автоматически расширить имеющееся описание синтаксиса и семантики предметно-ориентированного языка таким образом, что результирующий язык поддерживает композицию с помощью шаблонов и аспектов.
\end{itemize}

\afsubsection{Достоверность научных результатов и выводов} обеспечивается формальной строгостью описания процесса композиции языков, обоснованностью применения математического аппарата, результатами тестирования алгоритмов и программного обеспечения, а также эмпирическими данными, полученными в результате применения разработанных программных компонент.

\afsubsection{Внедрение результатов работы.} Результаты, полученные в ходе диссертационной работы, были использованы 

-- при выполнении НИР по программе УМНИК ()
-- в учебном процессе 
-- при реализации рабочего пакета "Встроенные предметно-ориентированные языки" проекта "Продуктивность разработки ПО" в рамках шестилетней программы разработок научно-исследовательского центра STACC (Тарту, Эстония)

\afsubsection{Апробация работы.} Изложенные в диссертации результаты обсуждались на 11 российских и международных научных конференциях и семинарах и школах, включая V, VI и VII всероссийскую межвузовскую научную конференцию молодых ученых (2008, 2009 и 2010 гг., Санкт-Петербург), международную научную конференцию ``Компьютерные науки и информационный технологии'' (2009 г., Саратов), международные научные школы ``Generative and Transformational Techniques in Software Engineering'' (2009 г., Брага, Португалия), ``Aspect-Oriented Software Development'' (2009 г., Нант, Франция) и ``15$^{th}$ Estonian Winter School in Computer Science'' (2010 г., Палмсе, Эстония), а так же международные семинары ``Teooriapäevad'' (2009 и 2010 гг., Эстония), семинар Лаборатории математической логики и семантики языков программирования Научно-исследовательского института кибернетики Эстонской Академии наук (2009 г., Таллинн, Эстония) и научном семинаре ``Computer Science Клуба'' при ПОМИ РАН (2009 г., Санкт-Петербург).

\afsubsection{Публикации.} По теме диссертации опубликовано пять печатных работ (из них две --- в изданиях, соответствующих требованиям ВАК РФ к кандидатским диссертациям).

-- Что считать, годы Вестника

\afsubsection{Личный вклад автора}
???

\afsubsection{Структура и объем работы.} Диссертация состоит из введения, четырех глав, списка литературы (??? наименований) и 3 приложений. Содержит ??? с. текста (из них ??? основного текста и ??? --- приложений), включая ??? рис. и табл.

\afsection{Содержание работы}

\afsubsection{Во введении} обосновывается актуальность проблемы, формулируются цель и задачи исследования, отмечаются научная новизна и практическая значимость результатов, перечисляются основные положения, выносимые на защиту.

\afsubsection{Первая глава} посвящена обзору существующих подходов к автоматизации разработки предметно-ориентированных языков. Долгое время в этой области доминировали инструменты, предназначенные для разработки трансляторов текстовых языков, основанные на принципах синтаксически управляемой атрибутной трансляции. Наиболее популярными инструментами этого семейства являются \tool{Yacc}, \tool{Bison} и \tool{ANTLR}. Такие инструменты подразумевают описание синтаксиса языка с помощью контекстно-свободной грамматики, с продукциями которой ассоциированы правила вычисления атрибутов, описывающих семантические свойства предложений языка. С помощью атрибутов могут быть реализованы все стадии семантического анализа от разрешения имен до преобразования в машинный код. 

Основным недостатком традиционных средств разработки трансляторов применительно к созданию предметно-ориентированных языков является трудоемкость разработки. Это вызвано в первую очередь тем, что такие инструменты создавались как универсальные средства разработки языков программирования и не предназначены для быстрого решения типичных задач, возникающих при создании небольших языков.

Распространенным способом решения проблемы быстрой разработки ПОЯ является встраивание таких языков в языки общего назначения. При этом используются встроенные в язык общего назначения средства мета-программирования или гибкие синтаксические конструкции. Чаще всего ПОЯ встраиваются в динамические языки, такие как \tool{Ruby} или \tool{Groovy}, и в языки с гибкой системой типов, такие как \tool{Haskell} или \tool{Scala}. Основным недостатком данного подхода является привязка к тому или иному языку общего назначения: ПОЯ, легко встраиваемый в один язык, может быть практически невозможно встроить в другой. Это затрудняет повторное использование ПОЯ и приводит к смешиванию уровней абстракции: аспекты реализации сильно влияют на высокоуровневое описание предметной области.

Другое направление в развитии средств разработки языков основано на использовании графического синтаксиса вместо текстового. Толчок к развитию этого направление дал UML, унифицированный язык моделирования. Чаще всего нотации таких языков используют диаграммы, представляющие собой графы с различными типами вершин и ребер, уложенные на плоскости. Такие языки показали высокую эффективность при решении определенного круга задач, но они не заменяют текстовых языков, в частности, потому, что требуют серьезной инструментальной поддержки, что создает трудности с совместимостью и переносимостью. Следует отметить также, что в области графического синтаксиса существует подход, реализованный в проекте \tool{MPS}, объединившим сильные стороны текстового и графического синтаксиса. Однако в настоящее время этот подход также требует серьезной инструментальной поддержки, что создает препятствия к его использованию в индустрии.

С развитием идеи предметно-ориентированных языков стали появляться инструменты, ориентированные на быструю разработку автономного транслятора вместе с интегрированной средой разработки. Преимущества этих инструментов состоят в том, что они, во-первых, позволяют создавать простые трансляторы с минимальными затратами времени, а во-вторых, автоматически генерируют расширения для той или иной интегрированной среды, облегчающие разработчику создание и поддержку программ на новом языке. Как правило, такие инструменты ориентированы на решение типичных проблем, и реализация поддержки более сложных механизмов с их помощью затруднительна, однако на практике оказывается, что простота и скорость разработки являются очень весомым фактором, и именно этим инструментам отдают предпочтение разработчики. Возникает необходимость расширять возможности подобных средств, не увеличивая нагрузки на программиста, то есть идет о полной или практически полной автоматизации разработки тех или иных механизмов в языке. Одним из таких механизмов является композиция, обеспечивающая повторное использование спецификаций, основными примерами универсальных механизмов такого рода являются шаблоны (статически проверяемые макроопределения) и аспекты. Существующие подходы к автоматизации разработки таких механизмов не позволяют автоматически получать языки, поддерживающие их в полном объеме.

\afsubsection{Вторя глава} описывает предметно-ориентированный язык для описания текстового синтаксиса, поддерживающий композицию с помощью модулей, шаблонов и аспектов. 

\afsubsection{Третья глава}

\afsubsection{Четвертая глава}

\afsubsection{Основные результаты диссертационной работы}

\par
\begin{center}
\afsubsection{Публикации по теме диссертационной работы}
\end{center}


\end{document}