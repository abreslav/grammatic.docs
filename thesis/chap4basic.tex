\chapter{Описание ядра}

текстовая нотация для моделей

\begin{lstlisting}
object : NAME '@' INT '{' (NAME '=' value ';')* '}' ;
value
	: object
	: ref
	: list
	: set
	: 'NULL'
	: INT
	: STRING
	: 'true' | 'false'
	: CHAR
	: NAME '.' NAME     // Enum.Literal
	;
ref	: 'ref' '(' NAME '@' INT ')';
list : <Collection '[', ']'> ;
set : <Collection '{', '}'> ;
template Collection<start, end> : Production+ {
	: <?start> <?end>
	: <?start>  <List object, ','> <?end>
	: <?end> <List ref, ','> <?end>
}
\end{lstlisting}

\begin{lstlisting}
list
	: '[' ']'
	: '[' object (',' object)* ']'
	: '[' ref (',' ref)* ']'
	;
set
	: '{' '}'
	: '{' object (',' object)* '}'
	: '{' ref (',' ref)* '}'
	;
\end{lstlisting}

система типов, порожденная мета-моделью

Язык типов:

C -- класс => ref(C), val(C)
T -- простой тип => T
A -- тип => A?, A*, A+

разные типы для множеств и списков?

иногда разделяем атрибуты и ссылки

Требования целевой мета-модели моделируются с помощью отношения ``подтип'', обозначаемого следующим образом:
$$
	\mbox{подтип} \subtype \mbox{супертип}.
$$
В остальном система типов выражена традиционным способом (см. \cite{Pierce}). \term{Контекст} $\Gamma$ представляет собой множество утверждений вида $e : \tau$, где $e$ --- шаблонное выражение, а $\tau$ --- тип; такие утверждения читаются как ``\term{$e$ имеет тип $\tau$}''. 

определить подтипы

\newcommand{\class}[4]{\mathbf{class}\; #1 \langle #2, #3, #4\rangle}
иногда -- нет

top type?
\newcommand{\classf}[3]{\mathbf{class}\; #1 \langle #2, #3\rangle}
\newcommand{\type}[2]{#1\left(#2\right)}
\newcommand{\valts}{\mathbf{val}}
\newcommand{\valt}[1]{\type{\valts}{#1}}
\newcommand{\refts}{\mathbf{ref}}
\newcommand{\reft}[1]{\type{\refts}{#1}}
\newcommand{\refv}[2]{\mathbf{ref}\; #1@#2}
\newcommand{\reference}[3]{\mathbf{#1}\langle #2 : #3\rangle}
\newcommand{\attribute}[2]{\mathbf{attr}\langle #1 : #2\rangle}
\newcommand{\obj}[3]{#1@#2\left\{#3\right\}}
\newcommand{\fromMM}{\MM{M} \Vdash}
$$
	\infer[object]{
		\fromMM \obj{C}{id}{f_i = v_i} : \valt{C}
	}{
		\classf{C}{\_}{f_i : \tau_i} \in \MM{M}&
		\fromMM v_i : \tau_i
	}
$$
$$
\infer[ref]{
	\fromMM \refv{C}{id} : C
}{}
\quad
\infer[null]{
	\fromMM NULL : \tau^?
}{}
$$
$$
\infer[elist]{
	\fromMM [\,] : \tau^*
}{}
\quad
\infer[list]{
	\fromMM [x_1, \ldots, x_n] : \type{TO}{\tau}^+
}{
	\forall i\in[1:n]. \; \fromMM x_i : \type{TO}{\tau} &
	TO \in \{\valts,\, \refts\}
}
$$
$$
\infer[eset]{
	\fromMM \{\} : \tau^*
}{}
\quad
\infer[set]{
	\fromMM \{x_1, \ldots, x_n\} : \type{TO}{\tau}^+
}{
	\forall i\in[1:n]. \; \fromMM x_i : \type{TO}{\tau} &
	TO \in \{\valts,\, \refts\}
}
$$
$$
\infer[relax]{
	\fromMM x : \tau^?
}{
	\fromMM x : \tau
}
\quad
\infer[relax^+]{
	\fromMM x : \tau^*
}{
	\fromMM x : \tau^+
}
\quad
\infer[subtype]{
	\fromMM x : \sigma
}{
	\fromMM x : \tau &
	\tau \subtype \sigma
}
$$
$$
\infer[superclass]{
	\forall i \in [1:n].\; \type{TO}{C} \subtype \type{TO}{S_i}
}{
	\classf{C}{\{S_i, \ldots, S_n\}}{\_} \in \MM{M} &
	TO \in \{\valts,\, \refts\}
}
$$
$$
\infer[enum]{
	\forall i \in [1:n]. \; \fromMM T.L_i : T
}{
	\mathbf{enum}\;T\{L_1,\ldots,L_n\} \in \MM{M}
}
\quad
\infer[bool]{
	\fromMM b : \mathtt{Boolean}
}{
	b \in \{\mathbf{true}, \mathbf{false}\}
}
$$
$$
\infer[int]{\fromMM INT : \mathtt{Integer}}{}
\quad
\infer[str]{\fromMM STRING : \mathtt{String}}{}
$$
$$
\infer[char]{\fromMM CHAR : \mathtt{Char}}{}
$$

структурная эквивалентность объектов

$$
\infer[obj]{
	\obj{C}{id_1}{f_i = v_i} \cong \obj{C}{id_2}{f_i = u_i}
}{
	\forall i. \; v_i \cong u_i
}
$$
$$
\infer[ref]{
	\refv{C}{id_1} \cong \refv{C}{id_2}
}{
	id_1 = id_2
}
\quad
\infer[refl]{x \cong x}{}
$$
$$
\infer[elist]{[\,] \cong [\,]}{}
\quad
\infer[list]{
	[x_1,\ldots,x_n] \cong [y_1,\ldots,y_n]
}{
	\forall i \in [1:n]. \; x_i \cong y_i
}
$$
$$
\infer[eset]{\{\} \cong \{\}}{}
\quad
\infer[set]{
	\{x_1,\ldots,x_n\} \cong \{y_1,\ldots,y_n\}
}{
	\forall i :\; x_i \cong y_{\pi(i)}
}
$$
$\pi$ -- перестановка

размыкание неагрегирующих ссылок

преобразование древовидной модели (все классы допускают локальную замену) в грамматику

семантические акции для разбора + анализ имен

мета-модель

грамматика

преобразование
