\documentclass[12pt]{report}

\usepackage[utf8x]{inputenc}
\usepackage{ucs}
\usepackage[T2A]{fontenc}
\usepackage{indentfirst} 

%\PrerenderUnicode{}

\usepackage[russian]{babel}
\usepackage{enumerate}
\binoppenalty=10000
\relpenalty=10000

\newcommand{\ignore}[1]{}

\usepackage[
	colorlinks=true,
	linkcolor=blue,
	citecolor=blue,
	urlcolor=blue
]{hyperref}
%\hypersetup{pdftitle={Breslav}} 

\usepackage{listings}
\usepackage{underscore}

\usepackage{graphicx}

\usepackage{multirow}
\usepackage{rotating}

\usepackage{amsthm}
\usepackage{amsmath}
\usepackage{amssymb}
\usepackage{array}

\usepackage{color}

\usepackage{proof}

\newcommand{\MM}[1]{\mathcal{#1}}%
\newcommand{\trule}[3]{%
\infer[\mbox{#3}]{#2}{#1}%(\mbox{\textsc{#3}})%
}%
\newcommand{\Inst}[2]{\mathcal{I}_{#1} \left[ #2 \right]}%
\newcommand{\match}[3][]{#2 \; \mathbf{match}_{#1} \; #3}
\newcommand{\ang}[1]{\mathsf{<}#1\mathsf{>}}
\newcommand{\wcard}[2]{\mathsf{<} #1 : #2 \mathsf{>} }
\newcommand{\ME}{\Upsilon}
\newcommand{\meitem}[2]{\left\{ #1 \mapsto #2 \right\}}
\newcommand{\meempty}{\left\{\right\}}
\DeclareMathOperator{\MEjoin}{\mbox{\Large$\uplus$\hspace{-2pt}}}
\DeclareMathOperator{\mejoin}{\uplus}
%\newcommand{\mejoin}{\uplus}
\newcommand{\mereplace}{\pitchfork}
\newcommand{\meflatten}[1]{\overline{#1}}
\newcommand{\subtype}{\preceq}
\newcommand{\suptype}{\succeq}
\newcommand{\myinfer}[3][]{\infer[\mbox{#1}]{#2}{#3}}

%\usepackage{bold-extra}
%\renewcommand{\ttdefault}{pcr}

\newcommand{\GRM}{\tool{Grammatic}}
\newcommand{\ATF}{\tool{Grammatic$^{SDT}$}}

\definecolor{Brown}{cmyk}{0,0.81,1,0.60}
\definecolor{OliveGreen}{cmyk}{0.64,0,0.95,0.40}
\definecolor{CadetBlue}{cmyk}{0.62,0.57,0.23,0}
\definecolor{MyDarkBlue}{rgb}{0,0.08,0.45} 

\lstset{
	inputencoding=utf8x, 
	extendedchars=\true, 
	basicstyle=\ttfamily\footnotesize,
	keywordstyle=\sffamily\bfseries,
	stringstyle=\sffamily,
	commentstyle=\color{OliveGreen}\ttfamily,
	tabsize=4,
	captionpos=b,
	showstringspaces=false,
	xleftmargin=1cm,
	texcl,
}
\lstdefinelanguage{Grammatic}
	{
		morestring=[b]',
		morekeywords={lex,empty,*,?,+,before,after,at},
		morecomment=[l]{//},
	}
\lstdefinelanguage{Typesystem}
	{
		morestring=[b]',
		morekeywords={typesystem,language,for,backend,type},
		morecomment=[l]{//},
	}
\lstdefinelanguage{ANTLR}{
	morecomment=[s][\itshape\color{MyDarkBlue}]{\{}{\}},
	morekeywords={returns,int},
	morestring=[b]',
	morecomment=[l][\color{red}]{//!},
	morecomment=[l]{//},
}


\sloppy

\newcommand{\figref}[1]{Рис.~\ref{#1}}
\newcommand{\lstref}[1]{Лист.~\ref{#1}}
\newcommand{\tabref}[1]{Таб.~\ref{#1}}
\newcommand{\term}[1]{\emph{#1}}
\newcommand{\code}[1]{\mbox{\texttt{#1}}}
\newcommand{\tool}[1]{\textsc{#1}}
\newcommand{\bad}[1]{{\color{red}\textbf{#1}}}

%\newtheorem{Def}{Определение}[section]
%\newtheorem{Prop}{Утверждение}[section]
%\newtheorem{Note}{Замечание}

\DeclareFontFamily{OML}{txmi}{\skewchar\font127 }
\DeclareFontShape{OML}{txmi}{m}{it}{
   <-> txmi1%
}{}
\DeclareFontShape{OML}{txmi}{bx}{it}{
   <-> txbmi1%
}{}

\DeclareFontShape{OML}{txmi}{l}{it}{<->ssub * txmi/m/it}{}
\DeclareFontShape{OML}{txmi}{b}{it}{<->ssub * txmi/bx/it}{}

\SetSymbolFont{letters}{bold}{OML}{txmi}{bx}{it}
\SetSymbolFont{letters}{normal}{OML}{txmi}{m}{it}

\DeclareSymbolFont{EulerExtension}{U}{euex}{m}{n}
\DeclareMathSymbol\intop\mathop{EulerExtension}{"52}
\DeclareMathSymbol\ointop\mathop{EulerExtension}{"48}

\textwidth=150mm
\textheight=235mm
\topmargin=-10mm
\oddsidemargin=14.6mm
\renewcommand{\baselinestretch}{1.3}
\hfuzz=2pt
\tolerance=1500
\emergencystretch=3em
\setlength{\parindent}{12.7mm}
\raggedbottom
\mathsurround=2pt
\righthyphenmin=2
\sloppy
\pagestyle{myheadings}
\makeatletter
\renewcommand{\@oddhead}{\hfil \thepage}
\makeatother

\theoremstyle{definition}
\newtheorem{Def}{Определение~}[part]
\theoremstyle{plain}
\newtheorem{Th}{Теорема~}[part]
\newtheorem{Lemm}[Th]{Лемма~}
\newtheorem{Prop}[Th]{Предложение~}
\newtheorem{Cor}{Следствие~}[Th]
\newtheorem{Note}{Замечание~}[part]
\newtheorem{Ex}{Пример~}[part]
\renewcommand{\proofname}{Доказательство}

\renewcommand{\theLemm}{\thepart.\arabic{Lemm}}
\renewcommand{\theTh}{\thepart.\arabic{Th}}
\renewcommand{\theProp}{\thepart.\arabic{Prop}}
\renewcommand{\theCor}{\theTh.\arabic{Cor}}
\renewcommand{\theNote}{\thepart.\arabic{Note}}
\renewcommand{\tablename}{Таблица}
\renewcommand{\lstlistingname}{Листинг}
\renewcommand{\figurename}{Рис.}
\renewcommand{\baselinestretch}{1.23}

\makeatletter
\@addtoreset{chapter}{part}
\newcommand{\intro}[1]{%
\newpage%
\begin{center}\bf\Large #1\end{center}%
\addcontentsline{toc}{part}{\bf #1}%
}
\renewcommand{\thepart}{\arabic{part}}
\renewcommand{\part}[1]{%
\newpage~\\
%
\refstepcounter{part}%
\begin{flushleft}\bf\LARGE Глава~\thepart.\\
#1\end{flushleft}
\addcontentsline{toc}{part}{\bf Глава~\thepart. #1}%
}
\renewcommand{\chapter}[1]{%
\refstepcounter{chapter}%
\begin{flushleft}\bf\Large \S~\thechapter. #1\end{flushleft}
\nopagebreak[4]%
\addcontentsline{toc}{section}{\mbox{\S~\thechapter.} #1}%
}
\renewcommand{\section}[1]{%
\nopagebreak[4]
\refstepcounter{section}%
\begin{flushleft}\bf \thesection. #1\end{flushleft}%
\addcontentsline{toc}{section}{\hspace{1em}\thesection. #1}%
}
\renewcommand{\subsection}[1]{%
\nopagebreak[4]
\refstepcounter{subsection}%
\begin{flushleft}\bf \thesubsection. #1\end{flushleft}%
}
\makeatother

\makeatletter
\renewcommand*\l@part{\@dottedtocline{1}{0pt}{0em}}
\renewcommand*\l@section{\@dottedtocline{2}{0pt}{2.8em}}
\renewcommand{\tableofcontents}
{\begin{center}\Large\bf Оглавление\end{center}
\thispagestyle{plain}
\@starttoc{toc}}
\makeatother

\makeatletter
\renewenvironment{thebibliography}[1]
     {\intro{Список литературы}
      \list{\@biblabel{\@arabic\c@enumiv}}%
           {\settowidth\labelwidth{\@biblabel{#1}}%
            \leftmargin\labelwidth
            \advance\leftmargin\labelsep
            \@openbib@code
            \usecounter{enumiv}%
            \let\p@enumiv\@empty
            \renewcommand\theenumiv{\@arabic\c@enumiv}}%
      \sloppy
      \clubpenalty4000
      \@clubpenalty \clubpenalty
      \widowpenalty4000%
      \sfcode`\.\@m}
     {\def\@noitemerr
       {\@latex@warning{Empty `thebibliography' environment}}%
      \endlist}
\makeatother



\begin{document}\large
\thispagestyle{empty}
\begin{center}\Large 
ГОУВПО ``Санкт-Петербургский государственный университет информационных технологий, механики и оптики''
\end{center}
\vfill

\begin{flushright}\Large
на правах рукописи
\end{flushright}
\vfill

\begin{center}\Large\bf
Бреслав Андрей Андреевич
\end{center}
\vfill

\begin{center}\Large
Автоматизация разработки механизмов композиции в предметно-ориентированных языках
\end{center}
\vfill

\begin{center}\Large
Специальность 05.13.11~--- Математическое и
программное обеспечение вычислительных машин, комплексов и
компьютерных сетей
\end{center}
\vfill

\begin{center}\Large
Диссертация на соискание ученой степени 
кандидата физико-математических наук
\end{center}
\vfill

\hfill
\begin{minipage}{0.7\textwidth}
\begin{flushleft}\Large
Научный руководитель \\
кандидат физико-математических
наук, доцент Ф.~А.~Новиков
\end{flushleft}
\end{minipage}
\vfill

\begin{center}\Large
Санкт-Петербург -- 2011
\end{center}
\newpage

\tableofcontents
\newpage

\newcommand{\afsubsection}[1]{\par \textbf{#1}}

\intro{Введение}
\afsubsection{Актуальность темы.}
Развитие технологии программирования исторически идет по пути повышения уровня абстракции поддерживаемого инструментами, используемыми при разработке программного обеспечения~(ПО). Решающим шагом на этом пути явилось создание языков программирования высокого уровня, позволяющих лишь в небольшой степени заботиться об особенностях конкретной аппаратной архитектуры. Повышение уровня абстракции --- один из ключевых факторов, определяющих сокращение сроков создания программных средств, поскольку высокоуровневые инструменты позволяют избегать определенных типов ошибок и повторно использовать разработанные решения, а также облегчают командную разработку.

С развитием языков высокого уровня неразрывно связан процесс развития автоматизированных инструментов разработки трансляторов, базирующийся на достижениях теории формальных языков и грамматик, в частности, алгоритмах преобразования контекстно-свободных грамматик в магазинные автоматы и формализация семантики с помощью атрибутных грамматик.

Дальнейшее повышение уровня абстракции привело к возникновению идеи \term{предметно-ориентированных языков} (ПОЯ), предназначенных для решения задач в относительно узкой предметной области и нередко непригодных за ее пределами. ПОЯ противопоставляются языкам общего назначения, являющимся вычислительно универсальными и позволяющим решать любые задачи. Основной мотивацией к разработке и использованию ПОЯ является тот факт, что моделирование предметной области в языках общего назначения часто бывает недостаточно явным, что приводит к большим объемам кода и затруднениям при чтении программ. При использовании ПОЯ эта проблема снимается, поскольку такие языки оперируют непосредственно понятиями предметной области и могут даже позволить специалистам в этой области, не имеющим квалификации разработчиков ПО, принимать участие в написании программ. В настоящее время ПОЯ применяются во множестве областей, начиная с систем управления базами данных и заканчивая системами моделирования бизнес-процессов.

Использование ПОЯ снижает затраты на разработку ПО в данной предметной области, но разработка самих ПОЯ также требует затрат. Если эти затраты высоки, то использование ПОЯ может быть нецелесообразным, поэтому возникает задача автоматизации разработки таких языков с целью минимизировать затраты на их создание и поддержку. Традиционные средства разработки трансляторов не обеспечивают необходимый уровень автоматизации, поэтому разрабатываются новые подходы и инструменты, позволяющие быстро разрабатывать небольшие языки с поддержкой все более сложных механизмов. В частности, существенный интерес представляют механизмы композиции, обеспечивающие повторное использование кода, написанного на ПОЯ. %Эти механизмы достаточно сложны и реализация их вручную требует существенных затрат, поэтому возникает необходимость в автоматизации решения этой задачи.
%Причем важно, чтобы получающиеся языки были удобны в использовании, иначе выгода от повышения уровня абстракции может быть сведена на нет неудобством языка. Общая цель использования ПОЯ --- повысить качество ПО, поэтому спецификации, написанные с использованием ПОЯ должны сами отвечать таким критериям качества как модульность, повторное использование, которые обеспечиваются механизмами (де-)композиции ПО, такими как модули, полиморфизм и аспекты. Для того, чтобы ПОЯ это могли, нужно, чтобы автоматизированные средства разработки позволяли поддерживать эту функциональность без существенных затрат времени.

\afsubsection{Предметом исследования} являются механизмы композиции, пригодные для использования в предметно-ориентированных языках.

\afsubsection{Целью работы} является исследование и обоснование подходов и методов, позволяющих автоматически расширять предметно-ориентированные языки механизмами композиции, поддерживающими повторное использование.

\afsubsection{Задачи исследования.} Достижение поставленной цели подразумевает решение следующих задач:
\begin{itemize}
%\item Сравнительный анализ механизмов композиции, используемых в современных предметно-ориентированных языках, с целью обоснования требований к средствам автоматизации.
\item Проектирование и реализация предметно-ориентированного языка для хорошо изученной области --- описания текстового синтаксиса искусственных языков --- поддерживающего все основные механизмы композиции в полном объеме, с целью выявления связей между этими механизмами и их характерных особенностей, влияющих на подходы к автоматизации.
\item Проверка адекватности разработанного языка нуждам конечных пользователей на примере описания синтаксиса сложных языков.
\item Обобщение рассмотренных механизмов композиции в виде формализованных языковых конструкций. Описание их семантики и соответствующих систем типов.
\item Разработка алгоритмов автоматического расширения языка механизмами композиции, обоснование их корректности.
\item Применение предложенного подхода к существующему предметно-ориентированному языку.
\end{itemize}

\afsubsection{Методы исследования} включают методы инженерии программного обеспечения, анализа алгоритмов и программ, аппарат теории типов, теории графов и теории формальных грамматик.

\afsubsection{Научная новизна} результатов работы состоит в том, что:
\begin{itemize}
\item Спроектирован и реализован предметно-ориентированный язык для описания текстового синтаксиса, поддерживающий композицию спецификаций с помощью модулей, шаблонов (типизированных макроопредений) и аспектов.
\item На основе указанного языка разработан генератор трансляторов, поддерживающий проверку типов в семантических действиях и гарантирующий отсутствие ошибок типизации в сгенерированном коде для многих языков реализации.
\item Предложена формализация механизмов композиции на основе шаблонов и аспектов, и доказаны свойства данной формализации, гарантирующие раннее обнаружение ошибок программиста при использовании этих механизмов.
\item Разработаны и апробированы алгоритмы, позволяющие автоматизировать расширение предметно-ориентированных языков механизмами композиции, основанными на шаблонах и аспектах.
\end{itemize}

\afsubsection{Практическую ценность} работы составляют:
\begin{itemize}
\item Разработанная библиотека, обеспечивающая трансляцию предложенного языка описания текстового синтаксиса.
\item Программные генераторы, использующие данную библиотеку, позволяющие автоматически получать трансляторы и компоненты интегрированной среды разработки.
\item Методика и алгоритмы расширения предметно-ориентированных языков механизмами композиции.
\end{itemize}

\afsubsection{На защиту выносятся следующие положения:} 
\begin{itemize}
\item Предметно-ориентированный язык для описания текстового синтаксиса, поддерживающий композицию спецификаций с помощью модулей, шаблонов и аспектов.
\item Подход к описанию и генерации трансляторов, позволяющий порождать код на нескольких языках программирования по одной спецификации, и гарантирующий отсутствие ошибок типизации в сгенерированном коде.
\item Метод автоматического расширения имеющихся описаний синтаксиса и семантики предметно-ориентированного языка таким образом, что результирующий язык поддерживает композицию с помощью шаблонов и аспектов.
\end{itemize}

\afsubsection{Достоверность научных результатов и выводов} обеспечивается формальной строгостью описания процесса композиции языков, обоснованностью применения математического аппарата, результатами тестирования алгоритмов и программного обеспечения.

\afsubsection{Внедрение результатов работы.} Результаты, полученные в ходе диссертационной работы, были использованы в компании OpenWay (Санкт-Петербург) при разработке предметно-ориентированного языка для написания отчетов, а также при выполнении НИОКР ``Технология разработки предметно-ориентированных языков'' по программе ``У.М.Н.И.К.'' Фонда содействия развитию малых форм предприятий в научно-технической сфере.

\afsubsection{Апробация работы.} Изложенные в диссертации результаты обсуждались на 11 российских и международных научных конференциях, семинарах и школах, включая V, VI и VII всероссийские межвузовские научные конференции молодых ученых (2008, 2009 и 2010~гг., Санкт-Петербург), международную научную конференцию ``Компьютерные науки и информационный технологии'' (2009 г., Саратов), международные научные школы ``Generative and Transformational Techniques in Software Engineering'' (2009~г., Брага, Португалия), ``Aspect-Oriented Software Development'' (2009~г., Нант, Франция) и ``15$^{th}$ Estonian Winter School in Computer Science'' (2010~г., Палмсе, Эстония), а также международные семинары ``Teooriapäevad'' (2009 и 2010~гг., Эстония), семинар Лаборатории математической логики и семантики языков программирования Научно-исследовательского института кибернетики Эстонской Академии наук (2009~г., Таллинн, Эстония) и научном семинаре ``Computer Science Клуба'' при ПОМИ РАН (2009~г., Санкт-Петербург).

\afsubsection{Публикации.} По теме диссертации опубликовано пять печатных работ (из них две статьи --- в изданиях, соответствующих требованиям ВАК РФ к кандидатским диссертациям по данной специальности).

\afsubsection{Структура и объем работы.} Диссертация состоит из введения, четырех глав, списка литературы (? наименований) и 3 приложений. Содержит ? с. текста (из них ? основного текста и ? --- приложений), включая ? рис. и табл.


%\part{Предварительные сведения об инженерии компьютерных языков}\label{part1}

В настоящей главе приводится обзор современного состояния инженерии языков. Основное внимание уделяется средствам разработки трансляторов и механизмам композиции.

%\chapter{Мета-моделирование}

К последнее десятилетие широкое распространение получили \term{объектно-ориентированные модели программного обеспечения} и \term{мета-моделирование} (Meta-Modeling, \cite{MetaModeling}). Этот подход используется для проектирования, документирования и автоматического построения ПО. Он получил признание благодаря UML, унифицированному языку моделирования (Unified Modeling Language, \cite{UML}) и MDA, архитектуре, управляемой моделями (Model-Driven Architecture, \cite{MDA}).
Позднее возникли и другие языки и технологии, опирающиеся на те же принципы.

В данном разделе мы приводим основные определения и примеры, необходимые для понимания концепции мета-моделирования. 

\section{Модели}

Одним из центральных понятий в данной области является ``объектно-ориентированная модель''.  Также говорят ``модель предметной области'' или просто ``модель''. Под моделью понимается структурированное описание некоторой сущности реального мира (например, программной или аппаратной системы, инфраструктуры предприятия и т.д.). С формальной точки зрения модель представляет собой граф, вершинами в котором являются типизированные объекты, обладающие типизированными атрибутами \cite{KM3}. Естественным визуальным представлением моделей являются \term{диаграммы}; в качестве примера рассмотрим простую диаграмму модели зависимостей между компонентами ПО (\figref{DiagramExample}).

\begin{figure}[htbp]
// Диаграмма: модель билда с зависимостями
\caption{Модель зависимостей между компонентами ПО}\label{DiagramExample}
\end{figure}

Вершины графа на \figref{DiagramExample} соответствуют компонентам, а ребра --- зависимостям. Каждое ребро направлено от зависимой компоненты (\term{клиента}) к компоненте, от которой она зависит (к \term{серверу}). Таким образом, данная модель состоит из трех объектов, ссылающихся друг на друга. Аналогично объектно-ориентированным языкам программирования \cite{JLS}, каждый объект имеет тип, определяемый его классом. В данном случае все три объекта относятся к одному классу \code{Component} (см. \figref{ComponentClass}). 

\begin{figure}[htbp]
// Код, UML-диаграмма и диаграмма объектов для класса Component
class Component {
	attr name : String;
	ref dependsOn : Component[*];
}
\caption{Мета-модель для компонент}\label{ComponentClass}
\end{figure}

На \figref{ComponentClass} (a) приведен псевдокод объявления класса Component. Данный класс декларирует \term{атрибут} \code{name} \term{примитивного типа} \code{String}, значение которого на диаграмме отображается внутри вершины, и \term{ссылку} \code{dependsOn} типа \code{Component}, значения которой на диаграмме отображаются в виде ребер между объектами типа \code{Component}. Поскольку одна компонента может зависеть от нескольких других компонент, ссылка \code{dependsOn} является \term{множественной}, что определяется аннотацией в квадратных скобках (``*'' соответствует кратности (multiplicity) ``ноль или более'').

Во многих объектно-ориентированных языках программирования (\tool{Java}, \tool{C\#}, \tool{SmallTalk} \cite{SmallTalk}) классы являются также и объектами. В нашем случае мы можем описать класс \code{Component} как модель. Как видно из \figref{ComponentClass} (b), эта модель состоит из следующих объектов: 
\begin{enumerate}
\item самого класса \code{Component};
\item атрибута \code{name};
\item ссылки \code{dependsOn};
\item типа \code{String}.
\end{enumerate}
Все объекты имеют имена, а ссылка \code{dependsOn} -- еще и кратность.
Ребра на диаграмме подписаны в соответствии со связями, которые они отображают: класс владеет атрибутами и ссылками, атрибуты и ссылки имеют типы. 

Одну и ту же модель можно изображать с помощью диаграмм по-разному, так на \figref{ComponentClass} (c) показана та же самая модель для класса Component, отображенная в виде стандартной диаграммы классов языка UML \cite{UML}. Как видно из рисунка, эта диаграмма является более компактной за счет того, что атрибут отображаются внутри вершины, соответствующей классу, который ими владеет, а ссылки --- в виде ребер, направленных от класса, который владеет ссылкой, к классу, являющемуся типом ссылки (в UML ссылки моделируются с помощью \term{ассоциаций}, обладающих гораздо более выразительной семантикой, и в частности, позволяющих выразить отношения произвольной арности \cite{???}). Этот пример является очень полезным, поскольку позволяет увидеть, что объект, имеющийся в модели (ссылка), может быть изображен на диаграмме в виде ребра.

\section{Мета-модели и иерархия мета-уровней}

Модель, изображенная на \figref{ComponentClass} описывает типы элементов модели, изображенной на \figref{DiagramExample}, то есть является для нее \term{мета-моделью}. Каждый элемент на диаграмме \figref{ComponentClass} имеет соответствующий элемент в мета-модели (\term{мета-элемент}), и называется его \term{экземпляром}. Отношение ``экземпляр-мета-элемент'' проиллюстрировано на \figref{ConformsToRelation}.

\begin{figure}[htbp]
\caption{Отношение ``экземпляр-мета-элемент''}\label{ConformsToRelation}
\end{figure}

Поскольку в любой модели все объекты типизированы, каждая модель имеет мета-модель  (то есть любая модель является экземпляром некоторой мета-модели). Говорят об ``иерархии мета-уровней'', которые схематически изображены на \figref{ConformsToRelation}  и помечены как $M^1$, $M^2$, $M^3$\ldots Модели уровня $M^{i+1}$ являются мета-моделями для моделей уровня $M^i$. В принципе, иерархия мета-уровней может быть сколь угодно высокой или даже бесконечной, но для практических целей обычно используется иерархия из трех уровней. Это достигается за счет эффекта ``раскрутки'' (bootstrapping, см. \cite{Wirth}): на уровне $M^3$ помещается ровно одна мета-модель, которая типизирует сама себя. Такая модель называется \term{мета-мета-моделью}, именно мета-мета-модели обычно составляют основу модельно-ориентированных инструментов разработки и стандартов, таких как MOF \cite{MOF}, KM3 \cite{KM3}, VPM \cite{VPM} или \tool{Ecore} \cite{EMF}. Большинство используемых на практике мета-мета-моделей эквивалентны по выразительности, то есть существуют автоматизированные средства преобразования между их экземплярами (см. напр. \cite{KM3}). Здесь и далее мы будем использовать мета-мета-модель \tool{Ecore}, которая лежит в основе библиотеки EMF (Eclipse Modeling Framework, \cite{EMF}) и хорошо зарекомендовала себя в качестве платформа для разработки приложений с использованием моделей.

\section{Мета-мета-модель \tool{Ecore}}

\begin{figure}[htbp]
\centering
\includegraphics[width=\textwidth]{ecore.png}
\caption{Мета-мета-модель \tool{Ecore}}\label{Ecore}
\end{figure}

На \figref{Ecore} изображена диаграмма мета-мета-модели \tool{Ecore}. Согласно объектно-ориентированной парадигме, центральным понятием в \tool{Ecore} является класс. Экземпляры \tool{Ecore} (то есть мета-модели) представляют собой наборы классов, организованные в пакеты и связанные отношением ``общее-частное'', аналогичным наследованию в объектно-ориентированных языках программирования \cite{OOAD}. На \figref{ComponentClassEcore} показано, как элементы мета-модели, описывающей модели зависимостей между компонентами (такие как \figref{DiagramExample}) связаны с соответствующими мета-элементами в \tool{Ecore}.

//ComponentClassEcore

Следует обратить внимание на то, что \tool{Ecore} сама состоит из классов, что и позволяет ``оборвать'' иерархию мета-уровней, не используя $M^4$ (в действительности, иерархия не обрывается, просто все уровни, начиная с $M^3$ совпадают между собой и содержат только мета-мета-модель, то есть в нашем случае --- \tool{Ecore}).

В этом разделе мы опишем несколько основных концепций, используемых \tool{Ecore} (и другими мета-мета-моделями), которые будут использованы при изложении дальнейшего материала.

\paragraph*{Классы.} Как мы уже отмечали выше, основными элементами мета-моделей, описанных с помощью \tool{Ecore}, являются классы. Каждый класс имеет имя уникальное в пределах пакета, список суперклассов и список структурных элементов, атрибутов и ссылок. Экземпляры классов хранят значения для всех атрибутов и ссылок, объявленных самим классом и всеми его суперклассами (что соответствует наследованию членов классов в объектно-ориентированных языках). Класс может быть помечен как абстрактный. Такой класс не может иметь непосредственных экземпляров, экземпляры могут иметь только его подклассы, не являющиеся абстрактными.

Заметим, что понятие абстрактного класса в \tool{Ecore} является чисто номинальным, никаких технических препятствий к созданию экземпляра класса (таких как, например в \tool{Java} или \tool{C++}) быть не может, поскольку \tool{Ecore} не определяет тела методов. 

\paragraph*{Примитивные типы и перечисления.} Классы --- не единственное средство типизации в \tool{Ecore}, кроме них используются примитивные типы данных (Data Types) и перечисления (Enums). Примитивный тип данных представляет собой именованную сущность, семантика которой либо не фиксируется, либо определяется тем языком программирования, на котором реализовано моделируемое ПО (в случае \tool{Ecore} это \tool{Java}). 

\tool{Ecore} определяет несколько встроенных типов данных, таких как строки, целые и вещественные числа и т. д.

Перечисления --- это типы имеющие конечное множество значений (литералов).

\paragraph*{Структурные элементы классов: атрибуты и ссылки.} Основное отличие классов и примитивных типов данных --- в их назначении. Классы являются типами объектов, из которых состоят модели, а примитивные типы и перечисления --- типами значений атрибутов этих объектов. Таким образом, ссылки, определяемые классами могут иметь в качестве типа только класс, а атрибуты --- только примитивный тип или перечисление.

В остальном ссылки и атрибуты очень похожи. И те, и другие имеют имя и кратность, которая задается как пара чисел, определяющих нижнюю и верхнюю границу для количества значений, хранимых атрибутом или ссылкой. Так, например, ссылка \code{dependsOn} на \figref{DiagramExample} имеет кратность ``от 0 до $\infty$'', что означает, что компонента может зависеть от нуля или более других компонент. Атрибут \code{name} на том же рисунке имеет кратность ``от 1 до 1'', что означает, что каждая компонента должна иметь имя, причем ровно одно.

Как атрибуты, так и ссылки могут быть помечены как ``упорядоченные''. Этот флаг имеет значение для множественных ссылок и атрибутов (верхняя граница кратности которых больше единицы), и означает, что порядок объектов, на которые указывают ссылки (или значений атрибута) важен. Например, в классе \code{Block}, описывающего блок кода в программе и имеющего ссылку \code{statements}, указывающую на последовательность предложений внутри блока, ссылка \code{statements} должна быть упорядоченной, поскольку нам важно знать, в каком порядке выполнять предложения, из которых состоит блок.

\paragraph*{Перекрестные ссылки и агрегация.} Ссылки в \tool{Ecore} подразделяются на два вида: \term{перекрестные} и \term{агрегирующие}. Различие состоит в том, что на один объект не может быть более одной агрегирующей ссылки, то есть объект, имеющий агрегирующую ссылку, ``владеет'' объектом, на который ссылается. Для перекрестных ссылок таких ограничений нет, на любой объект может быть сколько угодно перекрестных ссылок.

Таким образом, если рассматривать модель как граф, ребрами в котором являются ссылки, то любой цикл обязан проходить хотя бы по одной перекрестной ссылке, а агрегирующие ссылки в модели определяют остовный лес \cite{cormen01introduction}.

Явное выделение агрегирующих ссылок позволяет легко представлять модели в древовидной форме, как показано на \figref{ModelTree}.

\begin{figure}[htbp]
\centering
%\includegraphics[width=\textwidth]{model_tree.jpg}
\caption{Представление модели в древовидной форме}\label{ModelTree}
\end{figure}

\paragraph*{Параметризованные типы.} Поскольку \tool{Ecore} развивается на платформе \tool{Java}, с выходом \tool{Java} 5 в мета-мета-модель были добавлены \term{обобщенные типы} (generic types, \cite{GJ}). Любой класс или тип данных может иметь параметры, которые могут служить типами для структурных элементов.

\section{Текстовая нотация для мета-моделей}
//


\chapter{Предметно-ориентированные языки}

Одновременно с мета-моделированием развивается концепция \term{предметно-ориентированных языков (ПОЯ)} (Domain-Specific Languages, DSLs \cite{StateMachine}). В рамках этой концепции утверждается, что во многих предметных областях существуют типичные задачи, для решения которых целесообразно разрабатывать специализированные языки, позволяющие легко выразить специфичные для данной области понятия. ПОЯ противопоставляются \term{языкам общего назначения} (General-Purpose Languages, GPLs), таким, например, как популярные языки программирования (\tool{C}, \tool{C++}, \tool{Java}, \tool{C\#} и т.д.). Задачи, специфичные для данной области, можно решать и с помощью языков общего назначения, но при таком подходе решения получаются гораздо большими по объему и содержат много однотипного кода, который не всегда возможно выделить в библиотеку. Кроме того, поскольку ПОЯ оперируют понятиями предметной области, они становятся более доступными для понимания экспертами в этой области, не имеющими навыков программирования, что облегчает процесс общения с заказчиком и позволяет снизить затраты.

В качестве примеров ПОЯ можно привести
\begin{enumerate}
\item издательскую систему \TeX ;
\item язык разметки \tool{HTML};
\item языки, используемые в конфигурационных файлах, например, для веб-сервера \tool{Apache};
\item язык для вывода графики \tool{PostScript};
\item язык описания графов \tool{GraphViz};
\item нотация EBNF для описания контекстно-свободных грамматик;
\item структурированный язык запросов для реляционных баз данных \tool{SQL};
\item и многие другие.
\end{enumerate}

Некоторые ПОЯ, например \TeX , являются универсальными в том смысле, что с их помощью можно описать любое вычисление, однако они спроектированы так, что с их помощью легко решать задачи в соответствующей предметной области, а описывать другие вычисления --- гораздо сложнее (как правило сложнее, чем на языках общего назначения).

\chapter{Понятие языка}

Несмотря на то, что концепции мета-моделирования и ПОЯ развивались независимо, они тесно связаны между собой, поскольку мета-модели логично рассматривать как языки описания моделей (а мета-мета-модель, соответственно, как язык описания мета-моделей). Таким образом, \tool{Ecore} (MOF, KM3 и т. д.) можно рассматривать как ПОЯ для описания ПОЯ. Эта точка зрения приобретает все большую популярность \cite{}.

Для более глубокого понимания взаимосвязей мета-моделирования и ПОЯ требуется определить само понятие ``язык''. В разных областях математики, логики и компьютерных наук это делается по-разному \cite{???}, например в математической логике термин ``язык'' часто используется как синоним термина ``множество''; нередко имеется в виду множество, описанное определенным образом \cite{}. Здесь и далее мы не будем говорить о языках ``вообще'', а ограничим наше рассмотрение \term{языками моделей}:

\newcommand{\Lang}[1]{\mathcal{L}\left(#1\right)}%
\newcommand{\LMM}[1]{\Lang{\MM{#1}}}%

\begin{Def}
Пусть задана мета-модель $\MM{M}$. Будем называть множество всех моделей, являющихся экземплярами $\MM{M}$, \term{языком, порожденным этой мета-моделью}, и обозначать $\Lang{\MM{M}}$.
\end{Def}

Это определение позволяет описывать как языки моделирования, такие как UML \cite{UML}, которые изначально задаются с помощью мета-моделей, так и языки в более традиционном понимании. Например, в теории формальных языков \cite{???} принято определять язык как множество строк над конечным алфавитом; под наше определение подпадают те из языков, определенных таким образом, для которых есть конечное описание в виде (не обязательно контекстно-свободной) грамматики \cite{???}. Мета-модель, соответствующая грамматике состоит из классов, описывающих структуры термов соответствующей индуцированной алгебры \cite{???}.


\section{Шаблоны}

В предыдущем разделе мы видели, в какой мере модули могут быть выражены как инфраструктура нотации. Общий вывод таков: статическая композиция выражается легко, а динамическая --- не выражается. Это вполне естественно, поскольку само понятие инфраструктуры нотации основывается на \emph{статической} композиции, а не на динамической. Грубо говоря, инфраструктура описывает часть программы, которая может быть полностью вычислена во время компиляции.

Модули представляют собой самый простой механизм композиции. В данном разделе мы рассмотрим механизм \term{макроопределений} (или ``шаблонов''), который, не котором смысле обобщает механизм модулей и предоставляет возможности для более гибкого повторного использования.

\paragraph*{Происхождение термина ``шаблон''.} Термин ``шаблон'' (template) позаимствован из языка программирования \tool{C++} \cite{???}. Этот язык позволяет программисту определять не только классы и функции, но и шаблоны классов и функций --- параметризованные определения, которые можно \term{инстанцировать}, указав значения параметров. Как правило, параметрами являются типы, но можно использовать и константы. Изначально шаблоны были введены в язык для того, чтобы поддержать параметрический полиморфизм \cite{???}, который необходим для реализации удобной библиотеки контейнеров \cite{???}, однако оказалось, что шаблоны позволяют сделать гораздо больше \cite{???}, например, и их помощью можно вычислить во время компиляции любую рекурсивную функцию \cite{???}, то есть компилятор \tool{C++} представляет собой интерпретатор вычислительно универсального языка, основными элементами которого являются шаблоны.

Так или иначе, шаблоны являются частью инфраструктуры нотации языка, поскольку на этапе компиляции каждое упоминание шаблона разворачивается в результате подстановки аргументов на место параметров, и становится обыкновенной функцией или классом. Здесь необходимо заметить, что вследствие вычислительной универсальности шаблонов, процесс их разворачивания может никогда не закончиться (можно написать ``программу'' на шаблонах, порождающую бесконечную цепь рекурсивных обращений шаблонов к самим себе), этот сценарий мы будем считать ошибкой компиляции, то есть функция $Meaning$ для такого входа возвращает $\bot$.

// Haskell \textbf{Template} Meta-programming

\paragraph*{Происхождение термина ``макроопределение''.} Термин ``макроопределение'' (macro-definition или macro\footnote{В некоторых русскоязычных источниках можно встретить слово ``макрос'' (как единственное число существительного мужского рода в именительном падеже), которое образовано транслитерацией слова ``macros'', являющегося формой множественного числа. В силу его грамматической несогласованности, мы избегаем использования этого термина.}) восходит к языкам семейства \tool{Lisp} \cite{???}. Так называется определение в программе, часто --- параметризованное, на которое можно сослаться по имени как на функцию, но оно будет заменено в тексте программы до его выполнения. Например, в языке, где параметры функциям передаются по значению, повторно используемую подпрограмму для обмена переменных местами можно реализовать в виде макроопределения \cite{???}.

\begin{lstlisting}[language=Lisp,label=scheme_swap]
 (define-syntax-rule (swap x y)
    (let ([tmp x])
      (set! x y)
      (set! y tmp)))
\end{lstlisting}

Обращение к этому определению будет заменено на фрагмент программы, меняющий местами значения двух переменных, тогда как вызов функции менял бы местами значения их локальных копий в стеке.

Макроопределения (как и шаблоны) часто используются для создания встроенных предметно-ориентированных языков \cite{???}, поскольку они позволяют в некотором смысле расширить язык программирования новыми конструкциями (то есть, в каком-то смысле, добавить в нотацию новые инфраструктурные элементы).

Очевидны сходства макроопределений \tool{Lisp} с шаблонами \tool{C++}: и те, и другие являются параметризованными фрагментами программ, которые разворачиваются во время компиляции. 

\paragraph*{Обработка идентификаторов при разворачивании макроопределений и шаблонов.}
Разворачивание макроопределений (и шаблонов) связано с проблемой повторения идентификаторов. Если обратиться к макроопределению, меняющему местами значение двух переменных через переменную \code{tmp}, дважды, имя \code{tmp} будет объявлено дважды, что во многих языках вызовет ошибку. В других языках это может привести к изменению значения другой переменной, имя которой случайно совпало с \code{tmp}.

В некоторых системах (как, например, в упоминавшемся выше препроцессоре языка \tool{C}, поддерживающем также и макроопределения) решение этих проблем полностью возлагается на программиста: его внимательность и осторожность. Более совершенные системы гарантируют обнаружение подобных ошибок во время компиляции.

В случае шаблонов \tool{C++} задача относительно проста, поскольку тело шаблона (функция или класс) является пространством имен, поэтому необходимо лишь отслеживать отсутствие повторений в именах шаблонов, а также обеспечить различение результатов развертывания с различными параметрами. Так, компилятор считает использования одного и того же шаблона класса \code{std::vector<int>} и \code{std::vector<string>} разными типами, несмотря на то, что имя шаблона повторяется. Таким образом, идентификатором результата развертывания является имя шаблона и его аргументы. При многократном упоминании одного и того же идентификатора (например, \code{std::vector<string>}) шаблон разворачивается только один раз. Можно говорить о том, что в \tool{C++} проблема именования при развертывании шаблона решена \emph{ad hoc}, за счет ограничения набора элементов, которые могут быть результатом развертывания.

В \tool{Lisp}\footnote{Речь идет о семействе \tool{Lisp} 2 \cite{???}} эта проблема решается в более общем виде: с помощью \term{генерирования свежих имен} и \term{гигиены}. Генерирование свежих имен реализовано специальной операцией \code{gensym}, которая порождает имя, не занятое в текущей области видимости. ``Гигиеной'' (hygiene) называется система средств, позволяющих во время развертывания макроопределения проверять, определено ли уже то или иное имя, и переименовывать элементы в случае необходимости.

// Подробнее



Те же модули, но с параметрами (полиморфные или обобщенные)
	то есть включение, но с внедрением пользовательской информации. Контроль на стороне клиента
	LISP
	макросы в C
	С++
	m4/ST/Vel
	Haskell Template Meta-programming
	Nemerle
	Macros as MultiStage Computations: TypeSafe,Generative, Binding Macros in MacroML

\section{Аспекты}

Если смотреть на модули и шаблоны (макроопределения) как на механизмы композиции, они являются последовательными этапами на пути придания гибкости процессу комбинации фрагментов программы. Пусть есть два фрагмента: M (от Main, основная программа) и L (от Library, библиотека), и M должен использовать часть функциональности L. В этом случае, при использовании простых модулей, M должен выбрать фрагменты L, которые ему нужны, и использовать их. В случае шаблонов, M не только выбирает нужные ему фрагменты, но и может модифицировать их, подставляя собственные значения шаблонным параметрам. Сказанное иллюстрируется \figref{Composition}.

\begin{figure}[htbp]
// Картинка: кружочек, кружочек с дырками, подписано, что L, а что M
\caption{Виды композиции}\label{Composition}
\end{figure}

Развитие этой линии приводит нас к следующему механизму композиции: M вообще не обязан выбирать сам, L может самостоятельно определить, каким частям программы понадобятся те или иные элементы. Такой механизм является дополнительным к описанным выше; он лежит в основе \term{аспектно-ориентированного программирования} (АОП, \cite{AOP}).

\paragraph*{Язык \tool{AspectJ}.} АОП получило распространение благодаря языку \tool{AspectJ} \cite{AspectJ}, созданному на основе \tool{Java}, добавляя конструкции для нового способа композиции (сразу видно, что речь во многом идет об инфраструктурной функциональности). Основными концепциями в новом языке стали \term{точки присоединения} (join points), \term{срезы} (point-cuts) и \term{советы} (advice). Все эти элементы определяются внутри структурных элементов, называемых \term{аспектами}. Аспекты определяют в себе черты классов, объединяющих взаимосвязанные функции и данные, и модулей, хранящих независимые элементы, одновременно.

Аспектно-ориентированная композиция заключается в том, что код \term{совета} встраивается в основную программу. Например, для записи информации о ходе выполнения в журнал, требуется вписать однотипный код во множество мест в программе. Аспекты позволяют решить эту задачу, написав необходимый фрагмент кода только один раз.

Позиции, в которые код может встраиваться, называются \term{точками присоединения}. В \tool{AspectJ} это позиции перед и после вызовов функций, присваиваний, создания объектов и т. д. Для того, чтобы присоединить \term{совет} сразу ко множеству точек, это множество задается с помощью выражений, называемых \term{срезами}. Срезы описывают статическое или динамическое положение точки в программе, например ``вызов метода a()'' или ``создание объекта класса, имя которого начинается на A, которое происходит в потоке управления метода x()''. При присоединении \term{совета} к срезу указывается относительное положение: \term{совет} может присоединяться \term{до} (before), \term{после} (after) или \term{вместо} (around) каждой точки присоединения, соответствующей срезу.

Частным случаем такого подхода является добавление аспектом полей и методов в существующие классы, причем исходный код самих этих классов не изменяется. Этот механизм носит название ITD (Inter-Type Declarations). Например, специальный аспект может реализовывать методы \code{toString} для нескольких классов в одном файле, объединяя таким образом эту функциональность в одном модуле:

\begin{lstlisting}[language={[AspectJ]Java}]
aspect ToString {
	public String A.toString() {
		return "A: " + data;
	}

	public String B.toString() {
		return "B: " + data;
	}

	public String C.toString() {
		return "B: " + data + " " + moreData;
	}
}
\end{lstlisting}

\paragraph*{Характеристические свойства АОП.} С момента появления языка \tool{AspectJ} разработано множество концепций, так или иначе ``напоминающих'' идеи, заложенные в этом языке. В 2000 году в работе \cite{Obliviousness} была предпринята попытка выработать определение АОП, чтобы иметь эффективный критерий, позволяющий сказать, является ли данный язык аспектно-ориентированным. Название работы говорит само за себя: ``Aspect-oriented programming is quantification and obliviousness''\footnote{\textit{(англ.)} Аспектно-ориентированное программирование --- это квантификация и незнание.}
Под \term{квантификацией} (quantification) понимается возможность охарактеризовать множество точек присоединения предикатом, записанном на специальном языке. Под \term{незнанием} (obliviousness) подразумевается, что программа, в которою встраиваются \term{советы}, не знает о том, что они есть, то есть никак не зависит от их кода (код советов может зависеть от кода этой программы).

Таким образом, можно говорить о том, что аспекты --- это ``шаблоны наоборот'': не M использует L, дополняя ее своими фрагментами, а L ``сама'' встраивается в M.

\paragraph*{Аспекты как инфраструктурная функциональность.} \tool{AspectJ} статически компилируется в байт-код платформы \tool{Java}\footnote{Это не единственный способ присоединения аспектов к \tool{Java}-программам. Кроме него поддерживается \term{динамическое встраивание}, при котором байт-код модифицируется при загрузке классов во время выполнения}, из которого можно однозначно восстановить исходный код Java-программы. Это преобразование можно представить как трансформацию кода \tool{AspectJ} в код на \tool{Java} с последующей компиляцией. То есть аспекты можно реализовать как элемент инфраструктуры нотации \tool{AspectJ}.

// Не сказать ли про другие языки?


Сложные правила композиции (взаимопроникающие модули, инвазивная композиция): внедрение доп. информации без ведома клиента (слабее зависимости), прозрачность и т.д.
	
	AspectJ, терминология
	АПОЯ
		примеры для разных доменов
		примеры для грамматик


\section{Распространение рассмотренных механизмов композиции}

табличка: что где есть

мы рассмотрели вот эти языки. эти фичи есть почти везде, значит их приходится часто реализовывать, значит это надо автоматизировать.

\section{Диалекты}

// Рассказать про задачу разработки диалектов трех типов
// Сослаться на статьи про эволюцию языков


%\chapter{Инструменты для автоматизации разработки текстовых языков}

В данном разделе мы рассматриваем существующие инструменты, автоматизирующие разработку языков. В основном, эти инструменты ориентированы на разработку предметно-ориентированных языков, поскольку такие языки необходимо разрабатывать часто, что делает затраты на реализацию даже элементарных возможностей таких языков часто повторяющимися, в результате чего возникает необходимость в максимальном сокращении таких затрат.

В первую очередь нас интересует, насколько существующие подходы позволяют автоматизировать поддержку механизмов композиции, описанных выше, однако мы будем также отмечать и другие особенности этих подходов, в частности, поддержку механизмов композиции в языках, используемых самими инструментами.

\ignore{
\section{Контекстно-свободные грамматики}

--- Описание грамматик
	BNF
	EBNF (шаблоны)
	
	КСГ
	продукция 
	правило 
	правая часть 
	символ 
	терминал 
	нетерминал

	индуцированная алгебра термов
	частичный тип

\section{Атрибутная трансляция}

определения
JustAdd, ITD в нем

	атрибут
	семантическое действие

\section{Синтаксически-управляемая трансляция}

недостаток АТ

определение СУТ, L-атрибутные определения

Генераторы на основе СУТ
	Yacc/Bison и компания
		LALR, только синтез. аттр
	ANTLR
		LL, и те, и те
		
	внедренные семантические действия
}				
\section{Внутренние ПОЯ}

Часто ПОЯ разделяют на \term{внешние} (external) и \term{внутренние} (internal) \cite{StateMachine}. Внешними называют языки, имеющие специальные средства обработки, например, транслятор, написанные именно для данного ПОЯ. Внутренние языки, напротив, не имеют специальных средств обработки: они ``встраиваются'' в языки общего назначения, как библиотеки или расширения другого рода. В принципе, любой прикладной программный интерфейс (Application Programing Interface, API) можно рассматривать как внутренний ПОЯ: функции API выполняют роль ``ключевых слов'' внутреннего языка. Как правило, о внутренних языках говорят, если соответствующий язык общего назначения позволяет обращаться к библиотекам, используя достаточно гибкие синтаксические конструкции.

// Пример про fluent interfaces

Популярными языками для встраивания ПОЯ являются, например, \tool{Groovy} \cite{Groovy}, \tool{Haskell} \cite{Haskell98}, \tool{Scala} \cite{Odersky2008} и \tool{Java} \cite{JLS}. Существуют языки, имеющие специальные механизмы для расширения синтаксиса, позволяющие очень удобно интегрировать внутренние языки, например, \tool{Nemerle} \cite{Nemerle} и \tool{PLOT} \cite{PLOT}.
		
\section{Модульность грамматических определений}

Как ни странно, далеко не все популярные инструменты поддерживают повторное использование грамматических определений, например, \tool{Bison} \cite{BisonBook}, \tool{CoCo/R} \cite{CocoR} и \tool{JavaCC} \cite{JavaCC} не поддерживают никаких механизмов такого рода. Это связано в первую очередь с тем, что грамматические определения ``двумерны'': они содержат как описание синтаксической структуры (продукции КСГ), так и описание вычислений (семантические действия), что затрудняет композицию. Кроме того, специальные классы грамматик (например, $LL(k)$) не замкнуты относительно объединения, что накладывает дополнительные трудности на композицию \cite{???}. Так или иначе, инженерная дисциплина при разработке трансляторов находится на гораздо более низком уровне, чем при разработке других видов ПО \cite{Grammarware}.

В данном разделе мы рассмотрим механизмы повторного использования грамматических определений, основанные на цитировании, то есть модули и шаблоны. Аспектно-ориентированная композиция грамматик рассматривается в следующем разделе.

\subsection{Модули и ограничение видимости} В работе \cite{SysProg-2006} приведен обзор наиболее популярных средств для разработки трансляторов и выполнено сравнение по нескольким критериям, одним из которых является повторное использование.

Простейшая реализация модулей представлена в инструментах \tool{Elkhound} \cite{Elkhound} и \tool{ANTLR} версии 3 \cite{ANTLR}, который поддерживает директиву \code{include} для подключения определение из других файлов. 

Несколько иной подход использован в генераторе \tool{Menhir} \cite{Menhir}, который, принимая на вход несколько файлов, просто ``склеивает'' их содержимое вместе, но позволяет контролировать, какие символы являются открытыми (public) и могут быть использованы в других файлах, а какие --- нет. Закрытые (private) символы автоматически переименовываются для того, чтобы избежать коллизий. Особенность идеи ``склеивания'' файлов состоит в отсутствии директивы цитирования (\code{include} или \code{import}), что облегчает реализацию механизма, но делает зависимости между модулями \emph{невидимыми}: читая отдельный файл, трудно понять, какие из упоминаемых символов определены в других модулях, а главное --- нет никакой возможности определить, в каких именно модулях они определены.

Более традиционным образом модули организованы в системе \tool{ASF+SDF} \cite{ASF+SDF}: аналогично подходу, принятому в языках программирования, каждый модуль имеет имя, соответствующее имени файла, в котором он определен, элементы, объявленные в модуле, делятся на открытые (\code{exports}) и закрытые (\code{hidden}), причем другие модули могут использовать только открытые элементы, что обеспечивает инкапсуляцию. Директива цитирования \code{import} позволяет подключать модули друг к другу. При импортировании может оказаться, что символ, объявленный в одном модуле имеет то же имя, что и символ объявленный в другом. Как мы видели выше, в некоторых языках эта проблема решается с помощью квалифицированных имен (среди инструментов разработки трансляторов такого подхода придерживается \tool{Rats!} \cite{Rats!}); в \tool{ASF+SDF} используется другой подход: директива цитирования позволяет при необходимости \term{переименовать} символ, импортируемый из данного модуля, например:

\begin{lstlisting}
module example/Example

imports library/Lib[ A => B ]
\end{lstlisting}

Теперь внутри модуля \code{example/Example} символ \code{A}, определенный в \code{library/Lib} будет иметь имя \code{B}.

\subsection{Наследование в грамматических определениях}
% Compiler generation based on grammar inheritance
Идея \term{наследования грамматик} была впервые предложена в работе \cite{GInh}, и основывается на том, что грамматика может наследоваться от другой грамматики, добавляя новые правила и переопределяя старые. Первоначальная реализация была выполнена на языке \tool{Smalltalk} для рабочей станции \tool{Sun 3}, и не получила широкого распространения, однако в дальнейшем наследование грамматик было реализовано в других инструментах.

В \tool{ANTLR} версии 2 \cite{ParrQ95} наследование грамматик является единственным механизмом их повторного использования. Грамматика может быть унаследована от не более, чем одной родительской грамматики, при этом возможно переопределение правил, а именно: изменение семантических действий (при сохранении синтаксической структуры),  добавление новых альтернатив к существующим правилам. Кроме переопределения, также можно определять и совершенно новые правила (которые могут ссылаться на символы, определенные в родительской грамматике). К недостаткам такого подхода к повторному использованию можно отнести тот факт, что наследование является одиночным, и у разработчика нет возможности использовать несколько независимых модулей при разработке одного языка. Создатели \tool{ANTLR} 2 в качестве основного сценария использования наследования грамматик указывали создание нескольких диалектов одного языка \cite{???}.
% http://www.antlr2.org/doc/inheritance.html

Множественное наследование атрибутных грамматик предложено в работе \cite{MAGInh} и реализовано в инструменте \tool{LISA}. %http://portal.acm.org/citation.cfm?id=606666.606678
Авторы уделили больше внимание повторному использованию семантических действий, но и синтаксическая структура может быть унаследована, дополнена и частично переопределена.
Важным дополнением к наследованию грамматик в \tool{LISA} являются \term{шаблоны}.

В некоторых объектно-ориентированных языках (таких, например, как \tool{Scala} \cite{Odersky2008}) альтернативой наследованию является ``подмешивание'' (mixins). Похожий механизм для грамматик реализует инструмент \tool{xText} \cite{xText}. Результирующий механизм близок к множественному наследованию грамматик, но более строго и просто определяет поведение системы в случае ``ромбовидного наследования'' \cite{C++}. Mixin в \tool{xText} может добавлять новые правила или продукции, а также переопределять существующие.

Единственным известным нам инструментом, позволяющим не только добавлять, но и удалять продукции, является \tool{Rats!} \cite{Rats!}. Этот инструмент не используем явным образом метафору наследования для грамматик: авторы говорят лишь о ``модификации импортируемых модулей'', однако функциональность, реализованная этой операцией схожа с тем, что в других инструментах достигается с помощью наследования грамматик: есть возможность добавить продукцию, заменить или даже удалить ее.


\paragraph*{Параметризованные определения (шаблоны).}
Идею использования шаблонов (макроопределений для продукций или правил) при описании формальных грамматик легко проиллюстрировать на следующем примере: в синтаксисе языков программирования очень часто встречаются списки --- последовательности элементов (например, имен переменных), разделенных специальным символом (например, запятой). Элементы и разделители разнятся в зависимости от контекста. В грамматике языка \tool{Java} \cite{JLS} такие конструкции встречаются не менее 14 раз. Для того, чтобы решить проблему списков однажды и навсегда, можно определить следующий шаблон:

\begin{lstlisting}
	list<item, sep> -> item (sep item)*;
\end{lstlisting}

В результате возникает возможность коротко описывать такие разные языковые конструкции как вызов функции и арифметическое выражение:

\begin{lstlisting}
	functionCall -> NAME '(' list<expression, ','> ')';
	arithExpr -> list<product, plusOrMinus>;
\end{lstlisting}

Действительно, в скобках при вызове функции указывается список выражений, разделенных запятыми, а арифметическое выражение --- это алгебраическая сумма произведений, то есть список произведений, разделенных плюсами или минусами.

Стандартизированная нотация EBNF \cite{EBNF} имеет поддержку таких несложных шаблонов, хотя большинство инструментов реализует шаблоны по-своему. Прообразом грамматик с шаблонами можно считать двухуровневые грамматики \cite{???} использованные при описании \tool{Algol68} \cite{Algol68}, однако они не получили широкого распространения: идея была упрощена и приспособлена для понимания инженерами.

Реализация шаблонов в \tool{Menhir} наиболее близка к требованиям стандарта EBNF: индивидуальные правила могут иметь параметры, которым сопоставляются значения при использовании. В \tool{LISA} поддерживаются шаблоны для семантических действий: параметрами являются вхождения атрибутов.

Еще один способ использования шаблонов при описании грамматик состоит в определении не отдельных правил или продукций с параметрами, а в параметризации целой грамматики. Этот подход может служить хорошей альтернативой наследованию грамматик (он, фактически, аналогичен \term{делегированию} в объектно-ориентированных языках \cite{Gamma1995}). Например, грамматика, определяющая арифметические выражения, может принимать синтаксическую форму для атомарных выражений в виде параметра:

\begin{lstlisting}
grammar template Arith<atom> {
	sum    -> mult (('+' | '-') mult)*;
	mult   -> factor ('*' factor)*;
	factor -> '(' sum ')';
	       -> <atom>;
}
\end{lstlisting}

Этот шаблон позволяет порождать грамматики для арифметических выражений над разными (атомарными) объектами, например, над целочисленными константами и переменными:

\begin{lstlisting}
grammar Arith<INT | VAR>;

INT -> [0-9]+;
VAR -> [a-zA-Z_][a-zA-Z_0-9]*;
\end{lstlisting}

Или над комплексными константами:

\begin{lstlisting}
grammar Arith<complex>;

complex -> '(' FLOAT ',' FLOAT ')';
FLOAT   -> INT '.' INT;
\end{lstlisting}

Параметризацию модулей поддерживают инструменты \tool{ASF+SDF} и \tool{Rats!}, но эти возможности реализованы в них по-разному. \tool{Rats!} позволяет использовать в качестве параметров только модули целиком: параметризованный модуль может импортировать модуль-параметр и обращаться к символам внутри этого модуля. Такой подход в наибольшей степени схож с идеей делегирования в ООП: модуль рассматривается как класс, а набор имен символов --- как интерфейс\footnote{Здесь правомерно говорить об аналогии со ``статическим полиморфизмом'' в \tool{C++} \cite{C++}.}. В \tool{ASF+SDF}, напротив, модуль не может быть параметром, в качестве таковых используются только отдельные символы. 

Оба подхода имеют свои недостатки: в \tool{Rats!}, чтобы передать всего один символ в качестве параметра, придется создать модуль, а в \tool{ASF+SDF} неудобно передавать много символов за один раз. Кроме того, шаблоны отдельных правил и выражений в этих инструментах создавать нельзя. Нам не известно о существовании инструмента, поддерживающего все три способа параметризации.
	
\section{Аспектно-ориентированные грамматические определения}

Следующим шагом в развитии средств повторного использования грамматических определений является поддержка аспектов.

// Переписать рассуждения про то, откуда croscutting concerns в грамматиках
// Добавить объяснения про то, что grammar duplication и entanglement при генерации специфических тулов

% Separation of concerns in compiler development using aspect-orientation --- 2006
В работе \cite{Wu06} отмечается, что даже использование аспектно-ориентированного языка общего назначения (\tool{AspectJ}) облегчает разработку трансляторов. Многие инструменты интегрировали поддержку аспектов с более традиционной функциональностью, описанной выше.

Широкую известность получил подход, использованный в системе \tool{JastAdd} \cite{JastAdd}, основанной на расширенном определении атрибутных грамматик. В рамках этого подхода (использованного также и в других работах, см. \cite{Silver}) типы вершин AST рассматриваются как классы, а атрибуты определяются как методы с помощью механизма ITD.

Системы \tool{Silver} \cite{Silver} и \tool{AspectLISA} \cite{LISA} также используют аспекты для присоединения семантических действий к продукциям грамматики. Однако, если \tool{JastAdd} позволяет ссылаться лишь на имена символов и не имеет, строго говоря, механизма срезов (point-cuts), что делает квантификацию (см. раздел ???) довольно примитивной, то эти инструменты уже используют срезы для определения точек присоединения. \tool{Silver} позволяет цитировать целиком текст продукции, что гораздо слабее полноценного механизма срезов, используемого в \tool{AspectLISA}: этот инструмент позволяет использовать подстановочные знаки и параметры при определении срезов, аналогично тому как это сделано в \tool{AspectJ} \cite{AspectJ}.

Механизм срезов присутствует и в расширении \tool{ASF+SDF}, названном \tool{AspectASF} \cite{AspectASF}. Этот язык реализует вычисления над AST с помощью переписывающих правил \cite{TermRewriting}; срезы сопоставляют левые части и имена правил, а советы позволяют расширить функциональность, добавляя действия либо перед, либо после обработки правила.

\tool{AspectG} \cite{AspectG} также поддерживает срезы. Этот язык является аспектно-ориентированным расширением входного языка генератора \tool{ANTLR} \cite{ANTLR}. Особенность \tool{AspectG} состоит в том, что он поддерживает срезы, описывающие как грамматическую структуру, так и код семантических действий. Необходимо заметить, что семантические действия в \tool{ANTLR} задаются в виде простых строк, структура которых практически не анализируется, поэтому срезы для таких действий основываются на поиске образца в строке. Советы в \tool{AspectG}, как и рассмотренных ранее инструментах, позволяют только добавлять семантические действия, не изменяя грамматической структуры.

Полноценную модификацию грамматической структуры, как было отмечено выше, позволяет только \tool{Rats!}. Механизм использованный в этом инструменте можно описать как аспектно-ориентированный, основанный на примитивных срезах: все продукции в \tool{Rats!} поименованы, и обращение происходит по имени продукции.

% JastAdd vs Polyglot: Modularity First: A Case for Mixing AOP and Attribute Grammars '08

% AspectG va AspectLisa: Domain-specific aspect languages for modularising crosscutting concerns in grammar

\section{Выводы}

// Добавить мотивацию для Grammatic

Приведенный выше обзор показывает, что, несмотря на то, что всевозможные механизмы композиции успешно опробованы в разных инструментах, автоматизирующих разработку текстового синтаксиса, ни один из них не поддерживает одновременно все наиболее успешные методы композиции, а именно:
\begin{enumerate}
\item модули с поддержкой атрибутов видимости;
\item шаблоны всех элементов нотации (не только модулей, и не только выражений), параметризуемые любыми элементами нотации (не только модулями, и не только символами);
\item аспекты, поддерживающие полноценные срезы, и позволяющие не только оперировать семантическими действиями, но и модифицировать правила грамматики.
\end{enumerate}

Появление инструмента, поддерживающего все эти возможности, удовлетворило бы одновременно потребности разработчиков очень многих языков. Разработка такого инструмента входит в задачи данной работы.

	
\chapter{Автоматизация разработки механизмов композиции}

Из предыдущего раздела видно, что предметно-ориентированные языки (в данном случае --- языки, предназначенные для описания текстового синтаксиса) нуждаются в инфраструктурной функциональности, обеспечивающей композицию, не меньше, чем языки общего назначения. В данном разделе мы рассмотрим инструменты, позволяющие в той или иной степени автоматизировать разработку такой функциональности для предметно-ориентированных языков.

\section{Анализ идентификаторов}

Идентификация элементов является тривиальной задачей для средств разработки графического синтаксиса, поскольку они хранят элементы моделей как объекты в памяти, и могут использовать естественную индивидуальность (identity \cite{OOAD}) этих объектов для идентификации. При сохранении на диск в формате XMI \cite{XMI} также существует стандартизированный механизм идентификации, но получающееся таким образом текстовое представлением неудобно для чтения человеком. Ситуация не облегчается и стандартной нотацией HUTN \cite{HUTN}, поскольку ее синтаксис достаточно громоздок и не очень далек от XMI.

При разработке текстового синтаксиса анализ идентификаторов является третьим этапом создания транслятора \cite{DragonBook}, после лексического и синтаксического анализа. В простых случаях этот механизм легко реализуется с помощью таблицы символов, более сложные случаи требуют усложненных структур данных. Дополнительные трудности накладывает наличие различных пространств имен (например, когда множество имен функций может пересекаться с множеством имен переменных, потому что из синтаксиса всегда ясно, идет ли речь о переменной или о функции) и вложение этих областей. Основанный на атрибутной трансляции инструмент \tool{Eli} \cite{Eli} предоставляет обширную библиотеку для реализации различных вариаций механизма разрешения идентификаторов, но использование этой библиотеки по-прежнему требует написания достаточно большого объема кода.

Генераторы сред разработки, такие как \tool{xText} \cite{xText} и \tool{TCS} \cite{TCS} автоматизируют разрешение имен в простых случаях. \tool{TCS} позволяет определять вложенные области видимости и непересекающиеся пространства имен с помощью специальных директив внутри грамматического определения. 

\section{Механизмы простого цитирования}

Как отмечалось выше, самым простым механизмом композиции являются модули, не использующие параметризацию. Этот механизм позволяет объединять элементы, имеющие идентификаторы, в группы, называемые \term{модулями}. Такой модуль может быть \term{подключен} в некоторой области видимости (как правило в другом модуле), что позволяет использовать объявленные в нем элементы в этой области видимости.

Ряд инструментов, автоматизирующих разработку языков, позволяет добавлять поддержку модулей достаточно легко. Так, инструменты, опирающиеся на графический или псевдотекстовый синтаксис \cite{EMF, Fujaba, MPS} используют для реализации модулей ссылки между моделями, поддерживаемые мета-мета-моделями, использующими XMI \cite{XMI} (например, MOF или \tool{Ecore}), что не требует от разработчика описывать модули дополнительно. Вместе с тем, такая реализация модулей достаточно примитивна, например, она не позволяет вводить \term{атрибуты доступа}, то есть разделять элементы модуля на внутренние (private) и внешние (public).

Для языков с текстовым синтаксисом поддержка модулей реализуется несколько сложнее. Так упоминавшийся выше инструмент \tool{Eli} \cite{Eli} предоставляет библиотеку для реализации нескольких вариантов этого механизма, однако ее использование требует написания большого количества кода. Гораздо более простая реализация имеется в \tool{xText} \cite{xText}: этот инструмент предоставляет ``текстовую оболочку'' для механизма ссылок между моделями, реализованного в \tool{Ecore}, а именно распознает специальное имя атрибута (\code{importURI}) и интерпретирует его как однородный идентификатор ресурса (Uniform Resource Identifier, URI \cite{uri}), который автоматически загружается в текущую область видимости. Как следует из сказанного выше, такой механизм не имеет автоматической поддержки для атрибутов доступа.

Инструменты, позволяющие более полную автоматизацию поддержки модулей, нам не известны, однако ниже мы рассматриваем системы, частично автоматизирующие поддержку шаблонов. Поскольку модули, как мы видели выше, легко представить в виде шаблонов без параметров, можно считать, что средства описания шаблонов поддерживают также и модули.

\section{Шаблоны и макроопределения}

Автоматически шаблоны или макроопределения поддерживают только внутренние (internal) ПОЯ, определенные в языках с поддержкой соответствующих конструкций, таких как \tool{Lisp} или \tool{MacroML}.

Для самостоятельных языков поддержку шаблонов автоматизируют инструменты \tool{COMPOST} \cite{COMPOST} и \tool{Reuseware} \cite{Reuseware}. Последний является развитием первого, поэтому здесь мы опишем только его возможности.
 
Принцип, на котором основывается \tool{Reuseware} его авторы называют инвазивной композицией ПО (Invasive Software Composition, ISC \cite{ISC}). Это методология, определенная для моделей и КС-грамматик, базирующаяся на введение в структуру языка дополнительных элементов, так называемых \term{точек изменения} (variation points), которые подразделяются на следующие типы:
\begin{itemize}
\item \emph{slot} --- помечает выражение в языке для последующей замены, аналогично параметру шаблона со значением по умолчанию \cite{C++};
\item \emph{hook} --- помечает позицию для вставки элементов (возможно, многих), аналогично простому шаблонному параметру;
\item \emph{anchor} --- помечает выражение идентификатором, на который в дальнейшем можно сослаться при описании композиции (см. ниже).
\end{itemize}

Предложения, содержащие точки изменения, называются \term{фрагментами} (Component Fragment). Располагая набором фрагментов, разработчик может описывать их \term{композицию} с помощью специального языка, который определен в инструментах \tool{Reuseware} и имеет графический синтаксис. Композиция заключается в соединении элементов типа \emph{anchor} с элементами двух других типов, причем при присоединении к элементу типа \emph{slot} происходит замена его содержимого, а в случае элемента типа \emph{hook} --- дополнение.

Важной особенностью механизма ISC является тот факт, что он контролирует структурную корректность результата: точки изменения имеют типы (соответствующие классам в мета-модели или нетерминалам в грамматике), и соединение точек с несовместимыми типами запрещено семантикой языка описания композиции, что позволяет гарантировать, что результат не нарушает требований, наложенных целевой мета-моделью. Как мы увидим ниже, те же принципы можно использовать для автоматизированной реализации поддержки аспектов.

Недостатком такого подхода является тот факт, что он \term{незамкнут}: для того, чтобы использовать ``шаблоны'', определенные таким образом, нужно привлекать дополнительный язык, зависящий в свою очередь от инструментов \tool{Reuseware}, то есть расширенная версия исходного языка теряет самостоятельность. Кроме того, способ организации точек изменения гораздо менее локален по сравнению с традиционными шаблонами: ``телом шаблона является вся программа'', что затрудняет инкапсуляцию и делает весь подход менее интуитивным, поскольку теряется аналогия шаблонов с функциями, вычисляемыми во время компиляции \cite{MacroML}.

\section{Аспекты}

Разработку поддержки аспектов в той или иной мере автоматизируют несколько инструментов и методологий. Мы начнем с применения инструментария \tool{Reuseware}, рассмотренного в предыдущем разделе.

Система фрагментов и точек изменения может быть рассмотрена как ``заготовка'' для аспектной композиции. Обеспечить полное \term{незнание} (см. ???) не удается, поскольку точки изменения отмечаются явно, но некоторая степень незнания все же достигается за счет использования внешнего языка композиции. \term{Квантификацию} позволяет обеспечить механизм \term{запросов} (fragment queries), позволяющий коротко описывать множества точек изменения. Этот механизм выполняет функцию срезов. Поиск точек изменения осуществляется по фрагменту имени и типу.  Достоинства этого подхода --- в его универсальности, а недостатки применительно к данной задаче --- в первую очередь, в отсутствии незнания.

Альтернативный подход представлен в работе \cite{VanWyk03}, которая основана на атрибутной трансляции и переписывании. AST, созданное в процессе разбора программы может быть переписано в соответствии с декларативно заданными правилами. При этом поддерживается с \term{перенаправление} атрибутов (forwarding). Идея перенаправления состоит в том, что атрибут элемента AST может быть вычислен путем обращения к атрибуту элемента, полученного в результате переписывания исходного. В такой среде аспекты реализуются следующим образом: срез является предикатом, который определяет применимость того или иного совета для каждого элемента AST (в рамках данного подхода срезы реализуются как функции высшего порядка, написанные на языке Haskell), советы добавляются в код посредством создания новых правил переписывания, а перенаправление позволяет распространить семантические проверки на добавленный таким образом код. Преимуществом этого подхода является возможность добавить аспекты в любой язык независимо от других расширений, причем гарантируется сохранение корректности статических проверок. Основным недостатком является необходимость написания срезов на языке общего назначения, что, как и в случае \tool{Reuseware} делает систему незамкнутой.

Работа \cite{Bagge06} описывает метод, основанный на использовании системы \tool{Stratego/XT} \cite{Stratego/XT} --- широко известного инструмента для трансформации программ, основанного на \emph{стратегическом программировании} (strategic programming), представляющем собой объединение возможностей логического программирования и переписывания термов. Основная идея состоит в том, чтобы описывать применение аспектов как трансформации, во многом аналогично работе \cite{VanWyk03}. В данном случае авторы предлагают вручную реализовать и синтаксис, и семантику аспектов, демонстрируя, что это достаточно легко сделать, располагая инструментами \tool{Stratego/XT} и реализацией самого исходного языка. Строго говоря, эта работа описывает не технологию автоматизации, а метод ручной разработки.

Мы отмечали отсутствие поддержки срезов как недостаток описанных выше подходов. Методология автоматизированного построения системы срезов (но не остальных необходимых элементов аспектного языка) описана в работе \cite{XCPL}. Авторы предлагают использовать ``универсальный'' язык срезов \tool{XCPL}, позволяющий выразить очень широкий класс предикатов над программными конструкциями. Этот язык предлагается сокращать по мере надобности для описания срезов для каждого конкретного языка, при этом необходимо каждый раз разрабатывать конкретный синтаксис сокращенного языка. Семантика \tool{XCPL} также не фиксирована, а должна быть реализована вручную при каждом применении. Таким образом, \tool{XCPL} является просто мета-моделью, позволяющей описывать срезы, и не автоматизирует по-настоящему их разработку.

Необходимо отметить, что существует обширная литература, касающаяся реализации аспектов с точки зрения сред времени выполнения (см., например, \cite{JAMI}). Поскольку мы рассматриваем инфраструктурную функциональность языков, которая по природе своей реализуется во время компиляции, а не во время выполнения, рассмотрение этих работ выходит за пределы нашего обсуждения.
	POPART
	XAspects --- встраивание аспектных языков в AspectJ	
	JAMI --- общий рантайм для DSAL
	
\section{Выводы}

Приведенный обзор средств автоматизации разработки механизмов композиции показывает, что на данный момент ни один из рассмотренных подходов не позволяет автоматически реализовать в замкнутой форме шаблоны, а также аспекты с полной поддержкой квантификации и незнания.

\chapter{Постановка задачи}

Целью данной работы является создание технологии автоматизации разработки языков, поддерживающих автоматизированную реализацию сложных механизмов композиции, а именно
\begin{itemize}
\item шаблонов в замкнутой форме, конструирующих и принимающих в качестве параметров любые элементы языка, и обеспечивающих структурную корректность результатов;
\item аспектов в замкнутой форме, поддерживающие незнание и квантификацию (срезы), и обеспечивающих структурную корректность результатов.
\end{itemize}

Для достижения поставленной цели необходимо детально изучить механизмы композиции, указанные выше и их взаимодействие с другими частями языкового процессора. Для этого создан инструмент автоматизации разработки текстового синтаксиса \tool{Grammatic}, поддерживающий данные механизмы композиции, которые были реализованные вручную. Приемы, примененные при реализации \tool{Grammatic} обобщены для произвольные языки, что позволяет получить требуемую технологию.

Будем:
	- Разрабатывать язык для грамматик, чтобы понять чего и как
	- Обобщим то, что там наработаем
	- Найдем удобный способ описания языков
		- для графики
		- для текста
	- Научимся конвертировать ядро в инфраструктуру
%\part{Описание текстового синтаксиса с помощью \tool{Grammatic}}

Что и зачем

\chapter{Мотивация}

Большинство инструментов, автоматизирующих разработку текстового синтаксиса, используют контекстно-свободные грамматики, однако каждый из них использует их по-своему. Оставляя за рамками обсуждения различия в правилах записи продукций, рассмотрим содержательные различия выходных языков различных инструментов.

\section{Операторы}

В каноническом виде КСГ строятся с помощью двух операций: $\rightarrow$ для формирования продукции и конкатенация для формирования правой части продукции. В инструментах семейства \tool{Yacc} (например, \cite{YACC}), порождающих восходящие анализаторы, используются только эти две операции, что отвечает содержанию нотации BNF.

Однако при построении нисходящих анализаторов набор операций часто расширяют, используя нотацию EBNF. Это связано с тем, что конфликты в LL-грамматиках часто можно устранить, если использовать \term{итерацию}, стандартную операцию в языках регулярных выражений, часто обозначаемую ``*'' (альтернативное обозначение --- фигурные скобки ``\{\ldots\}''). Выражение $A^*$ означает, что цепочка $A$ может повторяться ноль или более раз. Также используют другие операции, применяемые в регулярных выражениях:
\begin{itemize}
\item $A^+$ --- повторение один или более раз;
\item $A^?$ --- вхождение ноль или один раз (альтернативное обозначение --- $[A]$);
\item $A\,|\,B$ --- вхождение $A$ или $B$.
\end{itemize}
Использование этих операций позволяет избежать большинства типичных проблем при разработке LL-грамматик \cite{LL1Conf}. Разные инструменты поддерживают дополнительные операции в разной степени: так, например, \tool{SableCC} \cite{SableCC} разрешает применять их только на верхнем уровне, и итерировать только одиночные символы.

\section{Описание лексических анализаторов}

При классическом подходе к построению трансляторов \cite{DragonBook} фазы лексического и синтаксического анализа описываются отдельно друг от друга. Так, многие известные инструменты состоят из двух (независимых) программ: генераторов лексических и синтаксических анализаторов (\tool{Lex/Yacc}, \tool{flex/Bison}, \tool{Alex/Happy} и т.д.). Входные нотации двух генераторов, как правило, различны, поскольку решают разные задачи. Кроме того, для описания лексических анализаторов чаще всего используются регулярные выражения, имеющие полный набор операций, описанных в предыдущем разделе, а для описания КСГ, как отмечалось выше, часто используется более узкий набор операций.

Однако многие разработанные в последние годы инструменты (например, ANTLR \cite{ANTLR}) позволяют описывать лексику и синтаксис языка в одной спецификации (с использованием одних и тех же операций), а некоторые инструменты (например, ASF+SDF \cite{ASF+SDF}) вообще обходятся без фазы лексического анализа, описывая грамматику языка над алфавитом отдельных символов, а не лексем.

\section{Аннотации}

Большинство инструментов не работают с КСГ ``в чистом виде'', а используют нотации, дополняющие грамматики какой-то информацией, например, семантическими продукциями для вычисления атрибутов \cite{LISA}, типами вершин AST \cite{xText}, метками для подвыражений \cite{Rats!}, директивами форматирования \cite{Pretzel}, семантическими предикатами \cite{ANTLR}, указаниями на семантическую роль терминалов \cite{ASF+SDF} и т.д. Практически каждый инструмент использует грамматики, \term{аннотированные} дополнительной информацией.

\section{Обобщенная нотация}

Нотация языка \tool{Grammatic} призвана обобщить подходы, используемые в других инструментах: она должна позволять описывать как лексическую, так и синтаксическую структуру языка, используя все дополнительные операции EBNF, и снабжать описание аннотациями произвольной сложности. Необходимо, чтобы входной формат любого другого инструмента можно было бы преобразовать в формат \tool{Grammatic} так, чтобы и обратное преобразование было возможно. Таким образом, \tool{Grammatic} становится ``общим знаменателем'' для нотаций, использующих КСГ. Кроме того, нашей целью является реализация механизмов композиции, о которых говорилось в предыдущей главе.

Единственной известной нам попыткой разработки обобщенного формата для представления грамматик, полученных из различных источников, является BGF \cite{RecoverJLS}. Этот формат поддерживает все операции EBNF, но никак не обрабатывает аннотации и не поддерживает механизмов композиции.

\section{Сценарии использования \tool{Grammatic}}

Наличие обобщенной нотации позволяет использовать \tool{Grammatic} как универсальный инструмент для разработки языков, основанный на принципах \term{порождающего программирования} \cite{CzarneckiEisenecker00}. Библиотека, поддерживающая обобщенную нотацию, о которой говорилось выше, позволяет преобразовывать аннотированную грамматику во внутреннее структурированное представление (в виде модели), которое может быть подано на вход различных генераторам. 

// Схема грамматика + метаданные --генератор--> код

// Пример про ANTLR

Как показывает практика использования \tool{Grammatic}, этот инструмент удобен для решения следующих задач:
\begin{itemize}
\item генерация лексических и синтаксических анализаторов;
\item генерация трансляторов общего назначения, описываемых с помощью АТГ;
\item автоматическое построение классов AST;
\item создание специальных трансляторов: например, для
	\begin{itemize}
		\item автоматического форматирования кода,
		\item подсветки синтаксиса,
		\item автодополнения в среде разработки;
	\end{itemize}
\item генерация документации;
\item анализ и преобразование грамматик.
\end{itemize}

Примеры применения \tool{Grammatic} для этих задач мы приводим ниже.

\chapter{Основные конструкции языка}

Язык \tool{Grammatic} описывается мета-моделью, приведенной в Приложении \bad{???}. В данном разделе мы описываем конструкции ядра \tool{Grammatic}.

\section{Синтаксические правила}

Структура языка соответствует стандартному определению КСГ (\figref{GCore}): \term{грамматика} представляет из себя набор \term{символов}, каждый из которых определяется одной или несколькими \term{продукциями}. В правой части продукций стоят \term{выражения}, построенные с помощью операций EBNF из символов данной грамматики и \term{терминальных определений}.

\begin{figure}[htbp]
\caption{Основные элементы мета-модели \tool{Grammatic}}\label{GCore}
\end{figure}

Нотация, используемая в \tool{Grammatic} для записи правил КСГ основана на нотации популярного генератора ANTLR, но отличается явным выделением продукций и пустого слова. Проиллюстрируем ее использование на примере языка арифметических выражений, заданного продукциями, представленными на рисунке \figref{ArithProd}:
\begin{lstlisting}
	s
		: VAR '=' e
		;
	e
		: e '+' e
		: e '*' e
		: '(' e ')'
		: INT
		: VAR
		;
\end{lstlisting}

\begin{figure}[htbp]
\newcommand{\gp}[2]{#1 & \rightarrow & #2 }
$$
\begin{array}{rcllrcl}
\multicolumn{7}{c}{
	\begin{array}{rcl}
		\gp{S}{\mathbf{var} \, = \, E}\\
	\end{array}
}\\
\gp{E}{E \, \mathbf{+} \, E}&\quad&
\gp{E}{E \, \mathbf{*} \, E}\\
\gp{E}{\mathbf{int}} &&
\gp{E}{\mathbf{var}} \\
\multicolumn{7}{c}{
	\begin{array}{rcl}
		\gp{E}{\mathbf{(} E \mathbf{)}}\\
	\end{array}
}\\
\end{array}
$$
\caption{КС-продукции, описывающие арифметические выражения}\label{ArithProd}
\end{figure}

Как видно из примера, левая и правая части продукции отделяются друг от друга двоеточием, причем символ в левой части пишется один раз для всех продукций. В данном примере определяются только ``синтаксические правила'' \code{s} и \code{e}. В принципе, символы \code{INT} и \code{VAR} ничем не отличаются с точки зрения нотации, однако при традиционном подходе эти символы были бы терминальными. \tool{Grammatic} не разделяет символы на терминалы и нетерминалы, поскольку, как говорилось выше, для некоторых алгоритмов разбора это разделение не имеет смысла. Таким образом, все символы в спецификации являются нетерминальными, а терминалы представлены литералами в одинарных кавычках. Несмотря на отсутствие принципиального раздGCoreеления с точки зрения языка, мы придерживаемся обозначений, разделяющих ``разные'' с нашей точки зрения типы символов: имена терминалов мы пишем заглавными буквами, а нетерминалов --- начиная со строчной буквы\footnote{Этот способ именования также позаимствован из языка спецификаций ANTLR, в котором он является обязательным.}.

Определения символов \code{INT} и \code{VAR} выглядят так:
\begin{lstlisting}
	INT : DIGIT+;
	VAR : IDEN_START IDEN_PART*;
	DIGIT : ['0'--'9'];
	IDEN_START : ['a'--'z''A'--'Z''_'];
	IDEN_PAR : digit | idenStart;
\end{lstlisting}
Этот пример иллюстрирует использование различных элементов нотации \tool{Grammatic}. Полный перечень представлен в \tabref{operations}.
\begin{figure}[htbp]
\center
	\begin{tabular}{|c|l|}
	\hline
	\bf Нотация & \bf Значение \\
	\hline
	\code{\#empty} & Пустое выражение \\
	\code{a} & Ссылка на символ a \\
	\code{a b} & Последовательность \\
	\code{a | b} & Альтернатива \\
	\code{a*} & Итерация от 0 до бесконечности \\
	\code{a+} & Итерация от 1 до бесконечности \\
	\code{a?} & Итерация от 0 до 1 \\
	\code{['a'--'z']} & Множество символов от 'a' до 'z' \\
	\code{'abc'} & Строка символов 'abc' \\
	\code{(a | b) c} & Круглые скобки для группировки выражений \\
	\hline
	\end{tabular}
	\caption{Выражения \tool{Grammatic}}\label{operations}
\end{figure}
Еще одним хорошим примером послужит описание нотации \tool{Grammatic} с помощью нее самой, приведенное в Приложении \bad{???}.

// Приоритеты бинарных операций

\section{Метаданные}

Метаданными называют описательные элементы программ, которые не обрабатываются компилятором, но и не являются комментариями: к ним могут получить доступ дополнительные инструменты, работающие с кодом. По аналогии с \tool{Java} метаданные поддерживаются и в \tool{Ecore}: каждому элементу мета-модели можно сопоставить произвольное количество \tool{аннотаций}. В следствие сходства с этими подходами, мы тоже называет аннотации в \tool{Grammatic} метаданными.

Как отмечалось выше, для решения различных задач грамматики необходимо снабжать аннотациями, которые будут позже использоваться генераторами. Типичным примером аннотаций являются семантические действия, но это далеко не единственный пример. Так, для построения программы-форматировщика, добавляющей в текст пробелы и переводы строк для того, чтобы текст выглядел ``структурно'', правила форматирования также задаются в виде аннотаций к грамматике, но не содержат инструкций для вычисления каких-либо атрибутов (подробно эта задача будет рассмотрена ниже в этой главе). То же касается и многих других типичных задач, связанных с созданием интегрированных сред разработки: подсветкой синтаксиса, сворачиванием блоков, построением визуального представления структуры программы и т.д. Заранее предвидеть все возможные применения системы нельзя, поэтому \tool{Grammatic} предоставляет гибкий механизм для описания аннотаций произвольной сложности.

Аннотации могут быть присоединены к любому элементу грамматики: символу, продукции, выражению или всей грамматике целиком. Каждая аннотация представляет из себя набор пар ``имя-значение''. В текстовом синтаксисе это оформляется с помощью фигурных скобок:
\begin{lstlisting}
	s{a = 10} 
		: {b = 'abc'} 
		  (e{c = asd})* { d = {x = 5; y = 6} };	
\end{lstlisting}
Пары ``\tool{имя = значение}'' мы будем называть \term{атрибутами}\footnote{Использование этого термина может вызвать путаницу с атрибутами в АТГ, но в нашем обсуждении из контекста всегда будет понятно, о каких атрибутах идет речь.}. В приведенном примере 
\begin{itemize}
\item атрибут \code{a} сопоставлен с символом \code{s} и имеет значение \code{10} (целое число); 
\item атрибут \code{b} сопоставлен с продукцией (такие аннотации пишутся сразу после двоеточия, начинающего продукцию) и имеет значение \code{'abc'} (строка);
\item атрибут \code{с} сопоставлен с вхождением символа \code{e} в правую часть продукции (а не с самим символом!) и имеет значение \code{abc} (идентификатор);
\item атрибут \code{d} сопоставлен с выражением \code{e*} и имеет значение \code{{x = 5; y = 6}} (аннотация, состоящая из двух атрибутов).
\end{itemize}
Предопределенные типы значений, приведенные в \tabref{valtypes}, позволяют создавать довольно сложные аннотации. Наибольшую свободу предоставляет тип ``Последовательность'', значения которого являются последовательностями значений других типов и знаков препинания. Ниже мы увидим, как с помощью таких атрибутов можно создавать небольшие ``предметно-ориентированные языки'' внутри \tool{Grammatic}.
\begin{figure}[htbp]
\center
	\begin{tabular}{|c|l|}
	\hline
	\bf Нотация & \bf Значение \\
	\hline
	\code{'abc'} & Строка \\
	\code{10} & Целое число \\
	\code{abc} & Идентификатор \\
	\code{\{ a = b; c = 10\}} & Аннотация \\
	\code{ \{\{ a, b, c ; \}\} } & Последовательность \\
	\code{ <{<} s | (a b)* {>}> } & Грамматическое выражение \\
	\hline
	\end{tabular}
	\caption{Предопределенные типы значений атрибутов}\label{valtypes}
\end{figure}
Однако для некоторых целей предопределенных типов не хватает. В этом случае можно определять пользовательские типы значений, что достигается расширением иерархии классов мета-модели (см. Приложение \bad{???}).

В некоторых случаях атрибут играет роль флага: его значение не важно, а роль играет только наличие или отсутствие атрибута. В таких случаях значение атрибута можно не указывать: указывается только имя (без знака равенства). Например, при генерации трансляторов удобно помечать некоторые символы грамматики (как правило ``терминальные'') как элементы форматирования (whitespace) --- чтобы анализатор их игнорировал. Это актуально не только для настоящих символов форматирования (пробелов, переводов строк, табуляции), но и для комментариев. Для такой реализации пометки достаточно указать атрибут без значения:
\begin{lstlisting}
	WS{ignore} : [0x0000-0x0020];
\end{lstlisting}
Заметим, что имя атрибута, которое нужно указать, зависит от конкретного генератора, для которого пишется спецификация. Семантика языка \tool{Grammatic} никак не интерпретирует атрибуты.

Поскольку в одной грамматике могут встречаться атрибуты, предназначенные для разных генераторов, необходимо обеспечить уникальность имен, чтобы атрибуты не ``накладывались''. Это достигается с помощью введения \term{пространств имен} для атрибутов. Пространство имен идентифицируется однородным идентификатором ресурса (Uniform Resource Identifier, URI \cite{uri}), внутри данной грамматики для удобства ему присваивается локальное имя, это делается с помощью директивы \code{namespace}:
\begin{lstlisting}
	namespace example 'http://example.com/Namespace/Example';
\end{lstlisting}
Для нужд самого \tool{Grammatic} выделено системное пространство имен \code{system} с URI \code{grammatic:/}. По умолчанию все атрибуты определяются в системном пространстве имен. Чтобы указать другое пространство имен, используется квалифицированное имя атрибута, например:
\begin{lstlisting}
	A{example.size = 10}
\end{lstlisting}
Различные генераторы должны определять свои пространства имен, чтобы избегать наложения атрибутов.

\section{Пример: интерпретатор арифметических выражений}

показать примеры аннотаций: ATG/SDT, ASF+SDF, Bison, Pretzel, xText

\begin{lstlisting}
	WS{ignore} : [0x0000-0x0020];
	COMMENT{ignore} : '//' [^'\n']*;
	DIGIT{lexical; fragment} : ['0'--'9'];
	IDEN_START{lexical; fragment} : ['a'--'z''A'--'Z''_'];
	IDEN_PAR{lexical; fragment} : digit | idenStart;
	VAR{lexical} : IDEN_START IDEN_PART*;
	INT{lexical} : DIGIT+;
\end{lstlisting}

\begin{lstlisting}
{
	priorities = {{ '+' < '*' }};
}
\end{lstlisting}

\begin{lstlisting}
	e{returns=int} 
		: e{assignTo=a} '+' e{assignTo=b; code='result = a + b;'}
		: e{assignTo=a} '*' e{assignTo=b; code='result = a * b;'}
		: INT{assignTo=a; code='result = stringToInt(a);'}
		: '(' e{assignTo=a; code='result = a;'} ')'
		;
\end{lstlisting}

\begin{lstlisting}
	s
		: { scope = {{ put(s.scope, VAR, e.value) }} }
		VAR '=' e
		;
	e
		: { value = {{ e[1].value + e[2].value }} }
		  e '+' e
		: { value = {{ e[1].value * e[2].value }} }
		  e '*' e
		: { value = {{ stringToInt(INT) }} }
		  INT
		: { value = {{ get(s[].scope, VAR) }} }
		  VAR
		: { value = {{ e[1] }} }
		  '(' e ')'
		;
\end{lstlisting}

\section{Пример: язык для описания конечных автоматов}

Для демонстрации возможностей \tool{Grammatic} ниже мы будем использовать предметно-ориентированный язык StateMachine, предложенный М. Фаулером \cite{StateMachine} и широко используемый в литературе по ПОЯ (см., например,~\cite{Zdun}). В данном разделе мы приводим неформальное описание этого языка, дающее представление о его назначении и содержании.

Язык StateMachine позволяет описывать простые конечные автоматы Мура \cite{???}, то есть автоматы, допускающие только действиями в состояниях, но не на переходах. Целевая мета-модель этого языка представлена на \figref{SMMM}.

\begin{figure}[htbp]
	\caption{Целевая мета-модель языка StateMachine}\label{SMMM}
\end{figure}

В графической состояния изображаются вершинами графа, переходы --- направленными ребрами (см. \figref{SM}). В вершинах над горизонтальной чертой пишется уникальный идентификатор состояния, а под чертой --- выходные воздействия, генерируемые в этом состоянии. На переходах пишутся входные воздействия, их инициирующие.

\begin{figure}[htbp]
	\centering
	\includegraphics[scale=.7]{smgraph.png}
	\caption{Графическая нотация для языка StateMachine (из \cite{StateMachine})}\label{SM}
\end{figure}

В текстовой нотации автомат описывается с помощью ключевого слова \code{statemachine}, за которым следуем имя автомата и набор состояний в фигурных скобках. Пример использования текстовой нотации приведен в \lstref{SMText}. Cостояния описываются с помощью ключевого слова \code{state}; внутри состояния может находиться блок \code{do}, содержащий последовательность выходных воздействий, а также переходы, описываемые с помощью конструкции \code{on INPUT goto STATE}, где \code{INPUT} --- входное воздействие, а \code{STATE} --- состояние, в которое осуществляется переход.

Входные и выходные воздействия описываются вне блока \code{statemachine} с помощью ключевого слова \code{event}; каждое воздействие имеет уникальное имя и целочисленный код.

\begin{lstlisting}[label=SMText,float=htbp,caption=Текстовая нотация языка StateMachine]
event lockDoor 0; event unlockDoor 1;
event lockPanel 2; event unlockPanel 3;
event doorClosed 4; event doorOpened 5;
event lightOn 6; event drawOpened 7;
event panelClosed 8;

statemachine SecretCompartment {
	state idle {
		do {
			unlockDoor;
			lockPanel;
		}
		on doorClosed goto active;
	}
	state active {
		on lightOn goto waitingForDraw;
		on drawOpened goto waitingForLight;
	}
	state waitingForDraw {
		on drawOpened goto unlockedPanel;
	}
	state waitingForLight {
		on lightOn goto unlockedPanel;
	}
	state unlockedPanel {
		do {
			unlockPanel;
			lockDoor;
		}
		on panelClosed goto idle;
	}
}
\end{lstlisting}

Синтаксические правила языка StateMachine, выраженные в нотации \tool{Grammatic}, приведены в \lstref{SMGram}.

\begin{lstlisting}[xleftmargin=1cm,label=SMGram,caption=Грамматика языка StateMachine]
WS : [0x0000-0x0020];
COMMENT : '//' [^'\n']*;
DIGIT : ['0'-'9'];
NAME_START : ['a'-'z''A'-'Z'_];
NAME_PART : NAME_START | DIGIT;
NAME : NAME_START NAME_PART*;
INT : DIGIT*;

system : event* stateMachine;
stateMachine : 'statemachine' NAME '{' state* '}';
state : 'state' NAME '{' do? (transition ';')* '}';
do : 'do' block;
transition : 'on' eventRef 'goto' stateRef;
stateRef : NAME;
block : '{' (commandRef ';')* '}';
eventRef : NAME;
commandRef : NAME;
event : 'event' NAME INT;
\end{lstlisting}

\chapter{Модули}

Поддержка модулей позволяет разработчику разделять спецификацию на несколько отдельных файлов и при необходимости использовать определения из одного файла повторно.  С точки зрения целевой мета-модели \tool{Grammatic}, модулю соответствует грамматика, определенная в отдельном файле. В данном разделе мы подробно опишем механизм работы таких модулей.

\section{Цитирование и переименование}

Для описания модулей в \tool{Grammatic} не применяется никаких специальных синтаксических конструкций, поэтому файл, содержащий описание грамматики, уже является модулем. Для того, чтобы использовать один модуль внутри другого, применяется директива цитирования \tool{import}, синтаксис которой можно проиллюстрировать на следующем примере:
\begin{lstlisting}
import 'a/b/c/d.grammar' {A, B as C};

B : A C*;
\end{lstlisting}  
Директива \code{import} принимает два аргумента: идентификатор импортируемого файла в одинарных кавычках и список импортируемых символов в фигурных скобках. 

Идентификатором файла является его имя в \term{виртуальной файловой системе}, конфигурация которой подается на вход транслятору \tool{Grammatic} вместе с файлом основной грамматики. В простейшем случае (пустая конфигурация) виртуальная файловая система в точности соответствует физической, и идентификатором файла является просто путь на диске. Однако при использовании библиотек привязка к путям в физической файловой системе приводит к трудностям с переносимостью, поэтому виртуальная файловая система может предоставлять абстрактное представление физической, самостоятельно находя библиотечные модули. Подробнее формат описания виртуальной файловой системы разобран в Приложении \bad{???}.

Список импортированных символов указывается явно для того, чтобы подчеркнуть характер зависимости данного модуля от подключаемого. При необходимости импортировать все символы, список можно заменить знаком \code{\{*\}}.

Ключевое слово \code{as} используется в случае необходимости импортировать символ под другим именем. Так в нашем примере символ \code{B} переименовать необходимо, поскольку в импортирующем модуле определен символ с таким именем. В результате, правило, описанное здесь соответствует диаграмме на \figref{GRenaming}.

\begin{figure}[htbp]
	\caption{Иллюстрация результата переименования}\label{GRenaming}
\end{figure}

Другая форма директивы цитирования позволяет назначить имя самому импортируемому модулю и обращаться к его элементом с помощью квалифицированных имен:
\begin{lstlisting}
import 'a/b/c/d.grammar' as D;

B : D.A D.B*;
\end{lstlisting}  
Заметим, что при трансляции данного примера результат будет идентичным предыдущему, поскольку в правиле для символа \code{B} мы использовали те же символы из подключаемого модуля (\code{A} и \code{B}) в тех же позициях.

\section{Атрибуты доступа}

Нотация \tool{Grammatic} не предусматривает специальных средств для обозначения атрибутов доступа для правил. Тем не менее, соответствующую функциональность обеспечивают специальные атрибуты \code{private} и \code{public}, зарезервированные для этих целей в системном пространстве имен.

\tool{Grammatic} проверяет наличие атрибута \code{private}, и если он есть, запрещает импортировать символ или использовать его квалифицированное имя. Атрибут \code{public} определен для симметрии, и его можно не указывать. Так, в следующем примере недоступен другим модулям только символ \code{C}:
\begin{lstlisting}
A{public} : B C;
B : C;
C{private} : 'c'
\end{lstlisting}  

\section{Модульная грамматика для языка StateMachine}\label{ModularSMG}

В качестве примера, разделим на модули грамматику языка StateMachine, приведенную в \lstref{SMGram}. Первый модуль \code{smlexer.grammar} будет содержать ``лексические'' определения, которые будут использованы в других модулях:
\begin{lstlisting}
// smlexer.grammar
WS{private} : [0x0000-0x0020];
COMMENT{private} : '//' [^'\n']*;
DIGIT{private} : ['0'-'9'];
NAME_START{private} : ['a'-'z''A'-'Z'_];
NAME_PART{private} : NAME_START | DIGIT;
NAME : NAME_START NAME_PART*;
INT : DIGIT*;
\end{lstlisting}
Заметим, что доступными извне являются только символы \code{INT} и \code{NAME}, поскольку все остальные символы носят служебный характер.

Основные синтаксические правила мы разделим на два модуля: \code{smmain.grammar}, определяющий структуру автомата, и \code{smevents.grammar}, описывающий входные и выходные воздействия:
\begin{lstlisting}
// smmain.grammar

import 
	'smlexer.grammar' {NAME},
	'smevents.grammar' {*};

system : event* stateMachine;
stateMachine : 'statemachine' NAME '{' state* '}';
state : 'state' NAME '{' do? (transition ';')* '}';
do : 'do' block;
transition : 'on' eventRef 'goto' stateRef;
stateRef : NAME;
block : '{' (commandRef ';')* '}';

// smevents.grammar
import 
	'smlexer.grammar' {NAME, INT};
	
eventRef : NAME;
commandRef : NAME;
event : 'event' NAME INT;
\end{lstlisting}
Данный пример демонстрирует возможность повторного использования модуля \code{smlexer.grammar}.

\chapter{Шаблоны}

Как отмечалось выше, шаблоны (макроопределения) позволяют повторно использовать фрагменты грамматики, внося в них некоторые изменения, посредством подстановки на место параметров шаблона реальных значений. Например, если в нотации языка StateMachine мы бы хотели сделать последнюю точку с запятой в списках выходных воздействий и переходов необязательной (как в языке Pascal), нам пришлось бы дважды написать относительно запутанное выражение:
\begin{lstlisting}
state : 'state' NAME '{' do? 
			(transition (';' transition)* ';'?)* 
		'}';
block : '{' 
			(commandRef (';' commandRef)* ';'?)* 
		'}';
\end{lstlisting}

В грамматиках больш\'{и}х языков повторения подобных конструкций могут встречаться десятки раз. Чтобы избежать дублирования, можно определить шаблон вида ``\code{элемент~(разделитель~элемент)*~разделитель?}'', и применить его дважды, подставляя разные значения вместо параметров ``\code{разделитель}'' и ``\code{элемент}'', что позволяет сократить длину кода, улучшить его читаемость и уменьшить вероятность появления ошибок.

В нотации \tool{Grammatic} шаблоны объявляются следующим образом:
\begin{lstlisting}
template List<element : Expression, sep : Expression> : Expression {
	<element> (<sep> <element>)* <sep>?
}
\end{lstlisting}
Первая строка определяет \term{сигнатуру шаблона}, как мы покажем ниже, она используется для гарантий структурной корректности результатов применения шаблона. После двоеточия указывается тип параметра (или результата). Типы соответствуют классам мета-модели \code{Grammatic} (см. \figref{GCore}). Часто (в частности, в данном примере) типы можно опускать, поскольку транслятор способен определить их самостоятельно. В целом, шаблон напоминает функцию: у него есть имя, аргументы и возвращаемое значение; единственное отличие состоит в том, что при применении шаблона результат вычисляется транслятором статически, а не ``во время выполнения''\footnote{Что такое ``время выполнения'' для языка \code{Grammatic} не вполне ясно, тем не менее, что понятие ``статическиого вычисления'' не вызывает сомнений: такие вычисления выполняются транслятором в процессе вычисления функции $Meaning$.}. Аналогично функциям, чтобы использовать ранее определенный шаблон, нужно указать его имя и параметры:
\begin{lstlisting}
state : 'state' NAME '{' do? 
			<List transition, ';'>
		'}';
block : '{' 
			<List commandRef, ';'>
		'}';
\end{lstlisting}
Содержание этого определения символов \code{state} и \code{block} идентично приведенному выше.

Теперь приступим к более детальному описанию механизма шаблонов в \tool{Grammatic}.

\section{Элементы языка шаблонов}
Целевая мета-модель, описывающая шаблоны, приведена на \figref{TempMM}.

\begin{figure}[htbp]
	\centering
	\caption{Целевая мета-модель языка шаблонов}\label{TempMM}
\end{figure}

Выше было показано, что шаблоны реализуются как инфраструктурная функциональность. Это означает, что, добавляя шаблоны в язык, мы расширяем основную нотацию, добавляя новые конструкции так, чтобы получившуюся нотацию можно было транслировать в исходную. Можно считать, что мы создаем новый язык: на основе уже описанного языка грамматик строим языка шаблонов грамматик.

В этом новом языке центральным понятием является \term{шаблонное выражение}: то, что может являться телом шаблона. На \figref{TempMM} шаблонным выражениям соответствуют классы-наследники TemplateExpression. Простейшие шаблонные выражения --- это константы, то есть обычные выражения языка описания грамматик, не содержащие шаблонных параметров и обращений к шаблонам. Например, выражение \code{A B* | C}, как и все его подвыражения, является примером такой константы. Таким образом, нотация для шаблонных выражений включает в себя нотацию для грамматик как подмножество. Заметим, что шаблонные выражения имеют типы, соответствующие классам в целевой мета-модели языка грамматик.

Новыми элементами в нотации шаблонных выражений являются обращения к шаблонным параметрам (класс ParameterUsage), которые записываются в угловых скобках (\code{A <paramName>* | C}), и вызовы шаблонов (класс TemplateApplication), которые также заключаются в угловые скобки, но, кроме имени, содержат также список аргументов шаблона, разделенных запятыми:
\begin{lstlisting}
	<List expression, ';'>
\end{lstlisting}
Заметим, что, если шаблон не имеет параметров (то есть является константным), вызов такого шаблона синтаксически неотличим от обращения к параметру, что выражает близость этих двух понятий.

Шаблонные выражения записываются внутри объявлений шаблонов (класс Template), и представляют собой \term{тело} шаблона. Синтаксически объявление шаблона оформляется с помощью ключевого слова \code{template}:
\begin{lstlisting}
template List<element : Expression, sep : Expression> : Expression {
	<element> (<sep> <element>)* <sep>?
}
\end{lstlisting}
Как мы отмечали выше, типы, указанные в объявлении, используются для гарантий корректности результата применения данного шаблона. Шаблонные параметры, описанные в сигнатуре шаблона, могут быть использованы только в его теле. Объявления шаблонов не вкладываются одно в другое, также не допускаются рекурсивные вызовы шаблонов (прямые или косвенные), поскольку вычисление значения при вызове такого шаблона никогда не закончится.

Типы, используемые при объявлении шаблонов строятся по следующим правилам: 
\begin{itemize}
\item \term{Элементарные типы} --- это классы мета-модели грамматик (Expression, Sequence, Production, Symbol и т.д.).
\item \term{Составные типы} образуются из элементарных следующими операциями:
	\begin{itemize}
		\item ``?'' значение элементарного типа является необязательным;
		\item ``+'' непустой список значений элементарного типа;
		\item ``*'' произвольный список значений элементарного типа.
	\end{itemize}
\end{itemize}

\section{Примеры использования шаблонов}

Выше мы показали, как и для чего могут быть использованы шаблоны отдельных выражений. Для таких целей шаблоны применяются в EBNF \cite{EBNF} и \tool{Menhir} \cite{Menhir}. В Главе \ref{part1} мы также описывали параметризованные модули, используемые в \tool{ASF+SDF} и \tool{Rats!}, в данном разделе мы покажем, как шаблоны \tool{Grammatic} позволяют реализовать аналогичную функциональность. Кроме того, мы покажем, что можно использовать и шаблоны метаданных.

\subsection{Параметризованные модули}

Мы разработаем вариант описания языка StateMachine, который позволит нам варьировать содержание понятий входного и выходного воздействия. В разделе \ref{ModularSMG} мы выделили три модуля в грамматике языка StateMachine, один из которых, \code{smevents.gramar} содержал определения для символов \code{even}, \code{eventRef} и \code{commandRef}, которые использовались в главном модуле. Такая реализация уже позволяет изменять определения данных символов, не изменяя кода главного модуля, но она не позволяет использовать \term{параллельно} два варианта реализации понятия ``воздействие''. Для того, чтобы поддержать эту возможность, мы видоизменим главный модуль, введя шаблонные параметры:
\begin{lstlisting}
// smmain.grammar

import 
	'smlexer.grammar' {NAME};

template SMMain<event : Expression, eventRef : Expression, 
		commandRef : Expression> : Symbol+ 
{
	system : <event>* stateMachine;
	stateMachine : 'statemachine' NAME '{' state* '}';
	state : 'state' NAME '{' do? (transition ';')* '}';
	do : 'do' block;
	transition : 'on' <eventRef> 'goto' stateRef;
	stateRef : NAME;
	block : '{' (<commandRef> ';')* '}';
}
\end{lstlisting}
Важно заметить, что модуль \code{smevents.grammar} больше не используется, а вместо обращений к символам этого модуля введены шаблонные параметры.

Теперь определим два сосуществующих варианта описания входных и выходных воздействий: первый --- тот, что уже использовался, а второй --- позволяющий сопоставить каждому воздействию строку для записи в журнал событий. Первый вариант реализует уже знакомый нам модуль \code{smevents.grammar}:
\begin{lstlisting}
// smevents.grammar
import 
	'smlexer.grammar' {NAME, INT};
	
eventRef : NAME;
commandRef : NAME;
event : 'event' NAME INT;
\end{lstlisting}
Второй вариант мы реализуем в модуле \code{smevents-log.grammar}:
\begin{lstlisting}
// smevents-log.grammar
import 
	'smlexer.grammar' {NAME, INT};
	
logEventRef : NAME;
logCommandRef : NAME;
logEvent : 'event' NAME INT STRING?;

STRING : '\'' [^'\'''\n''\r']* '\'';
\end{lstlisting}

Теперь необходимо соединить каждую из этих реализаций с главным модулем. Для этого достаточно применить определенный этим модулем шаблон и передать соответствующие параметры. Для первого случая получим
\begin{lstlisting}
<SMMain event, eventRef, commandRef>
\end{lstlisting}
Для второго случая получим
\begin{lstlisting}
<SMMain logEvent, logEventRef, logCommandRef>
\end{lstlisting}
Шаблоны делают данные определения гораздо компактнее: если бы нам пришлось обходиться обычными модулями, определение главного модуля пришлось бы записать дважды --- по одному разу для каждого варианта реализации воздействий.

// Параметризация грамматики другими грамматиками

// импортирование vs встраивание

\subsection{Шаблоны метаданных}

Поскольку метаданные тоже входят в основную нотацию \tool{Grammatic}, можно создавать и шаблоны аннотаций. В основном, такие шаблоны понадобятся нам для реализации механизма аспектов, а в этом разделе мы приведем простой пример, показывающий, как с их помощью можно регламентировать структуру аннотаций.

Генератор документации для грамматик использует метаданные как источник дополнительной информации: специальные атрибуты хранят текст, описывающий назначение символа, примеры строк, которые он выводит и т.д. Для примера приведем описание оператора \code{if} некоторого гипотетического языка:
\begin{lstlisting}
if
{	doc.title = 'Conditional operator';
	doc.description = 
'If expression evaluates to true, <code>then</code> branch is taken,
otherwise -- <code>else</code> branch is taken';
	doc.examples = {{
		'if a > 1 then WriteLn(a)'
		'if (x < 0) and (x > -5) then x := -x else x := 2 * x'
	}}
} : 'if' expression 'then' statement ('else' statement) ;
\end{lstlisting}

Из такого описания генератор построит следующий текст:
\begin{center}
\fbox{
\parbox{0.7\textwidth}{
{{\bf Conditional operator}\\
\small
{\bf Syntax}: \\
\textit{if} ::= \texttt{if} \textit{expression} \texttt{then} \textit{statement} \texttt{else} \textit{statement}\\
{\bf Description}: 
If expression evaluates to true, \texttt{then} branch is taken,
otherwise -- \texttt{else} branch is taken\\
{\bf Examples}: 
\begin{itemize}
\item \textbf{if} a > 1 \textbf{then} WriteLn(a)
\item \textbf{if} (x < 0) \textbf{and} (x > -5) \textbf{then} x := -x \textbf{else} x := 2 * x
\end{itemize}
}}}
\end{center}

Мы используем шаблон аннотации для того, чтобы гарантировать, что разработчик укажет все атрибуты, необходимые для генерации документации:
\begin{lstlisting}
template Doc<title : String, description : String, 
	examples : String+> : Attribute+ 
{
	doc.title = <title>;
	doc.description = <description>;
	doc.examples = {{ <examples> }};
}
\end{lstlisting}
С применением этого шаблона метаданные для символа \code{if} будут выглядеть так:
\begin{lstlisting}
if
{Doc<
'Conditional operator',
'If expression evaluates to true, <code>then</code> branch is taken,
otherwise -- <code>else</code> branch is taken',
'if a > 1 then WriteLn(a)',
'if (x < 0) and (x > -5) then x := -x else x := 2 * x'
>} : 'if' expression 'then' statement ('else' statement) ;
\end{lstlisting}
В этом случае \tool{Grammatic} выдаст сообщение об ошибке, если один из элементов документации не будет указан.

\chapter{Аспекты}%
%
Еще одним механизмом композиции, реализованным в \tool{Grammatic} являются аспекты. Как мы отмечали выше, этот механизм обеспечивает выполнение двух основополагающих свойств:
\begin{itemize}
\item \term{Незнание} --- отсутствие необходимости специальным образом обозначать или иначе подготавливать участок программы, в который будет внесено изменение с помощью аспектов. Другими словами, наличие аспекта не влияет на структуру программы, в которую он встраивается, то есть программа \term{не знает} о существовании аспекта.
\item \term{Квантификация} --- возможность встроить один и тот же совет в несколько участков программы, описанных некоторым выражением. Это выражение играет роль \term{квантора} (аналогично квантору всеобщности в логике предикатов \cite{???}).
\end{itemize}

Незнание достигается за счет композиционных свойств семантики аспектов: встраивание преобразует одну корректную программу в другую корректную программу, не требуя от исходной программы никаких специальных свойств. Все обязательства по обеспечению совместимости берет на себя аспект.

Квантификация достигается с помощью механизма \term{срезов} (point-cuts) --- специальных выражений, которые описывают множества \term{точек встраивания} (join points). Далее \term{советы} (advice) ассоциируются не с отдельными точками встраивания, а со срезами, что позволяет избежать дублирования советов.

\section{Основные понятия АОП в \tool{Grammatic}}

В следующих подразделах мы покажем как эти общие положения АОП реализуются в \tool{Grammatic}, и как они могут быть использованы для достижения модульности грамматик, отделения семантики языков от синтаксиса и сокращения дублирования кода при автоматической генерации сред разработки.

\subsection{Точки встраивания и срезы}

Точками встраивания в \tool{Grammatic} являются все элементы грамматики: символы, продукции, выражения, аннотации и т.д. Таким образом, с помощью аспектов можно модифицировать любые структуры внутри грамматики. Такое решение требует от языка срезов достаточно большой выразительной силы: срезы должны позволять выбирать из грамматики объекты довольно сложной структуры.

Срезы напоминают язык регулярных выражений над грамматиками; они представляют собой образцы для сопоставления с конструкциями в грамматике. Самой простой формой среза является константный срез, то есть образец, который в точности повторяет фрагмент грамматики, с которым он сопоставляется. Фактически, этот вид среза представляет собой прямую цитату из грамматики, например:
\begin{lstlisting}
a : b | c d* ;
\end{lstlisting}
Этот образец успешно сопоставляется только с правилом грамматики, которое записывается в точности так же. Таким образом, срезы, как и шаблоны, включают те же базовые элементы, что и грамматики (см. \figref{operations}). Срезы дополняют этот набор элементов \term{подстановочными знаками} (wildcards).

Подстановочные знаки аналогичны символу ``.'' в стандартных регулярных выражениях, но, поскольку элементы грамматик являются типизированными, подстановочные знаки позволяют различать типы. Обобщенный подстановочный знак можно записать как ``\code{<?имя : тип>}''. Так, например, выражение, описывающее последовательность любых элементов, заключенных в скобки, будет успешно сопоставлено со следующим образцом:
\begin{lstlisting}
'(' <?s : Sequence> ')'
\end{lstlisting}
Например, если мы сопоставим с этим образцом выражение \code{'(' a* b ')'}, сопоставление пройдет успешно и с именем \code{s} будет связана последовательность \code{a* b}. Таким образом, имя в подстановочном знаке соответствует новой \term{переменной}. Имена переменных внутри подстановочных знаков не должны повторяться внутри одного среза. Если имя в подстановочном знаке не нужно, его можно опустить (оставив знак вопроса). Например, \code{<? : Production>} соответствует произвольной продукции, но при сопоставлении не связывает ее ни с каким именем.

Переменные можно ассоциировать не только с подстановочными знаками, но и с константами и более сложными структурами. Рассмотрим пример:
\begin{lstlisting}
?seqInBrack=('(' <?s : Sequence> ')')
\end{lstlisting}
Этот образец будет успешно сопоставлен с выражением \code{'(' a* b ')'}, и с переменными будут связаны следующие значения:
\begin{itemize}
\item s = \code{a* b};
\item seqInBrack = \code{'(' a* b ')'}.
\end{itemize}

Ранее определенную переменную можно использовать в том же образце, например:
\begin{lstlisting}
<?s : Sequence> (<? : Sequence> ?s)*
\end{lstlisting}
Приведенный пример успешно сопоставляется с выражениями \code{a (',' a)*} и \code{'?' var ('-' '?' var)*}, при этом в первом случае c переменной \code{s} связывается значение \code{a} (один элемент тоже является последовательностью), а во втором --- \code{'?' var}. 

Заметим, что определять переменную (то есть использовать имя перед знаком ``='' или внутри подстановочного знака) можно только однажды внутри одного образца, а использовать ее --- неограниченно много раз. Это ограничение гарантирует, что все объекты, связанные с данной переменной будут структурно идентичны. Для переменных, определенных с помощью знака ``='' также можно указывать типы, например:
\begin{lstlisting}
?(seqInBrack : Sequence)=('(' <?s : Sequence> ?(s : Sequence) ')')
\end{lstlisting}
Тип выражения в правой части должен быть подтипом для типа переменной. 

Для упрощения работы с подстановочными знаками, для некоторых типов введены короткие обозначения, приведенные в \tabref{ShortWildcards}.
\begin{figure}[htbp]
	\centering
	\begin{tabular}{|c|l|}
	\hline  \bf Короткая запись & \bf Значение \\ 
	\hline  
	\code{..}  & \code{<? : Sequence>} \\ 
	\code{...}  & \code{<? : Alternative>} \\ 
	\code{:.:}  & \code{<? : Production>} \\ 
	\code{\#}  & \code{<? : Symbol>} \\ 
	\code{\#lex}  & \code{<? : LexicalDefinition>} \\ 
	\code{\{*\}}  & \code{<? : Attribute*>} \\ 
	\hline 
	\end{tabular} 
	\caption{Короткие обозначения для подстановочных знаков}\label{ShortWildcards}
\end{figure}
С их помощью предыдущий пример можно записать в следующей форме:
\begin{lstlisting}
?s=.. (.. ?s)*
\end{lstlisting}

// Сопоставление метаданных

\subsection{Советы}

Срезы обеспечивают ``адресацию'' в грамматике и позволяют выбрать позиции для встраивания советов, то есть для добавления новых элементов или замены существующих. В общем виде это оформляется следующим образом:
\begin{lstlisting}[escapeinside={!}{!}]
!образец! ;
	before  !переменная! : !шаблонное выражение!;
	after   !переменная! : !шаблонное выражение!;
	instead !переменная! : !шаблонное выражение!;
\end{lstlisting}
Переменные должны быть связаны в образце. Шаблонные выражения могут использовать все переменные, связанные в образце, как шаблонные параметры. Оператор \code{instead} заменяет все вхождения переменной на результат разворачивания шаблонного выражения; операторы \code{before} и \code{after} --- вставляют результат разворачивания шаблонного выражения до или после каждого вхождения переменной, соответственно. Рассмотрим пример:
\begin{lstlisting}
?i=.. (?sep=#lex ?i);
	instead ?sep : <?sep>+;
\end{lstlisting}
Образец, использованный в данном примере, успешно сопоставляется с любым списком, где разделителем является содержимое переменной \code{?sep}. Совет \code{instead} помещает разделитель под оператор \code{+}. Например, если данный аспект применить к следующей грамматике:
\begin{lstlisting}
expr : factor ('*' factor)*;
factor : '(' NUM (',' NUM)* ')';
\end{lstlisting}
оба разделителя (\code{*} и \code{,}) будут заменены соответствующим образом, и в результате получится следующая грамматика:
\begin{lstlisting}
expr : factor ('*'+ factor)*;
factor : '(' NUM (','+ NUM)* ')';
\end{lstlisting}

Аспекты позволяют модифицировать и метаданные. Например, чтобы добавить атрибут к аннотации на данном выражении, можно применить следующий аспект:
\begin{lstlisting}
example : .. ?e=('(' .. ')') .. ;
	instead ?e : <?e> {a = 10} ;
\end{lstlisting}
Поскольку несколько аннотаций на одном и том же объекте объединяются в одну, данный пример добавляет атрибут, не удаляя уже имеющихся. Чтобы удалить имеющиеся атрибуты, необходимо связать их с некоторой переменной и заменить ее значение пустой аннотацией:
\begin{lstlisting}
example : .. ?e=('(' .. ')')?a={*} .. ;
	instead ?a : {};
\end{lstlisting}

// Примеры про before и after

\section{Поддержка метаданных}

советы для присоединения метаданных

\section{Пример: использование аспектов при описании языка StateMachine}

В данном разделе мы проиллюстрируем использование аспектов при декомпозиции грамматических определений. В качестве примера будем использовать язык StateMachine, описанный выше.

\subsection{Специализация грамматики}
добавить лог-строку

разрешить только одно выходное воздействие

\subsection{Семантические действия}

добавить дебаг-аспект

a : b c d ;
	instead a : a{s = 'sdfds'}
	
	он же 
	
	@a.s = 'sdfds'

\subsection{Отделение метаданных}


проблема дублирования грамматик и загрязнения кода аннотациями

(подсветка синтаксиса, форматирование...)

Форматирование, синтаксис, СУТ
	

\chapter{Применение \tool{Grammatic} для описания диалектов языка SQL}

Стандарт \tool{SQL-92} \cite{SQL92} охватывает широкий круг возможностей языка, и большинство реализаций поддерживают его лишь частично. Кроме того, некоторые реализации добавляют собственные синтаксические конструкции в язык. При создании инструментов разработки, не привязанных к конкретной реализации СУБД\footnote{Система управления базами данных.} возникает задача компактного описания нескольких диалектов \tool{SQL} для того, чтобы корректно обрабатывать код, написанный для различных реализаций.

В данном разделе описано применение \GRM{} для описания синтаксиса диалектов \tool{SQL} с помощью аспектов, применяемых к грамматике языка, описанной в стандарте. Данная грамматика содержит 961 правило, из них 514 синтаксических (остальные --- лексические), поэтому ни ее, ни даже описания диалектов полностью мы не приводим, но приводим примеры, дающие представление о способе описания и его эффективности.

С помощью \GRM{} спецификация \tool{SQL} была разбита на 17 модулей. В тексте спецификации были обнаружены следующие повторяющиеся конструкции, которые были заменены шаблонами выражений:
\begin{itemize}
\item перестановочные выражения:
\begin{lstlisting}
	template Commute<a, b> {<?a> <?b>? | <?b> <?a>?}
\end{lstlisting}
\item необязательное вхождение выражения в круглых скобках:
\begin{lstlisting}
	template OptParen<a> {('(' <?a> ')')?}
\end{lstlisting}
\item бинарные операции:
\begin{lstlisting}
	template OptParen<symbol, factor, op1, op2> {
		<?symbol>
			: <?factor>
			: <?symbol> op1 <?symbol>
			: <?symbol> op2 <?symbol> ;
	}
\end{lstlisting}
\item список с разделителем (шаблон приводился выше).
\end{itemize}

СУБД \tool{Derby} \cite{Derby} и \tool{PostgreSQL} \cite{PostgreSQL}

Derby:
\url{http://db.apache.org/derby/docs/10.0/manuals/reference/sqlj151.html}

PostgreSQL SELECT:
\url{http://www.postgresql.org/files/documentation/books/aw_pgsql/node278.html#SECTION0032784100000000000000}

\chapter{Сравнение с существующими инструментами}

Результаты сравнения \GRM{} с другими инструментами представлены в \tabref{GrmTable}. Строки таблицы соответствуют тем или иным возможностям, связанным с механизмами композиции. Для сравнения были выбраны девять инструментов, поддерживающих различные механизмы композиции, а также \tool{Yacc/Bison}, очень широко используемое представляющие семейство инструментов.

\begin{table}[htb]
	\centering
\newcommand{\dissonly}[1]{#1}
\newcommand{\hd}[1]{{\begin{sideways}\parbox{25mm}{\tool{#1}}\end{sideways}}}
%{\begin{sideways}\parbox{15mm}{\bf text}\end{sideways}}
{\small\noindent
\begin{tabular}{|ll|c|c|c|c|c|c|c|c|c|c|c|c|}
\hline 
&
	&\hd{Yacc/Bison}
	&\hd{ANTLR}
	&\hd{Menhir}
	&\hd{xText}
	&\hd{ASF+SDF}
	&\hd{JastAdd}
	&\hd{AspectG}
	&\hd{Silver}
	&\hd{LISA}
	&\hd{Rats!}
	&\hd{\textbf{Grammatic}}\\
\hline
\dissonly{
\multicolumn{2}{|r|}{\it Ссылка}
	&\cite{BisonBook}
	&\cite{ANTLR}
	&\cite{Menhir}
	&\cite{xText}
	&\cite{ASF+SDF}
	&\cite{JastAdd}
	&\cite{AspectG}
	&\cite{Silver}
	&\cite{LISA}
	&\cite{Rats!}
	&\\
\hline 
}
\dissonly{\multicolumn{13}{|l|}{Модули}\\%&&&&&&&&&&&&&\\
&Разделение на файлы&-&+&+&+&+&+&+&+&+&+&+\\
&Атрибуты доступа&-&-&+&-&+&+&-&+&+&+&+\\}
%&&&&&&&&&&&&&\\
\multicolumn{13}{|l|}{Шаблоны грамматик. Параметры:}\\%&&&&&&&&&&&&&\\
&символы&-&-&-&-&+&-&-&-&-&-&+\\
&модули&-&-&-&-&-&-&-&-&-&+&-\\
&произвольные выражения&-&-&-&-&-&-&-&-&-&-&+\\
%&&&&&&&&&&&&&\\
\multicolumn{13}{|l|}{Шаблоны выражений. Параметры:}\\%&&&&&&&&&&&&&\\
&символы&-&-&+&-&-&-&-&-&-&-&+\\
&произвольные выражения&-&-&-&-&-&-&-&-&-&-&+\\
%&&&&&&&&&&&&&\\
\multicolumn{2}{|l|}{Шаблоны сем. действий}&-&-&-&-&-&-&-&-&+&-&+\\
%&&&&&&&&&&&&&\\
\multicolumn{13}{|l|}{Аспекты}\\%&&&&&&&&&&&&&\\
&Незнание&-&-&-&-&-&+&+&+&+&-&+\\
&Квантификация&-&-&-&-&-&-&+&+&+&-&+\\
%&&&&&&&&&&&&&\\
\multicolumn{13}{|l|}{Гибкость при повторном использовании}\\%&&&&&&&&&&&&&\\
&Добавление продукций&-&+&+&+&+&+&+&+&+&+&+\\
&Замена элементов&-&+&-&+&-&-&-&-&-&+&+\\
&Удаление элементов&-&-&-&-&-&-&-&-&-&+&+\\
&Изменение метаданных&-&-&-&+&+&+&+&+&+&+&+\\
\dissonly{
\hline
\multicolumn{2}{|r|}{Cумма}&0&3&4&4&5&5&5&6&7&7&13\\}
\hline
\end{tabular}
}
	\caption{Сравнение \GRM{} с существующими инструментами}\label{GrmTable}
\end{table}

При сравнении механизма модулей учитывались само наличие этого механизма, то есть возможность разделения спецификации на несколько физических файлов, и поддержка атрибутов доступа, регулирующих область видимости определяемых в модуле элементов.

При сравнении механизма шаблонов учитывалось наличие поддержки 
	(а) шаблонов грамматик (параметризованные модули) с различными типами параметров, 
	(б) шаблонов выражений с различными параметрами и
	(в) шаблонов семантических действий. 
В случае \GRM{} последняя категория дополняется наличием поддержки шаблонов произвольных метаданных, а не только семантических действий. Такой возможности, как и вообще метаданных произвольного вида, нет ни в одном из рассмотренных инструментов.

При сравнении механизмов, основанных на аспектах, учитывались два основных фактора: незнание и квантификация. Конкретные возможности, которые в \GRM{} реализуются с помощью аспектов, указаны в таблице под заголовком ``Гибкость при повторном использовании'', поскольку некоторые другие инструменты реализуют эти возможности, не используя аспекты в явном виде (так, например, \tool{Rats!} не обеспечивает незнания и квантификации при удалении элементов грамматики, однако реализует эту функциональность, в отличие от других инструментов.

В нижней строке таблицы приведены количества ``очков'', набранных каждым инструментом. За один знак ``+'' засчитывалось одно ``очко''.

Результаты сравнения показывают, что \GRM{} представляет собой наиболее мощный инструмент (13 пунктов против 7 у ближайших конкурентов), поскольку сочетает в себе полноценную поддержку шаблонов и аспектов, которые в таком объеме не реализованы в других инструментах. Практическая полезность данных механизмов композиции была продемонстрирована выше на примерах создания компонент интегрированных сред разработки и описания диалектов языка \tool{SQL}.

В следующей главе описывается генератор трансляторов \ATF{}, который использует как механизмы композиции, так и метаданные \GRM{}.
%\part{Применение \tool{Grammatic} для описания диалектов языка SQL}\label{part3}

// Какие-то вводные слова

\chapter{Структурированный язык запросов SQL}

назначение

DDL DML QL

диалекты

Зачем надо много диалектов

\chapter{Описание диалектов SQL}

шаблоны для похожих конструкций
аспекты для диалектов

\chapter{Оценка эффективности механизмов композиции \tool{Grammatic}}

критерии

данные
\part{Расширение предметно-ориентированных языков механизмами композиции}\label{part4}

Механизмы композиции на основе шаблонов и аспектов являются \term{ инфраструктурными}: они практически не зависят от предметной области, в которой будет применяться язык. Это позволяет выработать метод реализации таких механизмов: ``ядро'' языка, состоящее из конструкций, выражающих понятия предметной области, может быть расширено этими механизмами, реализация которых добавляется к реализации ядра, практически с ней не взаимодействуя. По сути, данные механизмы работают в режиме препроцессора: сначала выполняется композиция, в результате которой получаются конструкции ядра, в свою очередь обрабатываемые сообразно назначению языка.

В данной главе этот метод описан в виде поэтапной процедуры, все шаги которой могут быть выполнены автоматически. Первый раздел посвящен способу описания ядра языка, основанному на существующих средствах разработки предметно-ориентированных языков, не поддерживающих сложных механизмов композиции. Второй и третий разделы посвящены расширению языков шаблонами и аспектами, соответственно. Далее приведен пример применения описанного метода и сравнение его с существующими подходами.

\chapter{Описание исходного языка}

размыкание неагрегирующих ссылок

преобразование древовидной модели (все классы допускают локальную замену) в грамматику

семантические акции для разбора + анализ имен

мета-модель

грамматика

преобразование

\chapter{Текстовая нотация для моделей}
Ниже нам понадобится записывать формализованные утверждения, вовлекающие мета-модели и их экземпляры. Для этого в данном разделе мы опишем нотацию, позволяющую описывать соответствующие понятия в виде текста, что удобно при записи правил вывода, например, при формализации систем типов.

\section{Элементы мета-моделей}

Для описания элементов мета-моделей в формулах мы будем использовать следующие обозначения:

\begin{itemize}
\item Класс описывается именем $C$ и списками суперклассов $S$, ссылок $R$ и атрибутов $A$: $\class{C}{S}{R}{A}$. Иногда удобно объединять атрибуты и ссылки в единый список структурных элементов~$F$: $\classf{C}{S}{F}$.
\item Атрибут или ссылка описываются именем и типом: $\attribute{x}{\tau}$ и ???, соответственно.
\end{itemize}
%
\newcommand{\type}[2]{#1\left(#2\right)}%
\newcommand{\valts}{\mathbf{val}}%
\newcommand{\valt}[1]{\type{\valts}{#1}}%
\newcommand{\refts}{\mathbf{ref}}%
\newcommand{\reft}[1]{\type{\refts}{#1}}%
%
Типы структурных элементов имеют следующий вид:
\begin{itemize}
\item \code{String}, \code{Integer}, \code{Boolean} или \code{Character} --- имя примитивного типа, используется только для атрибутов;
\item $\reft{C}$ --- неагрегирующая ссылка на класс $C$;
\item $\valt{C}$ --- агрегирующая ссылка на класс $C$;
\item $\tau^?$ --- ссылка может иметь значение $NULL$, атрибут может быть не указан;
\item $\tau^*$ --- коллекция может содержать ноль или более элементов;
\item $\tau^+$ --- коллекция может содержать один или более элементов.
\end{itemize}
%
\section{Элементы моделей}

\newcommand{\obj}[3]{#1@#2\left\{#3\right\}}%
\newcommand{\refv}[2]{\mathbf{ref}\; #1@#2}%
%
Нотация для элементов моделей несколько более сложна, поэтому мы сначала приводим ее компактное описание в виде контекстно-свободной грамматики в \lstref{modelGrammar}, а затем поясняем.

Общий вид обозначения для объекта таков: 
\[\obj{C}{id}{r_i = u_i, a_i = v_i},\]
 где $C$ --- класс данного объекта, $id$ --- его уникальный идентификатор\footnote{Согласно основополагающим принципам ООП, каждый объект обладает \term{идентичностью} (identity, см. \cite{Booch}).}, $r_i = u_i$ --- ссылки и их значения, а $a_i = v_i$ --- атрибуты и их значения. Как и в случае классов, часто бывает удобно объединять атрибуты и ссылки и писать $\obj{C}{id}{f_i = v_i}$.

\begin{lstlisting}[label=modelGrammar,float=htbp,caption={Грамматика, описывающая запись моделей в виде текста}]
object : NAME '@' INT '{' (NAME '=' value ';')* '}' ;
value : attrValue | refValue;
attrValue : INT | STRING | CHAR | 'true' | 'false' 
          | 'NULL' | enumValue;
enumValue : NAME '.' NAME;
refValue : object | ref | list | set | 'NULL';
ref	: 'ref' '(' NAME '@' INT ')';
list : <Collection '[', ']'> ;
set : <Collection '{', '}'> ;

template Collection<start, end> : Production+ {
	: <?start> <?end>
	: <?start>  <List object, ','> <?end>
	: <?end> <List ref, ','> <?end>
}
\end{lstlisting}

Значениями атрибутов могут быть литералы примитивных типов (числа, строки, символы, булевские значения) или перечислений.

Значениями ссылок могут быть либо объекты (в случае агрегирующих ссылок), либо ссылочные выражения вида $\refv{C}{id}$, указывающие класс и идентификатор объекта, на который делается ссылка. Также ссылки могут иметь множественные значения --- коллекции\footnote{В принципе, множественные значения могут иметь и атрибуты, но для дальнейшего обсуждения это не важно, и мы не будем это учитывать.}, которые могут быть упорядоченными (списки) и неупорядоченными (множества).

Шаблонные выражения в правой части правил для списков (\code{list}) и множеств (\code{set}) разворачиваются следующим образом:
\begin{lstlisting}
list
	: '[' ']'
	: '[' object (',' object)* ']'
	: '[' ref (',' ref)* ']'
	;
set
	: '{' '}'
	: '{' object (',' object)* '}'
	: '{' ref (',' ref)* '}'
	;
\end{lstlisting}
Важно отметить, что коллекция может одновременно хранить либо объекты, либо ссылочные выражения, но не те и другие сразу.

На объектах определено отношение структурной эквивалентности, обозначаемое $\cong$ и заданное правилами, приведенными на \figref{cong}.
%
\begin{figure}[htbp]
	\centering
$$
\infer[obj]{
	\obj{C}{id_1}{f_i = v_i} \cong \obj{C}{id_2}{f_i = u_i}
}{
	\forall i. \; v_i \cong u_i
}
$$
$$
\infer[ref]{
	\refv{C}{id_1} \cong \refv{C}{id_2}
}{
	id_1 = id_2
}
\quad
\infer[refl]{x \cong x}{}
$$
$$
\infer[elist]{[\,] \cong [\,]}{}
\quad
\infer[list]{
	[x_1,\ldots,x_n] \cong [y_1,\ldots,y_n]
}{
	\forall i \in [1:n]. \; x_i \cong y_i
}
$$
$$
\infer[eset]{\{\} \cong \{\}}{}
\quad
\infer[set]{
	\{x_1,\ldots,x_n\} \cong \{y_1,\ldots,y_n\}
}{
	\forall i :\; x_i \cong y_{\pi(i)}
}
$$
	\caption{Отношение структурной эквивалентности объектов}\label{cong}
\end{figure}
Два объекта связаны отношением $\cong$, если они являются точными копиями друг друга, то есть отличаются только идентификаторами. В правиле \rref{set} $\pi$ обозначает некоторую перестановку из $n$ элементов, которая необходима для того, чтобы выразить неупорядоченность коллекции.
Очевидно, что $\cong$ является отношением эквивалентности.

\section{Структурные ограничения в виде системы типов}

\newcommand{\fromMM}{\MM{M} \Vdash}
Мета-модель накладывает ограничения на структуру объектов через типы и кратность ссылок и атрибутов. Эти ограничения мы записываем в виде системы типов, которая связывает мета-модель и объект отношением $\Vdash$. Запись $\fromMM x : \tau$ следует читать как ``\term{объект $x$ удовлетворяет ограничениям мета-модели $\MM{M}$ и имеет тип $\tau$}''.

Правила указанной системы типов приведены на \figref{TypesMM}. В этих правилах используется отношение ``подтип'', обозначаемое следующим образом:
$$
	\mbox{подтип} \subtype \mbox{супертип}.
$$
Это отношение является транзитивным рефлексивным замыканием минимального отношения, удовлетворяющего следующему требованию: $\classf{C}{\{S_i\}}{\_} \subtype S_i$.

\begin{figure}[htbp]
	\centering
$$
	\infer[object]{
		\fromMM \obj{C}{id}{f_i = v_i} : \valt{C}
	}{
		\classf{C}{\_}{f_i : \tau_i} \in \MM{M}&
		\fromMM v_i : \tau_i
	}
$$
$$
\infer[ref]{
	\fromMM \refv{C}{id} : C
}{}
\quad
\infer[null]{
	\fromMM NULL : \tau^?
}{}
$$
$$
\infer[elist]{
	\fromMM [\,] : \tau^*
}{}
\quad
\infer[list]{
	\fromMM [x_1, \ldots, x_n] : \type{TO}{\tau}^+
}{
	\forall i\in[1:n]. \; \fromMM x_i : \type{TO}{\tau} &
	TO \in \{\valts,\, \refts\}
}
$$
$$
\infer[eset]{
	\fromMM \{\} : \tau^*
}{}
\quad
\infer[set]{
	\fromMM \{x_1, \ldots, x_n\} : \type{TO}{\tau}^+
}{
	\forall i\in[1:n]. \; \fromMM x_i : \type{TO}{\tau} &
	TO \in \{\valts,\, \refts\}
}
$$
$$
\infer[relax]{
	\fromMM x : \tau^?
}{
	\fromMM x : \tau
}
\quad
\infer[relax^+]{
	\fromMM x : \tau^*
}{
	\fromMM x : \tau^+
}
\quad
\infer[subtype]{
	\fromMM x : \sigma
}{
	\fromMM x : \tau &
	\tau \subtype \sigma
}
$$
$$
\infer[superclass]{
	\forall i \in [1:n].\; \type{TO}{C} \subtype \type{TO}{S_i}
}{
	\classf{C}{\{S_i, \ldots, S_n\}}{\_} \in \MM{M} &
	TO \in \{\valts,\, \refts\}
}
$$
$$
\infer[enum]{
	\forall i \in [1:n]. \; \fromMM T.L_i : T
}{
	\mathbf{enum}\;T\{L_1,\ldots,L_n\} \in \MM{M}
}
\quad
\infer[bool]{
	\fromMM b : \mathtt{Boolean}
}{
	b \in \{\mathbf{true}, \mathbf{false}\}
}
$$
$$
\infer[int]{\fromMM INT : \mathtt{Integer}}{}
\quad
\infer[str]{\fromMM STRING : \mathtt{String}}{}
$$
$$
\infer[char]{\fromMM CHAR : \mathtt{Char}}{}
$$
	\caption{Структурная корректность объектов}\label{TypesMM}
\end{figure}




\chapter{Автоматическое построение языков, поддерживающих типизированные макроопределения}

В настоящем разделе описан метод, позволяющий по описанию языка построить описание более богатого языка, поддерживающего композицию с помощью типизированных макроопределений. Мы будем называть такой пополненный язык \term{макро-языком}, построенным на базе исходного языка.

\begin{Note}[О терминологии]
В англоязычной литературе используется термин \term{macro} \cite{MacroML,Cpp,Nemerle}. 
В качестве русского перевода этого термина употребляется либо слово ``макрос'', являющееся в сущности сленговым и полученное транслитерацией множественного числа ``macros'', либо слово ``макроопределение'', соответствующее более узкому по смыслу термину ``macro definition''. 
Близкой по смыслу альтернативой является термин ``шаблон'' (англ. ``template''), используемый в языке 
\tool{C++} 
\cite{C++} 
и ряде других~\cite{HTMP,Velocity,UML}. 

Мы считаем термин ``macro'' более подходящим для наших целей, и будем, следуя традиции, использовать более формальный вариант его перевода --- \term{макроопределение}.
\end{Note}

\section{Неформальное описание механизма композиции, основанного на макроопределениях}

Наиболее известными системами, использующими макроопределения, являются язык программирования \tool{LISP}~\cite{Lisp} и препроцессор языка \tool{C}~\cite{C, Cpp}. В обоих случаях макроопределения служат для обогащения языка новыми конструкциями, которые преобразуются в базовые конструкции языка во время компиляции\footnote{Понятие ``время компиляции'' в данном контексте является собирательным и противопоставляется понятию ``время выполнения программы''.}, этот процесс называется \term{разворачиванием} макроопределений.

Приведем пример использования макроопределений в языке \tool{C}: пусть нам требуется реализовать односвязные списки. Для этого в первую очередь необходимо описать \emph{структуру} элемента списка, например, для списка целых чисел эта структура будет выглядеть следующим образом:
\begin{lstlisting}[language=C]
struct IntList {
	struct IntList* next;
	int data;
};
\end{lstlisting}
Для списка элементов другого типа, например, строк, структура будет очень похожей, но тип поля \code{data} будет отличаться:
\begin{lstlisting}[language=C]
struct StrList {
	struct StrList* next;
	char* data;
};
\end{lstlisting}
Дублировать описания для каждого нового типа элементов не очень удобно, поэтому мы напишем макроопределение, которое по данному типу элемента генерирует описание соответствующей структуры\footnote{Задача, которую мы решаем в этом примере с помощью макроопределений языка \tool{C}, более эффективно решается в языке \tool{C++} с помощью \emph{шаблонных структур}, которые можно рассматривать как узкоспециализированную разновидность макроопределений.}:
\begin{lstlisting}[language=C]
#define DEFLIST(name, type) struct name { \
	struct name *next; \
	type data; \
};
\end{lstlisting}
После директивы \code{\#define} следует \emph{имя} макроопределения, а за ним в скобках --- \emph{параметры}. Весь остальной текст --- это \emph{тело} макроопределения (знак ``$\backslash$'' используется для подавления перевода строки). Для того, чтобы использовать данное макроопределение, достаточно написать его имя и передать в скобках значения параметров (то есть \emph{аргументы}), например:

\begin{lstlisting}[language=C]
DEFLIST(StrList, char*)
\end{lstlisting}
В результате \emph{разворачивания} данного определения аргументы будут подставлены в тело вместо соответствующих параметров и мы получим определение структуры \code{StrList}, приводившееся выше. Аналогично можно получить определения структуры \code{IntList}, а также структуры элемента списка для любого типа. Заметим, что разворачивание происходит во время компиляции и результатом является исходный текст на языке \tool{C}, который, в свою очередь, транслируется в машинный код, причем транслятор ничего не знает о том, были использованы макроопределения или нет. 

Обобщая сведения, приведенные в данном примере, можно выделить следующие свойства, присущие механизму макроопределений\footnote{Такое обобщение имеет смысл, поскольку в различных языках и системах макроопределения функционируют схожим образом.}:
\begin{itemize}
\item Макроопределение состоит из \emph{списка параметров} и \emph{тела} и имеет уникальное \emph{имя}.
\item Тело макроопределения содержит конструкции на целевом языке (в нашем примере --- на языке \tool{C}) и ссылки на параметры. Также тело может содержать обращения к другим макроопределениям.
\item При разворачивании ссылки на параметры в теле макроопределения заменяются значениями соответствующих аргументов.
\item Разворачивание происходит во время компиляции программы.
\end{itemize}

Макроопределения представляют собой достаточно гибкий механизм повторного использования. В принципе, этот механизм не зависит от целевого языка, в который разворачиваются макроопределения. В частности, в языке \tool{C} поддержка макроопределений обеспечивается препроцессором \tool{Cpp} --- независимым программным средством, обрабатывающим исходный код \term{до} начала работы собственно компилятора языка \tool{C}. Препроцессор \tool{Cpp} может работать с любым текстом, не только с исходным кодом на языке \tool{C}, следовательно он (или аналогичный механизм) может применяться для повторного использования и в других языках, в частности в предметно-ориентированных, делая их более пригодными для использования в промышленных проектах.

Однако чисто текстовый препроцессор обладает одним важным недостатком: корректность результата разворачивания макроопределений никак не гарантируется, поскольку препроцессор манипулирует простым текстом и ``не знает'' о синтаксисе целевого языка.

Вернемся к примеру макроопределения, приведенному выше. Если программист допустит ошибку при использовании макроопределения \code{DEFLIST}, а именно перепутает порядок аргументов, что случается не так уж редко, препроцессор послушно выполнит свою работу:
\begin{lstlisting}[language=C]
DEFLIST(char*, StrList)
\end{lstlisting}
превратится в
\begin{lstlisting}[language=C]
struct char* {  // error: expected '\{' before 'char'
    struct char* *next; 
    StrList data; 
};
\end{lstlisting}
Получившийся код синтаксически некорректен, и компилятор, получив этот текст на вход, выдаст сообщение об ошибке:
\begin{lstlisting}[language=C]
DEFLIST(char*, StrList) // error: expected '\{' before 'char'
\end{lstlisting}
В итоге ошибка программиста обнаружена, но сообщение, выданное компилятором, записано в терминах программы, полученной после разворачивания макроопределений, и совсем не помогает программисту исправить ситуацию. Чтобы разобраться, в чем проблема, придется вручную рассмотреть код, полученный на выходе препроцессора, что является существенным затруднением при разработке больших проектов.
Описанная здесь проблема является основной причиной, по которой профессиональные программисты зачастую избегают широкого использования возможностей макроопределений в программах на \tool{C} \cite{CodeComplete}. 

Итак, чисто текстовый препроцессор позволяет легко обеспечить поддержку макроопределений в любом языке, но не обеспечивает своевременного обнаружения ошибок, что затрудняет разработку. В данной главе мы рассмотрим метод реализации макроопределений, который также пригоден для любого языка, но обеспечивает контроль корректности аргументов макроопределений с помощью специальной системы типов, что позволяет избежать проблем, присущих чисто текстовым препроцессорам. Такие макроопределения мы будем называть \emph{шаблонами} (templates).

\section{Шаблоны в языках, порожденных метамоделями}

Вернемся к примеру описания структуры элементов списка в языке \tool{C}. Для начала рассмотрим абстрактный синтаксис такого описания; соответствующая метамодель приведена на \figref{c-struct-mm}. 
%
\figdef{c-struct-mm}{Метамодель абстрактного синтаксиса описаний структур в языке \tool{C}}
%
Мы не ставим целью рассмотрение возможностей языка \tool{C} во всем их многообразии, поэтому наша метамодель позволяет оперировать лишь весьма ограниченным набором типов: структурами (\code{Struct} и \code{StructType}), указателями (\code{PointerType}) и примитивными типами \code{int} (\code{IntType}) и \code{char} (\code{CharType)}.

На \figref{int-list-struct} приведена диаграмма, соответствующая описанию структуры \code{IntList}, уже приводившемуся выше:
\begin{lstlisting}[language=C]
struct IntList {
	struct IntList* next;
	int data;
};
\end{lstlisting}
%
\figdef{int-list-struct}{Модель, описывающая структуру \code{IntList}}
%
Объект класса \code{Struct} хранит список, состоящий из объектов класса 	\code{Field}, каждый из которых имеет тип и имя.

Теперь преобразуем модель на \figref{int-list-struct} в шаблон, аналогичный макроопределению \code{DEFLIST} из предыдущего раздела:
\begin{lstlisting}[language=C]
#define DEFLIST(name, type) struct name { \
	struct name *next; \
	type data; \
};
\end{lstlisting}
Что для этого нужно сделать? Нужно добавить специальные объекты, представляющие структуру шаблона. На \figref{deflist-template} эти объекты выделены зеленым цветом фона.
%
\figdef{deflist-template}{Шаблон описания структуры элемента списка}
%
Рассмотрим новую диаграмму подробнее. Корневым элементом дерева встраивания является объект \code{DEFLIST} класса \code{Abstraction} --- это и есть шаблон, он содержит список \emph{параметров}, состоящий из двух объектов класса \code{Variable}, и \emph{тело} --- объект \code{struct}. Значение свойства \code{@name} объекта \code{struct} изменилось по сравнению с \figref{int-list-struct}: если раньше это была строка $\String{IntList}$, то теперь это объект класса \code{VariableUsage}, который ссылается на объект \code{name} класса \code{Variable}. Это соответствует использованию параметра теле шаблона. Аналогично изменилось из значение свойства \code{@type} объекта \code{data}: теперь это тоже объект класса \code{VariableUsage}, ссылающийся на параметр \code{type}.

Процедура разворачивания просто заменяет объекты \code{VariableUsage} в теле шаблона значениями соответствующих аргументов и получается модель, не содержащая ``шаблонных''объектов (зеленого цвета). Чтобы придать параметрам значения, требуется \code{применить} шаблон к соответствующим аргументам. 
%
\figdef{deflist-application-intlinst}{Применение шаблона}
%
На \figref{deflist-application-intlinst} приведена диаграмма, соответствующая применению шаблона \code{DEFLIST}, определенного выше, к аргументам $\String{IntList}$ и $\Object{}{\Ref{\String{IntType}}}{}$. Объект класса \code{Application} (\emph{применение}) содержит ссылку на применяемый шаблон и список аргументов. Соответствие между параметрами шаблона и аргументами устанавливается с помощью индексов в списках: нулевой аргумент соответствует нулевому параметру, первый --- первому и т.д. На рисунке параметры шаблона соединены с соответствующими аргументами пунктирными линиями. 

Заметим, что результатом разворачивания применения шаблона на \figref{deflist-application-intlinst} будет в точности модель на \figref{int-list-struct}, аналогично тому как разворачивание применения макроопределения
\begin{lstlisting}[language=C]
DEFLIST(IntList, int)
\end{lstlisting}
дает описание структуры \code{IntList}.

Заметим также, что аргументами шаблона в принципе могут быть и ``шаблонные'' объекты. 
%
\figdef{deflist-application-list-of-lists}{Использование применения шаблона в аргументах}
%
Так на \figref{deflist-application-list-of-lists} показано применения шаблона \code{DEFLIST} к списку аргументов, один из которых, в свою очередь, тоже является применением шаблона \code{DEFLIST}. В результате разворачивания шаблонов в этом примере получится описание структуры $\String{ListList}$ элементов списка, состоящего из списков целых чисел.

Чтобы облегчить понимание, мы позволили себе некоторую вольность, приводя модельные термы с шаблонами на диаграммах данном разделе. Рассмотрим, например, \figref{deflist-template}: объекты \code{struct} и \code{data}, изображенные на этом рисунке, не удовлетворяют требованиям метамодели, приведенной на \figref{c-struct-mm}, поскольку эта метамодель определяет свойство \code{name} класса \code{Struct} как строковое, а на нашей диаграмме оно хранит объект класса \code{VariableUsage}; аналогично для свойства \code{type} объекта \code{data}. 

Такое положение вещей неудивительно: язык, порожденный метамоделью на \figref{c-struct-mm}, не поддерживает шаблоны, а для того, чтобы добавить в язык поддержку нового механизма, нужно как минимум пополнить его новыми конструкциями. В последующих разделах мы опишем метод, позволяющую пополнять языки автоматически. Процесс разработки языка с поддержкой шаблонов на основе уже существующего языка схематически представлен на \figref{macro-workflows}.
%
\figdef{macro-workflows}{Разработка и использование языков с поддержкой шаблонов}
%
Метамодель и интерпретирующая семантика языка \emph{Language} разрабатывается архитектором, после чего к метамодели применяется автоматизированная процедура трансформации, которую мы опишем ниже, в результате чего получается новая метамодель, поддерживающая шаблоны. Пользователь создает программу, удовлетворяющую требованиям пополненной метамодели, в своей программе он может использовать шаблоны. К этой программе применяется процедура разворачивания шаблонов, включающая в себя проверку типов аргументов (эту процедуру мы также опишем ниже). В результате получается программа, удовлетворяющая исходной метамодели языка \emph{Language}, к которой применима интерпретирующая семантика, разработанная архитектором. Таким образом, интерпретирующая семантика пополненного языка \emph{T-Language} получается автоматически как композиция процедуры разворачивания и интерпретирующей семантики исходного языка.

\section{Базовый язык шаблонов}

Примеры, приведенные в предыдущем разделе, дают общее представление о том, как можно пополнить существующий язык (в данном случае --- подмножество языка \tool{C}, позволяющее описывать структуры) механизмом шаблонов. В данном разделе мы приступаем к формальному описанию этого метода.

Структура ``шаблонных'' объектов в примерах из предыдущего раздела соответствует упрощенной метамодели, приведенной на \figref{template-metamodel}.
%
\figdef{template-metamodel}{Упрощенная метамодель базового языка шаблонов}
%
Определению шаблона соответствует класс \code{Abstraction}, объекты которого содержат \emph{тело} (\code{body}) и список \emph{параметров} (\code{parameters}). Телом шаблона может являться объект класса \code{Term}, мы будем называть такие объекты \emph{шаблонными термами}. На \figref{template-metamodel} изображено два конкретных подкласса абстрактного класса \code{Term}: \code{Application} (применение шаблона) и \code{VariableUsage} (использование переменной), структуру которых мы обсуждали в предыдущем разделе. 

Заметим, что \figref{deflist-template} не удовлетворяет требованиям данной метамодели: в качестве тела шаблона указан объект класса \code{Struct}, не являющегося подклассом класса \code{Term}. Эта проблема сродни той, которую мы обсуждали в конце предыдущего раздела: базовый язык шаблонов, порождаемый метамоделью, изображенной на \figref{template-metamodel}, не поддерживает конструкций, специфичных для описания структур. Поэтому при построении шаблонного языка необходимо не просто добавить классы для шаблонов в исходную метамодель, но и изменить структуру метамодели так, чтобы конструкции исходного языка стали шаблонными термами. 

На \figref{template-metamodel} изображен подкласс класса \code{Term}, вместо идентификатора имеющий многоточие. Он символизирует шаблонные конструкции, полученные из структур исходного языка. На \figref{c-struct-template-mm} изображены классы метамодели, полученной в результате пополнения языка описания структур шаблонными конструкциями.
%
\figdef{c-struct-template-mm}{Метамодель пополненного языка описания структур}
%
Из рисунка видно, что каждый класс исходной метамодели (\figref{c-struct-mm}) теперь является подклассом класса \code{Term}, и все свойства в также имеют значения типа $\RefT{\String{Term}}$ или $\ClassT{\String{Term}}$. Это позволяет в качестве значения любого свойства, включая свойства классов на \figref{template-metamodel}, использовать как конструкции исходного языка (теперь являющиеся шаблонными термами), так и ``чисто-шаблонные'' термы, то есть применения шаблонов и ссылки на параметры.

%%%%%%%%%%%%%%%%%%%%%%%%%%%%%%%%%%%%%%%%%%%%%%%%%%%%%%%%%%%%%%%%%%%%%%%%%%%%%%%%
%%%%%%%%%%%%%%%%%%%%%%%%%%%%%%%%%%%%%%%%%%%%%%%%%%%%%%%%%%%%%%%%%%%%%%%%%%%%%%%%
%%%%%%%%%%%%%%%%%%%%%%%%%%%%%%%%%%%%%%%%%%%%%%%%%%%%%%%%%%%%%%%%%%%%%%%%%%%%%%%%
%%%%%%%%%%%%%%%%%%%%%%%%%%%%%%%%%%%%%%%%%%%%%%%%%%%%%%%%%%%%%%%%%%%%%%%%%%%%%%%%

\ignore{
\section{Базовый язык макроопределений}
Все языки шаблонов имеют довольно большую общую часть, называемую \term{базовым языком шаблонов}. Мы детально опишем эту часть в данном разделе. Детали, специфичные для конкретных языков шаблонов, будут описаны в следующих разделах.

\figdef{macro-workflows}{Создание и использование языка с поддержкой макроопределений}

Целевая мета-модель базового языка шаблонов (\figref{TempMM}) описывает основные понятия, на которых этот язык строится. Поскольку эта же мета-модель будет использована для описания аспектов, мы не используем в ней слова ``шаблон'' (template), ``параметр'' (parameter) и т.д., но приводим следующее ``отображение'' терминологии мета-модели на терминологию языка шаблонов:
\begin{itemize}
\item Шаблон представляется классом \code{Abstraction} и характеризуется именем, параметрами и телом.
\item Параметру соответствует класс \code{Variable}.
\item Шаблонные выражения представляются абстрактным классом \code{Term}, имеющим в базовом языке только два конкретных подкласса:
	\begin{itemize}
	\item использование шаблонного параметра (\code{VariableRef});
	\item применение шаблона (\code{Application}), описываемое ссылкой на шаблон и аргументами, подставляемыми вместо формальных параметров.
	\end{itemize}
\end{itemize}
Данная мета-модель также использует типы, указывать которые не обязательно. Тип характеризуется уникальным именем и кратностью, соответствующей одиночному вхождению или одной из операций ``\code{?}'', ``\code{*}'' или ``\code{+}''.

\begin{figure}[htbp]
	\centering
\begin{tabular}{p{.45\textwidth}p{.45\textwidth}}
\begin{lstlisting}[xleftmargin=0cm]
class Type {
  attr typeName : String;
  attr multiplicity 
  			: Multiplicity;
}

enum Multiplicity {
  VAL, MANY_VAL, REF
}

class Abstraction {
  attr name : String;
  val parameters : Variable*;
  val body : Term;
  val type : Type?;
}

\end{lstlisting}
&
\begin{lstlisting}[xleftmargin=0cm]
abstract class Term {}

class VariableRef extends Term {
  ref variable : Variable;
}

class Application extends Term {
  ref abstraction : Abstraction;
  val arguments : Term*;
}

class Variable {
  attr name : String;
  val type : Type?;
}
\end{lstlisting}
\end{tabular}
	\caption{Базовая мета-модель языка шаблонов}\label{TempMM}
\end{figure}

Контекстно-свободная грамматика базового языка шаблонов приведена в \lstref{TempG} (в нотации \GRM{}). Эта грамматика сама является шаблоном, поскольку базовый язык необходимо расширять для того, чтобы построить язык шаблонов на основе некоторого конкретного языка. В данном описании имеется два параметра, позволяющих добавлять новые виды выражений (\code{domainSpecificTerms}) и новые типы (\code{domainSpecificTypes}). Поскольку \GRM{} использует данный базовый язык шаблонов, мы не приводим здесь примеры использования его синтаксиса.

\begin{lstlisting}[label=TempG,float=htbp,caption=Базовый синтаксис языка шаблонов]
template Templates<domainSpecificTypes : Expression*, 
		domainSpecificTerms : Expression*> : Grammar {
	abstraction 
		: 'template' NAME <List variable, ','> type? 
			'{' term '}';
	variable : NAME type?;
	type : ':' typeName ('?' | '*' | '+')?;
	typeName
		: basicType
		: <?domainSpecificTypes> ;
	basicType : 'Integer' | 'String' | 'Boolean' | 'Character' ;
	term
		: genericTerm
		: <?domainSpecificTerms> ;
	genericTerm
		: templateVariableRef
		: '<' NAME term* '>' ;
}
\end{lstlisting}

Семантика языка шаблонов задается операцией \term{разворачивания}, описанной в композиционном стиле правилами на \figref{TempSem}. Мы обозначаем результат разворачивания шаблонов в выражении $e$ как $\Inst{\gamma}{e}$, где $\gamma$ (``\term{среда}'') является множеством значений параметров шаблонов вида $p = e$, где $p$ --- параметр, а $e$ --- шаблонное выражение. Как видно из правила \rref{app-inst}, когда разворачивается применение шаблона, среда пополняется информацией о текущих значениях параметров, при этом вызов происходит по значению, то есть аргументы разворачиваются до обработки тела вызываемого шаблона. 

\begin{figure}[htbp]
	\centering
$$
\trule{
	\{p = e\} \subseteq \gamma
}{
	\Inst{\gamma}{\ang{?p}} = e
}{var-inst}
$$ 
$$
\trule{
	\mathbf{template}\left(
		T \, \ang{p_1, \ldots, p_n} \, \{ b \}
	\right)
}{
	\gamma' = \bigcup\limits_{i=1}^{n} \{ p_i = \Inst{\gamma}{a_i} \}
	\qquad
	\Inst{\gamma}{\ang{T \, a_1, \ldots, a_n}} = \Inst{\gamma \cup \gamma'}{b}
}{app-inst}
$$
	\caption{Базовая семантика языка шаблонов}\label{TempSem}
\end{figure}


% встраивание/ссылки???, от этого зависит наличие val/ref в типах ниже

Для того, чтобы гарантировать, что результат применения шаблона будет корректным с точки зрения целевой мета-модели, мы определяем систему типов для языка шаблонов. Базовые правила этой системы типов приведены на \figref{TempTypes}, они должны быть дополнены специфичными правилами для поддержки конкретного языка. Отношение $\subtype$ является специфичным для расширяемого языка и не определяется в базовой системе типов, а лишь используется.

\begin{figure}[htbp]
	\centering
$$
\trule{}{\Gamma \cup \{v : \tau\} \vdash \ang{?v} : \tau}{var}
$$ 
$$
\trule{
	\Gamma \cup \bigcup\limits_{i=1}^{n} \{ p_i : \tau_i \} \vdash b : \sigma
}{
	\Gamma \vdash \mathbf{template}\left(
		T \ang{p_1 : \tau_1, \ldots, p_n : \tau_n} \,: \sigma \, \{ b \}
	\right)
}{abstr}
$$
$$
\trule{
	\Gamma \vdash \mathbf{template}\left(
		T \ang{p_1 : \tau_1, \ldots, p_n : \tau_n} \,: \sigma \, \{ b \}
	\right)
	&
	\forall i : [1:n].\; \Gamma \vdash a_i : \tau_i \
}{
	\Gamma \vdash \ang{T \, a_1, \ldots, a_n} : \sigma
}{appl}
$$
$$
\myinfer[null]{
	\Gamma \vdash NULL : \tau^?
}{}
\quad
\myinfer[relax]{
	\Gamma \vdash x : \tau^?
}{
	\Gamma \vdash x : \tau
}
\quad
\myinfer[relax$^+$]{
	\Gamma \vdash x : \tau^*
}{
	\Gamma \vdash x : \tau^+
}
$$
$$
\myinfer[subtype]{
	\Gamma \vdash x : \sigma
}{
	\Gamma \vdash x : \tau &
	\tau \subtype \sigma
}
$$
$$
\myinfer[elist]{\Gamma \vdash [] : \tau*}{}
\quad
\myinfer[list]{
	\Gamma \vdash [t_1,\ldots,t_n] : \tau^+
}{
	\forall i \in [1:n].\; \Gamma \vdash Item(t_i, \tau)
}
$$
$$
\myinfer[eset]{\Gamma \vdash \{\} : \tau*}{}
\quad
\myinfer[set]{
	\Gamma \vdash \{t_1,\ldots,t_n\} : \tau^+
}{
	\forall i \in [1:n].\; \Gamma \vdash Item(t_i, \tau)
}
$$
$$
\myinfer[item]{
	\Gamma \vdash Item(t, \tau)
}{
	\Gamma \vdash t : \tau
}
\quad
\myinfer[item*]{
	\Gamma \vdash Item(\ang{?v}, \tau)
}{
	\Gamma \vdash v : \tau^*
}
\quad
\myinfer[item$^+$]{
	\Gamma \vdash Item(\ang{?v}, \tau)
}{
	\Gamma \vdash v : \tau^+
}
$$
	\caption{Базовая система типов языка шаблонов}\label{TempTypes}
\end{figure}

В правилах \rref{list} и \rref{set} использовано отношение $Item(x,\tau)$, означающее, что $x$ может быть элементом коллекции $\tau$. Определение этого отношения дано на том же рисунке и сводится к тому, что внутри коллекции типа $\tau$ можно использовать не только одиночные значения этого типа, но и переменные, сами являющиеся коллекциями, что соответствует возможности вставки сразу нескольких элементов.

% mistake: we can add all <?v:\tau*> and get a collection of type \tau+

Дополнительно к правилам типизации, на язык шаблонов накладывается следующее требование: 
%\begin{enumerate}[(a)]
%\item 
\emph{Никакой шаблон не должен быть достижим из своего тела по ссылкам}. Это требование гарантирует отсутствие рекурсии в определениях шаблонов.
%\item \emph{Если множество использований некоторого параметра непусто, в нем должно содержаться хотя бы одно использование, на которое указывает агрегирующая ссылка.} Это требование гарантирует, что в результате разворачивания не возникнет объектов, не агрегируемых результирующей моделью.
%\end{enumerate}

\section{Генерация языка шаблонов}

\newcommand{\TM}{\mathcal{TM}}
\newcommand{\TC}[1]{\mathcal{TC}\left(#1\right)}
\newcommand{\TR}[1]{\mathcal{TR}\left(#1\right)}
\newcommand{\TA}[1]{\mathcal{TA}\left(#1\right)}
%\renewcommand{\vec}[1]{\overrightarrow{\mbox{#1}}}

Для того, чтобы построить язык шаблонов на основе заданного языка $L$, необходимо расширить базовую мета-модель, алгоритм разворачивания и систему типов конструкциями и правилами, специфичными для этого языка. 

Пусть целевая мета-модель языка $L$ обозначается $\MM{M}$, тогда мета-модель соответствующего языка шаблонов, $\TM$, состоит из классов, полученных применением преобразования $\TC{\bullet}$, описанного на \figref{TC}, к каждому классу мета-модели $\MM{M}$. Эта мета-модель использует как классы базовой мета-модели шаблонов (\code{Term}), так и примитивные типы и перечисления мета-модели $\MM{M}$.

\begin{figure}[htbp]
	\centering
$\TC{\class{}{S}{R}{A}} = \class{}
			{\mathtt{Term}}{\TR{R}}{\TA{A}}$
			
$\TC{C^*} = \TC{C}^*$

$\TC{C^+} = \TC{C}^+$

$\TC{C^?} = \TC{C}^?$

$\TR{\reference{ref}{r}{T}} = \reference{ref}{r}{\mathtt{Term}}$

$\TR{\reference{var}{r}{T}} = \reference{val}{r}{\mathtt{Term}}$

$\TA{\attribute{a}{T}} = \attribute{a}{T}$
	\caption{Преобразование конструкций языка в шаблонные выражения}\label{TC}
\end{figure}

Преобразование $\TC{\bullet}$ сопоставляет каждому классу соответствующий тип шаблонного выражения. При этом ссылки перенаправляются на класс \code{Term}, поскольку вместо конкретного объекта может выступать шаблонное выражение.

Аналогичное преобразование конкретного синтаксиса требует пополнения грамматики из \lstref{TempG} конструкциями языка $L$. Кроме того, сами эти конструкции должны допускать использование шаблонных выражений. Таким образом сначала грамматика языка $L$ преобразуется следующим образом: к каждому нетерминалу $N$, соответствующему классу в целевой мета-модели, добавляется продукция $N \rightarrow \mathtt{term}$, где \code{term} --- это нетермниал для шаблонных выражений из грамматики, приведенной в \lstref{TempG}. Это преобразование можно выразить следующим аспектным правилом в языке \GRM{}:
\begin{lstlisting}
#{class} : <p : Production+>;
	instead p : (p (: term));
\end{lstlisting}
Теперь необходимо придать значения параметрам грамматики базового языка шаблонов. Параметр \code{domainSpecificTypes} замещается множеством литералов, содержащих имена классов и примитивных типов из целевой мета-модели $L$, объединенных с помощью операции альтернативы, например
\begin{lstlisting}[language=Grammatic]
	'Sequence' | 'Alternative' | 'Literal'
\end{lstlisting}
Параметр \code{domainSpecificTerms} замещается множеством всех нетерминалов преобразованной грамматики языка $L$.

Построенная таким образом грамматика может оказаться неоднозначной. К сожалению, задача обнаружения неоднозначности является алгоритмически неразрешимой \cite{???}, и автоматизированные средства могут применять лишь эвристические методы для ее решения. В настоящий момент обнаружение и устранение неоднозначностей возлагается на разработчика.

Для обеспечения функционирования специфичных конструкций процедура разворачивания шаблонов пополняется правилами следующего вида:
$$
\trule{
%	C = \class{}{\_}{R=\{r_i\}}{A=\{a_i\}} &
	t = \obj{\TC{C}}{id}{r_i = rv_i,\, a_j = av_j}
}{
	\Inst{\gamma}{t} = \obj{\TC{C}}{id'}{r_i = \Inst{\gamma}{rv_i}, \,a_j = av_j }
}{ds-inst(С)}
$$ 
Задача таких правил --- развернуть шаблонные выражения, находящиеся внутри специфичных конструкций, поэтому все эти правила однотипны и просто осуществляют рекурсивные вызовы на значениях всех ссылок, выходящих из данного объекта. Поскольку в графе объектов могут быть циклы, в процессе разворачивания поддерживается служебное множество уже обработанных объектов, что соответствует стандартному алгоритму обхода графа в глубину \cite{Cormen}. %Заметим, что данное правило подразумевает, что $\Inst{\gamma}{e}$ всегда возвращает один и тот же объект на идентичных входных данных, что при реализации достигается за счет \term{мемоизации} (memoization, \cite{Memoize}).

\newcommand{\ct}[1]{\MM{M}\left(#1\right)}
Согласно приведенному выше правилу, результатом разворачивания шаблонного выражения является другое шаблонное выражение, полученное разворачиванием всех ссылок на параметры и применений шаблонов. Будем называть такие выражения имеющими \term{нормальную форму}. Для того, чтобы получить из выражения $e$ в нормальной форме экземпляр мета-модели $\MM{M}$, необходимо его преобразовать. Такое преобразование (обозначаемое $\ct{e}$) весьма просто: для каждого объекта класса $\TC{C}$ создается объект класса $C$ с той же структурой, то есть ссылки трансформируются рекурсивно, а атрибуты копируются. Ниже мы будем подразумевать выполнение данного преобразования после разворачивания шаблонов.

Система типов также пополняется правилами для специфичных конструкций. Эти правила имеют следующий вид:
$$
\myinfer[\mbox{ds-type(C)}]{
	\Gamma \vdash x : C
}{
	\begin{array}{l}
	x = \obj{\TC{C}}{id}{r_i = v_i; a_j = {v'}_j}\\
	\TM \Vdash x : \TC{C} 
	\end{array}	
	&
	\begin{array}{l}
	r_i = \TR{\rho_i : \tau_i}\\
	\Gamma \vdash v_i : \tau_i\\
	\end{array}	
}
$$ 
Однотипность этих правил также объясняется их рекурсивной природой: они нужны только для того, чтобы проверить типы в шаблонных выражениях внутри данного объекта, если он сам удовлетворяет требованиям мета-модели $\TM$.

Заметим, что шаблонное выражение, являющееся объектом класса $\TC{C}$ типизируется самим классом $C$. Это необходимо для соблюдения требований к наследованию. Поскольку в мета-модели $\TM$ все классы наследуются от класса \code{Term}, отношения наследования в ней не соответствуют таким же отношениям в $\MM{M}$. Поэтому в правиле \rref{subtype} на \figref{TempTypes} отношение $\subtype$ задается мета-моделью $\MM{M}$, а не $\TM$, и типы тоже берутся из $\MM{M}$. 

Чтобы показать, что ограничения, накладываемые системой типов адекватны требованиям мета-модели, докажем следующую лемму.
\begin{Lemm}[О нормальных формах]\label{LemmNF}
Если шаблонное выражение $e$ имеет нормальную форму и $\vdash e : \tau$, то $\fromMM \ct{e} : \tau$.
\end{Lemm}
\begin{proof}
Достаточно заметить, что в дереве вывода для $\vdash e : \tau$ правила \rref{var}, \rref{abstr}, \rref{appl}, \rref{item*} и \rref{item$^+$} не встречаются, а остальные правила в системе типов для шаблонов имеют прямые аналоги в системе требований мета-модели на \figref{TypesMM}.
\end{proof}

\section{Структурная корректность}

Система типов накладывает ограничения на шаблоны. Шаблоны и шаблонные выражения, удовлетворяющие правилам типизации, при разворачивании порождают конечные объекты, удовлетворяющие требованиям целевой мета-модели.
Для того, чтобы убедиться в этом, покажем, что для системы типов и семанитки языка шаблонов выполняются свойства \term{сохранения типов} и \term{нормализации}~\cite{Pierce}. 

Как отмечалось выше, необходимым условием корректного поведения является тот факт, что функция разворачивания получает на вход выражение, в котором соблюдаются правила системы типов. Это условие формализовано в следующем определении.

\begin{Def}\label{agree}
Среда $\gamma$ называется \term{согласованной с контекстом} $\Gamma$, если все ее элементы имеют допустимые типы:
$$
	\forall p \, : \, 
		\{p = e\} \subseteq \gamma 
			\Rightarrow 
		\left\{\begin{array}{l}		
		\{p : \tau\} \subseteq \Gamma \\
		\Gamma \vdash e : \tau
		\end{array}\right.
$$
\end{Def}

Теперь докажем первое из упомянутых выше свойств.

\begin{Th}[О сохранении типов]\label{ThTP}
Если среда $\gamma$ согласована с контекстом $\Gamma$ и \mbox{$\Gamma \vdash e : \tau$}, то \mbox{$\Gamma \vdash \Inst{\gamma}{e} : \tau$}. Другими словами, преобразование $\Inst{}{\bullet}$ сохраняет типы.
\end{Th}
\begin{proof}
Будем вести индукцию по определению $\Inst{\gamma}{e}$.

\noindent\textbf{База.} Правило \rref{ds-inst(C)} сохраняет типы, поскольку из объекта класса $\TC{C}$ получается объект того же класса, а правило \rref{ds-type(C)} выводит тип из класса объекта.

\noindent\textbf{Переход.} 
В правиле \rref{app-inst} среда расширяется, и нам необходимо показать, что результат согласован с контекстом. Это обеспечивается требованиями к типам параметров, накладываемыми правилом \rref{appl} и способом формирования контекста в правиле \rref{abstr}. Из этого по предположению индукции следует, что правило \rref{app-inst} сохраняет типы.

Так же по предположению правило \rref{var-inst} сохраняет типы.
\end{proof}

\newcommand{\h}[1]{h\left(#1\right)}
Свойство нормализации формулируется следующим образом:

\begin{Th}[О нормализации]\label{ThNorm}
Если среда $\gamma = \cup \{p_i = e_i\}$ согласована с контекстом $\Gamma$, все $e_i$ имеют нормальную форму и $\Gamma \vdash e : \tau$, то результат вычисления $\Inst{\gamma}{e}$ имеет нормальную форму.
\end{Th}
\begin{proof} 
Будем вести индукцию по определению $\Inst{\gamma}[\bullet]$.\\
\textbf{База.} Результат применения правила \rref{var-inst} имеет нормальную форму, поскольку все элементы среды имеют нормальную форму.
\textbf{Переход.} В правиле \rref{app-inst} среда пополняется значениями, имеющими нормальную форму, поэтому по предположению индукции вызовы $\Inst{\gamma}{a_i}$ и $\Inst{\gamma \cup \gamma'}{b}$ заканчиваются за конечное число шагов и результат имеет нормальную форму.

Правило \rref{ds-inst(C)} заменяет значения ссылок результатами разворачивания, имеющими нормальную форму, следовательно и результат применения правила имеет нормальную форму.
\end{proof}

Требование о том, чтобы все элементы среды имели нормальную форму выполняется для пустой среды, следовательно теорема применима для разворачивания шаблонных выражений, применяемых на практике.

Мы показали, что функция $\Inst{}{\bullet}$ корректно разворачивает все шаблонные параметры и применения шаблонов, а также что она не нарушает структурной корректности с точки зрения мета-модели. Результатом применения этой функции всегда является константное шаблонное выражение, которое, как отмечалось выше, тривиальным образом преобразуется в экземпляр мета-модели $\MM{M}$. Таким образом, описанный здесь механизм шаблонов работает корректно.

\section{Вывод типов}

Как отмечалось выше, в большинстве случаев типы в объявлениях шаблонов можно опускать, поскольку они могут быть реконструированы автоматически. Для этого применяется алгоритм, аналогичный использованному в \ATF{} (см. Главу \ref{part3} и \cite{Pierce}). Тип переменной выводится исходя из двух соображений: (а) какие аргументы ей присваиваются в применениях шаблона и (б) в каком контексте она используется в теле шаблона, то есть, если на использование переменной указывает ссылка $\TR{R}$, то учитывается тип ссылки $R$. Если система ограничений, построенная таким образом, не имеет решения, генерируется сообщение об ошибке типизации. Такой подход позволяет для переменной найти наиболее широкий тип объектов, которые могут быть подставлены на ее место.

Возникающие в процессе реконструкции неоднозначности разрешаются следующим образом: если параметр используется непосредственно внутри коллекции типа $T$, он получает тип $T^*$ и позволяет добавить в эту коллекцию ноль или более элементов. Исключение составляет случай, когда параметр является единственным элементом коллекции, в которой мета-модель требует наличия хотя бы одного элемента: в этом случае параметр получает тип $T^+$. 

}

\chapter{Автоматическое построение языков, поддерживающих аспекты}

\section{Базовый язык аспектов}
общая мета-модель аспектов

\begin{lstlisting}
class Aspect {
	rules : AspectRule*;
}
class AspectRule {
	pointcut : Application;
	advice : Application;
	subrules : AspectRule*; // scoped by variables?
}
class Wildcard extends Term {
	type : Type;
}
\end{lstlisting}

общий синтаксис
 -- добавить wildcards (во что?):
\begin{lstlisting}
term : '<' '?' n=NAME type? '>'
	instead n : n?;
	after type? : ('=' term)?;
\end{lstlisting}

 -- добавить bindings !!! 
 -- чем отличается variable reference от wildcard
\begin{lstlisting}
aspect : aspectRule*;
aspectRule : 'on' aspectTerm 'perform' substRule* ;
aspectTerm : 
substRule : 'instead' NAME ':' term ;
\end{lstlisting}

специализированная мета-модель образцов (получение из шаблонов)

   Ничего не надо

\section{Алгоритм сопоставления с образцом}

\subsection{Мульти-среды}

Понятие среды, введенное при формализации шаблонов, необходимо расширить для случая аспектов, поскольку в этом случае одной переменной может быть сопоставлено несколько значений, которые являются \term{структурно идентичными} (обозначается $\cong$), то есть либо совпадают, либо являются точными копиями друг друга. Такие структуры будем называть \term{мульти-средами} и обозначать $\ME$.

Основная операция на мульти-средах --- извлечение элемента, она возвращает (возможно, пустое) множество элементов, сопоставленных данной переменной: $\ME(var) = \{e_1,\ldots,e_n\}$.
Для построения мульти-сред будем использовать два конструктора и две операции композиции. Простейшая мульти-среда --- пустая --- обозначается конструктором $\meempty$, при этом $$\forall v :\; \meempty(v) = \emptyset.$$ 
Мульти-среда, сопоставляющая переменной $v$ одно значение $e$ обозначается конструктором $\meitem{v}{e}$, при этом 
$$
\left\{\begin{array}{ll}
\meitem{v}{e}(v) = \{e\}&\\
\meitem{v}{e}(x) = \emptyset, & \mbox{при } x \neq v
\end{array}\right..
$$
Более сложные мульти-среды строятся с помощью операций \term{объединения}  $\mejoin$ и \term{замены} $\mereplace$, заданных следующими правилами: 
$$\left(\ME_1 \mejoin \ME_2\right)(v) = \ME_1(v) \cup \ME_2(v)$$
$$(\ME_1 \mereplace \ME_2)(v) = \left\{\begin{array}{ll}
\ME_2(v), & \mbox{если } \ME_2(v) \neq \bot\\
\ME_1(v), & \mbox{если } \ME_2(v) = \bot\\
\end{array}\right.$$

Специальный элемент $\bot$ не является мульти-средой, но операции композиции доопределяются на нем следующим образом: 
$$\begin{array}{rclcl}
\ME &\mejoin& \bot &=& \bot\\
\bot &\mejoin& \ME &=& \bot\\
\ME &\mereplace& \bot &=& \bot\\
\bot &\mereplace& \ME &=& \bot\\
\end{array}
$$

\term{Уплотнением} мульти-среды $\ME$ называется среда $\meflatten{\ME}$ такая, что $$
\forall v, \, e \in \ME(v): \; \meflatten{\ME}(v) \cong e.
$$
Уплотнение соответствует ``склеиванию'' нескольких значений для каждой переменной в одно, в результате из мульти-среды получается обыкновенная среда.

\subsection{Операция сопоставления с образцом}

\term{Операция сопоставления} объекта $x$ с образцом $P$ в мульти-среде $\ME$ обозначается
$\match[\ME]{e}{P}$ и возвращает мульти-среду $\ME'$ в случае успеха и $\bot$ в случае неудачи. Семантика этой операции представлена на \figref{MatchSem}.

\begin{figure}[htbp]
	\centering

$$
\myinfer[match-var-e]{
	\match[\ME]{x}{\ang{?v}} = \ME \mejoin \meitem{v}{[x]}
}{
	\meflatten{\ME}(v) = [e]
	&
	e \cong x
}
$$	
$$
\myinfer[match-var-P]{
	\match[\ME]{x}{\ang{?v}} = (\match[\ME]{x}{P}) \mereplace \meitem{v}{[x]}
}{
	\meflatten{\ME}(v) = \ang{P}
}
$$	
$$
\myinfer[match-app]{
	\match[\ME]{x}{\ang{T\, a_1,\ldots,a_n}} 
		= \match[
			\ME'
		]{x}{B}
}{
	\mathbf{template}\left(
		T \, \ang{p_1, \ldots, p_n} \, \{ B \}
	\right)
	&
	\ME' = \left(\MEjoin\limits_{i} \meitem{p_i}{\ang{a_i}} \right) 
	           \mejoin \ME
}
$$	
$$
\myinfer[match-wc]{
	\match[\ME]{x}{\ang{? : C}} = \ME
}{
	x = \obj{C}{id}{f_i = v_i}
}
\quad
\myinfer[match-wc-?]{
	\match[\ME]{x}{\ang{? : C^?}} = \ME
}{
	\match[\ME]{x}{\ang{? : C}} = \ME
}
$$
$$
\myinfer[match-null]{
	\match[\ME]{NULL}{\ang{? : C^?}} = \ME
}{
}
$$		
$$
\myinfer[match-ds]{
	\match[\ME]{x}{\obj{\TC{C}}{id'}{f_i = P_i}} 
		= \left(\MEjoin\limits_i \ME_i \right) \mejoin \ME
}{
	x = \obj{C}{id}{f_i = v_i}
	&
	\match[\ME]{v_i}{P_i} = \ME_i
}
$$
$$
\myinfer[match-list]{
	\match[\ME]{x}{P} = matchList_{\ME}(x,\, P)
}{
	x = [x_1, \ldots, x_n]
	&
	P = [P_1, \ldots, P_m]
}
$$
$$
\myinfer[match-set]{
	\match[\ME]{x}{P} = matchSet_{\ME}(x,\, P)
}{
	x = \{x_1, \ldots, x_n\}
	&
	P = \{P_1, \ldots, P_m\}
}
$$
$$
\myinfer[match-prim]{
	\match[\ME]{x}{x} 
		= \ME
}{
	x \mbox{ --- значение примитивного типа}
}
$$
	\caption{Семантика операции сопоставления с образцом}\label{MatchSem}
\end{figure}


Сопоставление переменных и отличи шаблонов в мульти-среде от значений

-- функция возвращает ассоциированный с переменной терм
$$
\begin{array}{c}
\infer[wcard]{
	(\match{x}{\wcard{?var}{\tau}}) = \meitem{var}{x}
}{RTT(x) = \tau}
\\
\infer[var]{
	(\match{x}{?var = P_x}) = \ME_x \mejoin \meitem{var}{x}
}{
\match{x}{P_x} = \ME_x
}
\\
\infer[mvar]{
	(\match{x}{?var}) = \ME \mejoin \meitem{var}{x}
}{
\meflatten{\ME}(?var) \cong x
}
\end{array}
$$

\section{Семантика применения аспектов}

\begin{Def}
\term{Аспектным правилом} (или \term{атомарным аспектом}) называется тройка
$\mathcal{R} = \langle P, T, V \rangle$, где
\begin{itemize}
\item $P$ --- образец (срез), связывающий переменные $v_1,\ldots,v_m$;
\item $T$ --- наиболее конкретный тип элемента, который может быть успешно сопоставлен с $P$;
\item $V$ --- набор \term{правил замены}, то есть кортежей $\langle v_i, t_i \rangle$, где
	\begin{itemize}
		\item $v$ --- переменная;
		\item $t$ --- шаблонное выражение, содержащее ссылки на переменные $v_1,\ldots,v_m$ как на шаблонные параметры.
	\end{itemize}
\end{itemize}
\end{Def}
\newcommand{\rapply}[2]{#1@#2}
\term{Аспект} представляет собой множество аспектных правил. Применение аспекта к грамматике сводится к последовательному применению составляющих его аспектных правил. Каждое правило применяется следующим образом: последовательно перебираются все объекты $e$ типа $T$, представленные в данной грамматике, на каждом из них происходит \term{применение} аспектного правила (обозначается $\rapply{\mathcal{R}}{e}$), определенное ниже.

\newcommand{\subst}[2]{ #1 \mapsto #2 }
\newcommand{\apply}[2]{\left( #1 \right) \triangleright #2}
Для определения применения аспектного правила, нам потребуется операция \term{подстановки}, которая заменяет одни объекты внутри модели другими. Элементарная подстановка заменяет всего один объект и обозначается $\subst{x_1}{x_2}$. Применение подстановки $\sigma$ к модели $m$ обозначается $\apply{\sigma}{m}$ и для элементарного случая $\subst{x_1}{x_2}$ определяется следующим образом: все ссылки на объект $x_1$ заменяются ссылками на $x_2$.
Неэлементарные подстановки строятся с помощью операции \term{композиции}: $\apply{\sigma_1 \sqcup \sigma_2}{m}$ соответствует последовательному применению двух подстановок $\apply{\sigma_2}{\left( \apply{\sigma_1}{m} \right)}$.

\begin{Note}
Композиция подстановок в общем случае не коммутативна, что приводит к проблеме, взаимодействия советов (advice interaction), присущей всем аспектно-ориентированным языкам (см., например, \cite{JAMI}, где эта проблема рассмотрена очень подробно). Ниже мы введем в язык аспектов некоторые расширения, которые позволят в той или иной степени справиться с этой проблемой.
\end{Note}

Теперь определим операцию применения аспектного правила.
Пусть $\match{P}{e} = \ME \neq \bot$, \\$V = \left\{\langle v_i, t_i \rangle \,|\, i = 1..m \right\}$,  $\ME(v_i) = [e^i_1, \ldots, e^i_{n_i}]$, тогда
	$$\mathcal{R}@e
		= \apply{\bigsqcup\limits_{i=1}^{m} \bigsqcup\limits_{j=1}^{n_i}
			\subst{e^i_j}{\Inst{\meflatten{\ME}}{t_i}}}{e}$$
Фактически, результат сопоставления образца преобразуется в подстановку, заменяющую все значения, сопоставленные каждой переменной, результатами разворачивания соответствующих шаблонов, где средой является уплотнение результата сопоставления.

\begin{Note}
Определение аспектного правила, которое мы ввели, непосредственно соответствует советам с ключевым словом \code{instead}, поскольку происходит только замена. Ниже мы покажем, как реализовать два других типа советов.
\end{Note}


\section{Система типов}

Применение подстановки может нарушить структуру модели, например, если объект типа \code{Symbol} (не \code{SymbolReference}) заменяется объектом типа \code{Expression}. Проблема в том, что общий супертип заменяемого и заменяющего объектов не удовлетворяет требованию мета-модели (не является подтипом \code{Symbol}). Для того, чтобы исключить такие случаи, мы используем в аспектных правилах систему типов, основные правила которой приведены на \figref{AspTypes}. Эти правила не описывают процедуры \term{вывода типов} (type inference), то есть написаны в предположении, что для каждой переменной тип указан явно. Вывод типов мы опишем ниже.
\begin{figure}[htbp]
	\centering
$$
\begin{array}{ccc}
	\infer[wcard]{ \vdash \wcard{?v}{\tau} : \tau}{}
	&\quad&
	\infer[vref]{ \vdash ?(v : \tau) : \tau}{}\\
	&&\,\\
	\infer[vdef]{ \vdash ?(v : \tau) = P : \sigma}{
		 \vdash P : \tau & \tau \preceq \sigma	
	}
	&&
	\infer[subtype]{ \vdash e : \sigma}{
		 \vdash e : \tau & \tau \preceq \sigma	
	}\\
	&&\,\\
	\multicolumn{3}{c}{
	\infer[seq]{ \vdash e_1 \, e_2 : \mathtt{Sequence}}{
		 \vdash e_1 : \mathtt{Expression} & 
		 \vdash e_2 : \mathtt{Expression} & 
	}}\\
	&&\,\\
	\multicolumn{3}{c}{
	\infer[sym]{ \vdash (N : e) : \mathtt{Symbol}}{
		 \vdash N : \mathtt{Identifier} & 
		 \vdash e : \mathtt{Expression} & 
	}}\\
	\,\\
	\multicolumn{3}{c}{
	\infer[sym]{ \vdash (N : e) : \mathtt{Symbol}}{
		 \vdash N : \mathtt{Identifier} & 
		 \vdash e : \mathtt{Expression} & 
	}}\\
\end{array}
$$
	\caption{Структурная типизация в образцах}\label{AspTypes}
\end{figure}

Как и ранее, основной интерес представляют правила, связанные с переменными, а остальные --- лишь выражают типы более сложных конструкций через типы более простых. Типизация переменных дополнительно регламентируется правилами, приведенными на \figref{VarTypes}.
\newcommand{\tcomp}{\diamond}
\begin{figure}[htbp]
	\centering
$$
	\infer{\emptyset \tcomp \tau = \tau}{}
	\quad
	\infer{\tau \tcomp \emptyset = \tau}{}
	\quad
	\infer{\tau \tcomp \tau = \tau}{}
$$
$$
	\infer{\wcard{?v}{\tau} \vdash x : \emptyset}{v \neq x}
	\quad
	\infer{?(v : \tau) \vdash x : \emptyset}{v \neq x}
	\quad
	\infer{?(v : \tau) = e \vdash x : \sigma}{v \neq x & e \vdash x : \sigma}
$$
$$
	\infer{\wcard{?v}{\tau} \vdash v : \tau}{}
	\quad
	\infer{?(v : \tau) \vdash v : \tau}{}
	\quad
	\infer{?(v : \tau_2) = e \vdash v : \tau_1 \tcomp \tau_2}{
		e \vdash v : \tau_1 & 
		\tau_1 \tcomp \tau_2 \neq \bot &
		\vdash e : \sigma & \sigma \preceq \tau_2
	}
$$
$$
	\infer[seq]{e_1 \, e_2 \vdash v : \tau_1 \tcomp \tau_2}{
		e_1 \vdash v : \tau_1 & 
		e_2 \vdash v : \tau_2 & 
		\tau_1 \tcomp \tau_2 \neq \bot
	}
$$
$$
	\infer[symbol]{N : e \vdash v : \tau_1 \tcomp \tau_2}{
		N \vdash v : \tau_1 & 
		e \vdash v : \tau_2 & 
		\tau_1 \tcomp \tau_2 \neq \bot
	}
$$
$$
	\infer[ann]{e\{a\} \vdash v : \tau_1 \tcomp \tau_2}{
		e \vdash v : \tau_1 & 
		a \vdash v : \tau_2 & 
		\tau_1 \tcomp \tau_2 \neq \bot
	}
$$
	\caption{Определение типов переменных внутри образца}\label{VarTypes}
\end{figure}
В совокупности эти правила определяют, какие типы могут быть приписаны переменным и как тип переменной определяется внутри выражения. Для нужд описания используется вспомогательная функция объединения типов $\tcomp$, которая обеспечивает поглощение специального ``типа'' $\emptyset$, который приписывается переменным, не известным в данном контексте.

Теперь мы можем выписать правило, ограничивающее соотношение типов переменных в аспектном правиле и сопоставляемых им шаблонов. Для этого нам понадобится определить функцию $\Gamma(p)$, которая по образцу $p$ строит контекст $\Gamma$, соответствующий \figref{TempTypes}. Эта функция определяется так: $\Gamma(p)$ возвращает множество всех утверждений $v : \tau$, таких что переменная $v$ определена внутри $p$, и $p \vdash v : \tau \neq \emptyset$. С использованием этой функции типизация аспектных правил выглядит так:
$$
	\infer[aspect]{
		(\mathbf{instead} \; v \, : \, t) \in Allowed(\mathcal{R})
	}{
		\mathcal{R} = \langle p, T, V \rangle &
		p \vdash v : \tau &
		\Gamma(p) \vdash t : \sigma &
		\sigma \preceq \tau
	}
$$
Здесь $Allowed(\mathcal{R})$ обозначает множество всех пар $\langle v, t\rangle$, которые разрешены к использованию в $V$. Данное правило требует, чтобы тип результата шаблонного выражения (в контексте $\Gamma(p)$) был подтипом типа соответствующей переменной. Поскольку переменную можно заменить на любой объект ее типа, не нарушив структурной корректности, все такие правила допустимы.

\begin{Lemm}
Если $\match{e}{p} \neq \bot$, $\vdash e : \tau$ и $\vdash p : \sigma$, то $\tau \preceq \sigma$.
\end{Lemm}

\begin{Lemm}
Если $\match{e}{p} = \ME \neq \bot$, $e \vdash v : \tau$ и $x \in \ME(v)$, то $\vdash x : \tau$.
\end{Lemm}

\begin{Lemm}
Если $\match{e}{p} = \ME \neq \bot$ и $\Gamma(p) \vdash t : \tau$, то $\vdash \Inst{\meflatten{\ME}}{t} : \tau$.
\end{Lemm}

\begin{Def}
Элементарная подстановка $\subst{x}{y}$ называется \term{безопасной} в мета-модели $\MM{M}$, если в данной мета-модели любая ссылка, которая может указывать на $x$, может указывать также и на $y$.

Неэлементарная подстановка называется безопасной, если она получена композицией безопасных подстановок.
\end{Def}

\begin{Th}
Если $\mathcal{R}$ таково, что $V \subseteq Allowed(\mathcal{R})$, то в результате применения $\mathcal{R}@e$ не нарушаются структурные ограничение, накладываемые мета-моделью.
\end{Th}
\begin{proof}
Согласно определению операции применения, данное утверждение можно переформулировать так: подстановка
$$
\bigsqcup\limits_{i=1}^{m} \bigsqcup\limits_{j=1}^{n_i}
			\subst{e^i_j}{\Inst{\meflatten{\ME}}{t_i}}
$$
является безопасной.
\end{proof}

\subsection{Советы ``до'' и ``после''}

Ключевые слова \code{before} и \code{after} выражаются через замену. Например, вместо ``\code{before v : t}'' можно написать ``\code{instead v : t v}''. Единственное отличие состоит в том, что \code{t} в этом случае обязано иметь тип \code{Expression}, а переменную \code{v} должно быть разрешено заменить на объект типа \code{Sequence}. Этот факт описывается следующим образом:
$$
	\infer[before]{
		(\mathbf{before} \; v \, : \, t) \in Allowed(\mathcal{R})
	}{
		\mathcal{R} = \langle p, T, V \rangle &
		p \vdash v : \tau &
		\Gamma(p) \vdash t : \mathtt{Expression} &
		\mathtt{Sequence} \preceq \tau
	}
$$
Правило для \code{after} абсолютно аналогично.

\subsection{Вывод типов переменных}

Указывать типы переменных, кроме тех, которые определяются в подстановочных знаках, не обязательно, поскольку выражение в правой части определения имеет тип само по себе, и его можно сопоставить переменной автоматически. При этом необходимо помнить, что тип переменной определяет, какими объектами ее можно заменить, поэтому нужно учитывать не только (и не столько) определение переменной, но и контекст, в котором она используется. При этом нас интересует наиболее широкий тип. Например, переменная \code{b}, определенная следующим образом:
\begin{lstlisting}
a ?b='abc' d*
\end{lstlisting}
при определении получает тип \code{Literal}, но ее контекст позволяет использовать любой объект типа \code{Expression}, поэтому именно этот тип должен быть выведен.

Общее правило для вывода типов выглядит следующим образом: переменная получает \term{начальный} тип, соответствующий правой части в ее определении, далее для каждого вхождения этой переменной в образец (включая само определение) рассматривается тип ссылки в мета-модели, по которой переменная связана с объемлющим объектом. Если этот тип шире начального, он устанавливается как текущий. Если текущий тип уже был установлен, то предварительно проверяется чтобы очередной тип не был шире текущего. Такой же метод применяется и для переменных, определенных в подстановочных знаках.

\subsection{Контроль над недетерминированным поведением}

\begin{Def}
Класс $C$ в мета-модели $\MM{M}$ называется \term{допускающим локальную замену}, если в данной мета-модели все ссылки, имеющие типом этот класс и все его подклассы и суперклассы, являются агрегирующими.
\end{Def}
Например, в целевой мета-модели \tool{Grammatic} класс \texttt{Expression} и все его подклассы допускают локальную замену. По сути, это свойство означает, что для замены объекта класса $C$ при подстановке достаточно изменить всего одну ссылку в модели, поскольку всякий объект может одновременно указывать не более одной агрегирующей ссылки.

\begin{Def}
Подстановки $\subst{x}{y}$ и $\subst{z}{w}$ называются \term{совместимыми}, если $y$ и $z$ --- разные объекты и $x$ и $w$ --- разные объекты.

Композиция совместимых подстановок называется \term{правильной}.
\end{Def}

\begin{Lemm}
Пусть класс $C$ допускает локальную замену, тогда для любого объекта $x$ этого класса, безопасной подстановки $\subst{x}{y}$ и совместимой с ней безопасной подстановкой $\sigma$, операция композиции допускает перемену мест аргументов:
$$
	\subst{x}{y} \sqcup \sigma \equiv \sigma \sqcup \subst{x}{y} 
$$
\end{Lemm}
\begin{proof}
Введем обозначения:
$$\alpha := \subst{x}{y} \sqcup \sigma$$
$$\beta := \sigma \sqcup \subst{x}{y}$$
Две подстановки эквивалентны, если их применение к одной и той же модели всегда дает один и тот же результат. Пусть модель $m$ не содержит $x$, тогда
$$\apply{\alpha}{m} = \apply{\beta}{m} = \apply{\sigma}{m},$$
поскольку из-за требования совместимости $\sigma$ не может добавить $x$ в модель.
Если $m$ содержит $x$, то на него есть не более одной ссылки, то есть объекты, которые появляются после подстановок, не имеют ссылок на $x$. Следовательно, результат обеих подстановок будет одинаков.
\end{proof}

\begin{Lemm}
Все элементарные подстановки, составляющие 
$$
\bigsqcup\limits_{i=1}^{m} \bigsqcup\limits_{j=1}^{n_i}
			\subst{e^i_j}{\Inst{\meflatten{\ME}}{t_i}}
$$
попарно совместимы.
\end{Lemm}
\begin{proof}
Результат применения шаблона всегда является новым объектом и, следовательно, не может совпадать ни с одним из объектов в левой части подстановок.
\end{proof}

\begin{Th}
Результат применения аспектного правила, оперирующего только объектами классов допускающих локальную замену, не зависит от порядка объединения элементарных подстановок.
\end{Th}
\begin{proof}
Утверждение теоремы следует из двух предыдущих лемм.
\end{proof}

Аспектные правила применяются в порядке следования в аспекте.

Если происходит замена переменной, тип которой допускает неоднозначность, выдается предупреждение.

Дополнительный контроль во время выполнения: множественность сопоставления. 

// Расширить


\chapter{Пример: расширение языка \tool{Emfatic}}

\begin{lstlisting}
specification : classifier*;
classifier
	: class
	: enum
	: data
	;
class
	: 'abstract'? 'class' NAME ('extends' <List NAME, ','>)? '{'
		feature*
	  '}'
	;
feature
	: 'attr' sign
	: 'ref' sign
	: 'val' sign
	;
sign : NAME ':' type;
type : NAME ('*' | '+' | '?')?;
enum : 'enum' NAME '{' <List NAME, ','> '}' ;
data : 'data' NAME '=' STRING ;
\end{lstlisting}

\begin{lstlisting}
template List<N, T> {
	class N {
		
	}
}

\end{lstlisting}

шаблоны

аспекты

\chapter{Сравнение с существующими подходами}

Результаты сравнения предложенного в данной главе подхода с описанными в литературе приведены таблицах \ref{TmpTable} и \ref{AspTable}. Шаблоны и аспекты рассматривались отдельно, поскольку, кроме проекта \tool{Reuseware}, ни один подход не обеспечивает расширение языков обоими механизмами композиции. Детальное сравнение предложенного в данной главе подхода с подходом проекта \tool{Reuseware} приведено в конце данного раздела.

\begin{table}[htb]
	\centering
\newcommand{\dissonly}[1]{#1}
\newcommand{\hd}[1]{{\begin{sideways}\parbox{30mm}{\tool{#1}}\end{sideways}}}
%{\begin{sideways}\parbox{15mm}{\bf text}\end{sideways}}
{\small\noindent
\begin{tabular}{|l|c|c|c|c|c|c|c|c|}
\hline 
	&\hd{Lisp}
	&\hd{MacroML}
	&\hd{Haskel TMP}
	&\hd{cpp}
	&\hd{Velocity и др.}
	&\hd{MPS}
	&\hd{Reuseware}
	&\hd{\textsl{Данная работа}}\\
\hline 
Поддержка многих языков&-&-&-&+&+&+&+&+\\
Применимо для текстовых языков&n/a&n/a&n/a&+&+&-&+&+\\
Замкнутость&+&+&+&+&+&n/a&-&+\\
Контроль корректности&n/a&+&+&-&-&+&+&+\\
Логика в теле шаблона&+&+&+&+/-&+&+&-&-\\
Используемый термин&M&M&T&M&T&T&CF&T\\
\dissonly{\hline
\multicolumn{1}{|r|}{Сумма}&3&3.5&3.5&3.5&4&3.5&3&4\\}
\hline
\end{tabular}
}

	\caption{Поддержка шаблонов}\label{TmpTable}
\end{table}

Для сравнения возможностей по поддержке шаблонов (\tabref{TmpTable}) были выбраны семь проектов. Сравнение проводилось по следующим критериям:
\begin{itemize}
	\item[(а)] наличие поддержки внешних ПОЯ; три из семи проектов базируются на конкретных языках программирования и пригодны, таким образом, только для разработки исполняемых внутренних предметно-ориентированных языков;
	\item[(б)] поддержка текстовых языков (рассматривалось только для внешних ПОЯ);
	\item[(в)] \term{замкнутость} расширенного языка --- синтаксическая интеграция механизма шаблонов с другими конструкциями языка;
	\item[(г)] контроль корректности использования шаблонов, выполняемый \emph{до} разворачивания --- эта возможность очень важна, поскольку позволяет рано обнаруживать и быстро исправлять ошибки;
	\item[(д)] поддержка в теле шаблона сложной логики (условий, циклов и т. д.).
\end{itemize}	
Отдельной строкой в таблице указано, какой термин использует данный проект: ``шаблон'' (``T'' от ``template'') или макроопределение (``M'' от ``macro''); видно, что частота употребления терминов примерно совпадает\footnote{Проект \tool{Reuseware} использует термин ``component fragment'' для обозначения того же понятия, в таблице это отражено аббревиатурой ``CF''.}.
	
Из таблицы видно, что только предложенный в данной работе подход позволяет расширять текстовые внешние ПОЯ шаблонами в замкнутой форме, при этом обеспечивая контроль корректности. Подходы основанные на конкретных языках программирования не позволяют создавать внешние ПОЯ. Препроцессор \tool{cpp} и генераторы текста (\tool{Velocity} и др.) не осуществляют контроль корректности до разворачивания шаблона. \tool{MPS} не поддерживает текстовые языки, что создает необходимость в поддержке сложными инструментами и трудности в интеграции с другими средствами разработки, а \tool{Reuseware} создает незамкнутый механизм шаблонов.

Результаты аналогичного сравнения для аспектов приведены в таблице \tabref{AspTable}.  Сравнение проводилось последующим критериям:
\begin{table}[htb]
	\centering
\newcommand{\dissonly}[1]{#1}
\newcommand{\hd}[1]{{\begin{sideways}\parbox{30mm}{\tool{#1}}\end{sideways}}}
%{\begin{sideways}\parbox{15mm}{\bf text}\end{sideways}}
{\small\noindent
\begin{tabular}{|l|c|c|c|c|c|c|c|c|c|c|c|}
\hline 
	&\hd{JAMI}
	&\hd{AspectJ}
	&\hd{XCPL}
	&\hd{Reuseware}
	&\hd{Aspect.NET}
	&\hd{POPART}
	&\hd{XAspects}
	&\hd{Van Wyk'03}
	&\hd{Stratego/XT}
	&\hd{\textsl{Данная работа}}\\
\hline
\dissonly{
\multicolumn{1}{|r|}{\it Ссылка}
	&\cite{JAMI}
	&\cite{AspectJ}
	&\cite{XCPL}
	&\cite{Reuseware}
	&\cite{Aspect.NET}
	&\cite{POPART}
	&\cite{XAspects}
	&\cite{VanWyk03}
	&\cite{Bagge06}
	&\\
\hline 
}
Поддержка многих языков&+&-&+&+&+&+&+&+&+&+\\
\quad Независимость от платф. &-&-&+&+&-&+&-&+&+&+\\
\quad Неисполняемые языки&-&-&+&+&-&-&-&+&+&+\\
\quad Автоматизация&n/a&-&-&+&+&-&+&+&-&+\\
Незнание&+&+&+&-&+&+&+&+&+&+\\
Спец. язык срезов&-&+&+&-&+&+&+&-&+&+\\
Семантика встраивания&+&+&-&+&+&+&+&+&+&+\\
%Статическая семантика&-&+&-&+&+&+&+&+&+&+\\
\dissonly{
\hline
%\multicolumn{1}{|r|}{Сумма}&3&4&5&6&6&6&6&7&7&8\\}
\multicolumn{1}{|r|}{Сумма}&2&3&4&5&5&5&5&6&6&7\\}
\hline
\end{tabular}
}

	\caption{Поддержка аспектов}\label{AspTable}
\end{table}
\begin{itemize}
	\item[(а)] поддержка рассматриваемой технологией многих языков как общего назначения, так и предметно-ориентированных; отдельно рассматривались 
		независимость от платформы (здесь под платформой понимается аппаратно-программная среда, то есть виртуальные машины (\tool{Java}, \tool{.NET}) тоже считаются платформами,
		поддержка неисполняемых языков --- ярким примером такого языка \GRM{}, поддержка аспектов в котором весьма полезна,
		автоматизация разработки --- ряд работ предлагает методики, основанные на тех или иных парадигмах программирования, не автоматизирующих разработку, а предлагающих более удобный способ ручной разработки;
	\item[(б)] поддержка аспектами незнания;
	\item[(в)] наличие специализированного языка срезов: в работе \cite{VanWyk03} срезы предлагается реализовывать как функции на языке \tool{Haskell}, что затрудняет их написание и, давая б\'{о}льшие возможности, увеличивает вероятность ошибки;
	\item[(г)] описывается ли в рамках подхода семантика встраивания советов.
\end{itemize}
Из таблицы видно, что только предложенный в данной работе подход поддерживает все указанные возможности.

Как было сказано выше, подход проекта \tool{Reuseware} \cite{Reuseware} наиболее близок к предложенному в данной главе, поскольку он также обладает следующими важными преимуществами:
(а) обеспечивает структурную корректность результатов, (б) используем единый формализм для описания шаблонов и аспектов. Также как и предложенный подход, \tool{Reuseware} не поддерживает условных операторов и циклов в теле шаблона. Недостатками \tool{Reuseware} по сравнению с нашим подходом являются 
\begin{itemize}
\item незамкнутость шаблонов --- базовый язык шаблонов не интегрируется с расширяемым языком и требует отдельной инструментальной поддержки,
\item отсутствие незнания и языка срезов в аспектах --- точки встраивания должны быть специальным образом подготовлены для расширения и явно отмечены.
\end{itemize}

\chapter{Выводы}

Предложен автоматизированный подход к расширению предметно-ориентированных языков поддержкой композиции на основе шаблонов и аспектов, гарантирующих структурную корректность результатов композиции. Данный подход позволяет сократить сроки создания языков с поддержкой указанных механизмов композиции, поскольку данные механизмы не требуется разрабатывать вручную.

// Emfatic
\intro{Заключение}
В диссертации представлены следующие результаты:
\begin{itemize}
\item Разработан предметно-ориентированный язык описания текстового синтаксиса, реализующий механизмы композиции, основанный на использовании модулей, шаблонов и аспектов. Продемонстрирована эффективность механизмов композиции, использованных в данном языке.
\item Разработан генератор трансляторов, поддерживающий проверку типов в семантических действиях и гарантирующий отсутствие ошибок типизации в сгенерированном коде сразу для многих языков реализации.
\item Предложен метод автоматизации разработки предметно-ориентированных языков с поддержкой механизмов композиции, основанных на шаблонах и аспектов.
\item С использованием предложенного подхода созданы расширения языка \tool{Emfatic}.
\end{itemize}


\bibliographystyle{gost780s}  %% стилевой файл для оформления по ГОСТу
\providecommand{\bblsep}{Сентябрь}
\providecommand{\bblmay}{Май}
\bibliography{literature.bib}     %% имя библиографической базы (bib-файла)
%\documentclass[12pt,a4paper]{article}

\usepackage{afterpage,fullpage,amsfonts,amsthm,amssymb,amsmath,indentfirst}

\usepackage{ucs}
\usepackage[utf8]{inputenc}
\usepackage[T2A]{fontenc}

\usepackage[english,russian]{babel}

\voffset=-5mm
\rightmargin=0pt

%\input ../sty/aut14.cli
\input{aut14.cli}
\makeatletter
\def\paragraph{\@startsection{paragraph}{4}{\z@}%
{2ex plus 0.2ex minus.1ex}{-1em}{\reset@font\bf}}
 \makeatother
\date{}
\catcode`@ = 11 \catcode`@ = 12
\def\hiphantom#1{\underline{\phantom{#1}}}

\sloppy

\selectlanguage{russian} \makeatletter
\renewcommand{\@oddhead}{\hfill{\large -\thepage-}\hfill}
\renewcommand{\@oddfoot}{}
\renewcommand{\@evenhead}{\hfill{\large -\thepage-}\hfill}
\renewcommand{\@evenfoot}{}
%\renewcommand{\@makefntext}[1]{\parindent=1em\noindent
% \hbox to 1.8 em {\hss\Large$^{\@thefnmark}$}#1}
\makeatother \large
%title

\newcommand{\afsection}[1]{\par \begin{center}\textbf{\MakeUppercase{#1}}\end{center}}
\newcommand{\afsubsection}[1]{\par \textbf{\underline{#1}}}

\begin{document}

Механизмы композиции в предметно-ориентированных языках

и т.д.

\thispagestyle{empty}
\newpage


\setlength{\topmargin}{-5mm} \makeatletter
\renewcommand{\@oddhead}{\hfill{\large -\thepage-}\hfill}
\renewcommand{\@oddfoot}{}
\renewcommand{\@evenhead}{\hfill{\large -\thepage-}\hfill}
\renewcommand{\@evenfoot}{}
\makeatother

\afsection{Общая характеристика работы}

\afsubsection{Актуальность темы.}
широкая тема, имена
компьютерные языки высокого уровня, разработка языков программирования ...
синтаксис
Хомский, Наур (Бэкус)
семантика Кнут
автоматизация разработки трансляторов
Ахо (Yacc), Турчин, Гуревич, Мартыненко

узкая тема
предметно-ориентированные языки -- еще большее повышение уровня абстракции
непосредственно оперируют понятиями предметной области
примеры: SQL, make, graphviz, \TeX.
порождающее программирование: краткость, исключение многих ошибок
вручную создаются языки в таких областях как описание трансляторов, бизнес-процессов, компонентного и сервис-ориентированного ПО, безопасности, веб-приложений и т.д.

мотивация специфичной задачи
чтобы использовать ПОЯ было выгодно, необходимо, чтобы их разработка требовала малых затрат времени и денег. Это мотивирует автоматизацию разработки. Причем важно, чтобы получающиеся языки были удобны в использовании, иначе выгода от повышения уровня абстракции может быть сведена на нет неудобством языка. Общая цель использования ПОЯ --- повысить качество ПО, поэтому спецификации, написанные с использованием ПОЯ должны сами отвечать таким критериям качества как модульность, повторное использование, которые обеспечиваются механизмами (де-)композиции ПО, такими как модули, полиморфизм и аспекты. Для того, чтобы ПОЯ это могли, нужно, чтобы автоматизированные средства разработки позволяли поддерживать эту функциональность без существенных затрат времени.

\afsubsection{Предметом исследования} являются механизмы композиции, пригодные для использования в предметно-ориентированных языках.

\afsubsection{Целью работы} является решение задачи, имеющей существенное значение в области автоматизации разработки предметно-ориентированных языков, а именно --- исследование и обоснование подходов и методов, позволяющих автоматически расширять предметно-ориентированные языки механизмами композиции, поддерживающими повторное использование.

\afsubsection{Задачи исследования.} Достижение поставленной цели подразумевает решение следующих задач:
\begin{itemize}
\item Сравнительный анализ механизмов композиции, используемых в современных предметно-ориентированных языках, с целью обоснования требований к средствам автоматизации.
\item Проектирование и реализация предметно-ориентированного языка для хорошо изученной области (описание текстового синтаксиса искусственных языков), поддерживающего все основные механизмы композиции в полном объеме, с целью выявления связей между этими механизмами и их характерных особенностей, влияющих на подходы к автоматизации.
\item Демонстрация адекватности разработанного языка нуждам конечных пользователей на примере описания синтаксиса языка структурных запросов к реляционным базам данных SQL и его диалектов.
\item Обобщение рассмотренных механизмов композиции в виде формализованных языковых конструкций. Описание их семантики и соответствующих систем типов.
\item Разработка алгоритмов автоматического расширения языка механизмами композиции, обоснование их корректности.
\item Демонстрация работоспособности предложенного подхода на примере языка описания объектно-ориентированных мета-моделей Emfatic.
\end{itemize}

\afsubsection{Методы исследования} включают методы инженерии программного обеспечения, анализа алгоритмов и программ, аппарат теории типов и методы количественной оценки характеристик программного обеспечения.

\afsubsection{Научная новизна} результатов работы состоит в том, что:
\begin{itemize}
\item Спроектирован и реализован предметно-ориентированный язык для описания текстового синтаксиса, поддерживающий композицию спецификаций с помощью модулей, шаблонов (типизированных макроопредений) и аспектов.
\item Предложена формализация механизмов композиции на основе шаблонов и аспектов, и доказаны свойства данной формализации, гарантирующие раннее обнаружение ошибок программиста при использовании этих механизмов.
\item Разработаны и апробированы алгоритмы, позволяющие автоматизировать расширение предметно-ориентированных языков механизмами композиции, основанными на шаблонах и аспектах.
\end{itemize}

\afsubsection{Практическую ценность} работы составляют:
\begin{itemize}
\item Разработанная библиотека, обеспечивающая трансляцию предложенного языка описания текстового синтаксиса.
\item Программные генераторы, использующие данную библиотеку, позволяющие автоматически получать трансляторы и компоненты интегрированной среды разработки.
\item Методика и алгоритмов расширения предметно-ориентированных языков механизмами композиции.
\end{itemize}

\afsubsection{На защиту выносятся следующие положения:} 
\begin{itemize}
\item Предметно-ориентированный язык для описания текстового синтаксиса, поддерживающий композицию спецификаций с помощью модулей, шаблонов и аспектов.
\item Метод, позволяющий автоматически расширить имеющееся описание синтаксиса и семантики предметно-ориентированного языка таким образом, что результирующий язык поддерживает композицию с помощью шаблонов и аспектов.
\end{itemize}

\afsubsection{Достоверность научных результатов и выводов} обеспечивается формальной строгостью описания процесса композиции языков, обоснованностью применения математического аппарата, результатами тестирования алгоритмов и программного обеспечения, а также эмпирическими данными, полученными в результате применения разработанных программных компонент.

\afsubsection{Внедрение результатов работы.} Результаты, полученные в ходе диссертационной работы, были использованы 

-- при выполнении НИР по программе УМНИК ()
-- в учебном процессе 
-- при реализации рабочего пакета "Встроенные предметно-ориентированные языки" проекта "Продуктивность разработки ПО" в рамках шестилетней программы разработок научно-исследовательского центра STACC (Тарту, Эстония)

\afsubsection{Апробация работы.} Изложенные в диссертации результаты обсуждались на 11 российских и международных научных конференциях и семинарах и школах, включая V, VI и VII всероссийскую межвузовскую научную конференцию молодых ученых (2008, 2009 и 2010 гг., Санкт-Петербург), международную научную конференцию ``Компьютерные науки и информационный технологии'' (2009 г., Саратов), международные научные школы ``Generative and Transformational Techniques in Software Engineering'' (2009 г., Брага, Португалия), ``Aspect-Oriented Software Development'' (2009 г., Нант, Франция) и ``15$^{th}$ Estonian Winter School in Computer Science'' (2010 г., Палмсе, Эстония), а так же международные семинары ``Teooriapäevad'' (2009 и 2010 гг., Эстония), семинар Лаборатории математической логики и семантики языков программирования Научно-исследовательского института кибернетики Эстонской Академии наук (2009 г., Таллинн, Эстония) и научном семинаре ``Computer Science Клуба'' при ПОМИ РАН (2009 г., Санкт-Петербург).

\afsubsection{Публикации.} По теме диссертации опубликовано пять печатных работ (из них две --- в изданиях, соответствующих требованиям ВАК РФ к кандидатским диссертациям).

-- Что считать, годы Вестника

\afsubsection{Личный вклад автора.}
???

\afsection{Содержание работы}

\afsubsection{Во введении}

\afsubsection{Первая глава}

\afsubsection{Вторя глава}

\afsubsection{Третья глава}

\afsubsection{Четвертая глава}

\afsubsection{Основные результаты диссертационной работы}

\par
\begin{center}
\afsubsection{Публикации по теме диссертационной работы}
\end{center}


\end{document}
\end{document}
