\chapter{Текстовая нотация для моделей}
Ниже нам понадобится записывать формализованные утверждения, вовлекающие мета-модели и их экземпляры. Для этого в данном разделе мы опишем нотацию, позволяющую описывать соответствующие понятия в виде текста, что удобно при записи правил вывода, например, при формализации систем типов.

\section{Элементы мета-моделей}

Для описания элементов мета-моделей в формулах мы будем использовать следующие обозначения:
%
\newcommand{\class}[4]{\mathbf{class}\; #1 \langle #2, #3, #4\rangle}%
\newcommand{\classf}[3]{\mathbf{class}\; #1 \langle #2, #3\rangle}%
\newcommand{\reference}[3]{\mathbf{#1}\langle #2 : #3\rangle}%
\newcommand{\attribute}[2]{\mathbf{attr}\langle #1 : #2\rangle}%
%
\begin{itemize}
\item Класс описывается именем $C$ и списками суперклассов $S$, ссылок $R$ и атрибутов $A$: $\class{C}{S}{R}{A}$. Иногда удобно объединять атрибуты и ссылки в единый список структурных элементов $F$: $\classf{C}{S}{F}$.
\item Атрибут или ссылка описываются именем и типом: $\attribute{x}{\tau}$.
\end{itemize}
%
\newcommand{\type}[2]{#1\left(#2\right)}%
\newcommand{\valts}{\mathbf{val}}%
\newcommand{\valt}[1]{\type{\valts}{#1}}%
\newcommand{\refts}{\mathbf{ref}}%
\newcommand{\reft}[1]{\type{\refts}{#1}}%
%
Типы структурных элементов имеют следующий вид:
\begin{itemize}
\item \code{String}, \code{Integer}, \code{Boolean} или \code{Character} --- имя примитивного типа, используется только для атрибутов;
\item $\reft{C}$ --- неагрегирующая ссылка на класс $C$;
\item $\valt{C}$ --- агрегирующая ссылка на класс $C$;
\item $\tau^?$ --- ссылка может иметь значение $NULL$, атрибут может быть не указан;
\item $\tau^*$ --- коллекция может содержать ноль или более элементов;
\item $\tau^+$ --- коллекция может содержать один или более элементов.
\end{itemize}
%
\section{Элементы моделей}

\newcommand{\obj}[3]{#1@#2\left\{#3\right\}}%
\newcommand{\refv}[2]{\mathbf{ref}\; #1@#2}%
%
Нотация для элементов моделей несколько более сложна, поэтому мы сначала приводим ее компактное описание в виде контекстно-свободной грамматики в \lstref{modelGrammar}, а затем поясняем.

Общий вид обозначения для объекта таков: 
\[\obj{C}{id}{r_i = u_i, a_i = v_i},\]
 где $C$ --- класс данного объекта, $id$ --- его уникальный идентификатор\footnote{Согласно основополагающим принципам ООП, каждый объект обладает \term{идентичностью} (identity, см. \cite{Booch}).}, $r_i = u_i$ --- ссылки и их значения, а $a_i = v_i$ --- атрибуты и их значения. Как и в случае классов, часто бывает удобно объединять атрибуты и ссылки и писать $\obj{C}{id}{f_i = v_i}$.

\begin{lstlisting}[label=modelGrammar,float=htbp,caption={Грамматика, описывающая запись моделей в виде текста}]
object : NAME '@' INT '{' (NAME '=' value ';')* '}' ;
value : attrValue | refValue;
attrValue : INT | STRING | CHAR | 'true' | 'false' 
          | 'NULL' | enumValue;
enumValue : NAME '.' NAME;
refValue : object | ref | list | set | 'NULL';
ref	: 'ref' '(' NAME '@' INT ')';
list : <Collection '[', ']'> ;
set : <Collection '{', '}'> ;

template Collection<start, end> : Production+ {
	: <?start> <?end>
	: <?start>  <List object, ','> <?end>
	: <?end> <List ref, ','> <?end>
}
\end{lstlisting}

Значениями атрибутов могут быть литералы примитивных типов (числа, строки, символы, булевские значения) или перечислений.

Значениями ссылок могут быть либо объекты (в случае агрегирующих ссылок), либо ссылочные выражения вида $\refv{C}{id}$, указывающие класс и идентификатор объекта, на который делается ссылка. Также ссылки могут иметь множественные значения --- коллекции\footnote{В принципе, множественные значения могут иметь и атрибуты, но для дальнейшего обсуждения это не важно, и мы не будем это учитывать.}, которые могут быть упорядоченными (списки) и неупорядоченными (множества).

Шаблонные выражения в правой части правил для списков (\code{list}) и множеств (\code{set}) разворачиваются следующим образом:
\begin{lstlisting}
list
	: '[' ']'
	: '[' object (',' object)* ']'
	: '[' ref (',' ref)* ']'
	;
set
	: '{' '}'
	: '{' object (',' object)* '}'
	: '{' ref (',' ref)* '}'
	;
\end{lstlisting}
Важно отметить, что коллекция может одновременно хранить либо объекты, либо ссылочные выражения, но не те и другие сразу.

На объектах определено отношение структурной эквивалентности, обозначаемое $\cong$ и заданное правилами, приведенными на \figref{cong}.
%
\begin{figure}[htbp]
	\centering
$$
\infer[obj]{
	\obj{C}{id_1}{f_i = v_i} \cong \obj{C}{id_2}{f_i = u_i}
}{
	\forall i. \; v_i \cong u_i
}
$$
$$
\infer[ref]{
	\refv{C}{id_1} \cong \refv{C}{id_2}
}{
	id_1 = id_2
}
\quad
\infer[refl]{x \cong x}{}
$$
$$
\infer[elist]{[\,] \cong [\,]}{}
\quad
\infer[list]{
	[x_1,\ldots,x_n] \cong [y_1,\ldots,y_n]
}{
	\forall i \in [1:n]. \; x_i \cong y_i
}
$$
$$
\infer[eset]{\{\} \cong \{\}}{}
\quad
\infer[set]{
	\{x_1,\ldots,x_n\} \cong \{y_1,\ldots,y_n\}
}{
	\forall i :\; x_i \cong y_{\pi(i)}
}
$$
	\caption{Отношение структурной эквивалентности объектов}\label{cong}
\end{figure}
Два объекта связаны отношением $\cong$, если они являются точными копиями друг друга, то есть отличаются только идентификаторами. В правиле \rref{set} $\pi$ обозначает некоторую перестановку из $n$ элементов, которая необходима для того, чтобы выразить неупорядоченность коллекции.
Очевидно, что $\cong$ является отношением эквивалентности.

\section{Структурные ограничения в виде системы типов}

\newcommand{\fromMM}{\MM{M} \Vdash}
Мета-модель накладывает ограничения на структуру объектов через типы и кратность ссылок и атрибутов. Эти ограничения мы записываем в виде системы типов, которая связывает мета-модель и объект отношением $\Vdash$. Запись $\fromMM x : \tau$ следует читать как ``\term{объект $x$ удовлетворяет ограничениям мета-модели $\MM{M}$ и имеет тип $\tau$}''.

Правила указанной системы типов приведены на \figref{TypesMM}. В этих правилах используется отношение ``подтип'', обозначаемого следующим образом:
$$
	\mbox{подтип} \subtype \mbox{супертип}.
$$
Это отношение является транзитивным рефлексивным замыканием минимального отношения, удовлетворяющего следующему требованию: $\classf{C}{\{S_i\}}{\_} \subtype S_i$.

\begin{figure}[htbp]
	\centering
$$
	\infer[object]{
		\fromMM \obj{C}{id}{f_i = v_i} : \valt{C}
	}{
		\classf{C}{\_}{f_i : \tau_i} \in \MM{M}&
		\fromMM v_i : \tau_i
	}
$$
$$
\infer[ref]{
	\fromMM \refv{C}{id} : C
}{}
\quad
\infer[null]{
	\fromMM NULL : \tau^?
}{}
$$
$$
\infer[elist]{
	\fromMM [\,] : \tau^*
}{}
\quad
\infer[list]{
	\fromMM [x_1, \ldots, x_n] : \type{TO}{\tau}^+
}{
	\forall i\in[1:n]. \; \fromMM x_i : \type{TO}{\tau} &
	TO \in \{\valts,\, \refts\}
}
$$
$$
\infer[eset]{
	\fromMM \{\} : \tau^*
}{}
\quad
\infer[set]{
	\fromMM \{x_1, \ldots, x_n\} : \type{TO}{\tau}^+
}{
	\forall i\in[1:n]. \; \fromMM x_i : \type{TO}{\tau} &
	TO \in \{\valts,\, \refts\}
}
$$
$$
\infer[relax]{
	\fromMM x : \tau^?
}{
	\fromMM x : \tau
}
\quad
\infer[relax^+]{
	\fromMM x : \tau^*
}{
	\fromMM x : \tau^+
}
\quad
\infer[subtype]{
	\fromMM x : \sigma
}{
	\fromMM x : \tau &
	\tau \subtype \sigma
}
$$
$$
\infer[superclass]{
	\forall i \in [1:n].\; \type{TO}{C} \subtype \type{TO}{S_i}
}{
	\classf{C}{\{S_i, \ldots, S_n\}}{\_} \in \MM{M} &
	TO \in \{\valts,\, \refts\}
}
$$
$$
\infer[enum]{
	\forall i \in [1:n]. \; \fromMM T.L_i : T
}{
	\mathbf{enum}\;T\{L_1,\ldots,L_n\} \in \MM{M}
}
\quad
\infer[bool]{
	\fromMM b : \mathtt{Boolean}
}{
	b \in \{\mathbf{true}, \mathbf{false}\}
}
$$
$$
\infer[int]{\fromMM INT : \mathtt{Integer}}{}
\quad
\infer[str]{\fromMM STRING : \mathtt{String}}{}
$$
$$
\infer[char]{\fromMM CHAR : \mathtt{Char}}{}
$$
	\caption{Структурная корректность объектов}\label{TypesMM}
\end{figure}

